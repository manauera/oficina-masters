
\chapter{Funções periódicas}

\begin{definicao}
\label{def1}
    
Uma função $f(x)$ é dita periódica se existe uma constante $T > 0$, tal que 
\begin{equation}
    f(x + T) = f(x)
\end{equation}
para qualquer $T \in \mathbb{R}$. 
\end{definicao}
Essa constante T é chamada de período da função $f(x)$. As funções periódicas 
mais comuns são $\sin{x}$, $\cos{x}$, $\tan{x}$, etc. Funções periódicas surgem
em muitas aplicações matemáticas e em problemas de física e engenharia.\\

Se plotarmos o gráfico da função $y=f(x)$ em qualquer intervalo fechado 
\mbox{$a \leq x \leq a + T$}, é possível obter o gráfico de $f(x)$ através da 
repitição periódica da porção do gráfico correspondente a \mbox{$a \leq x \leq a + T$}.
Na figura ~\ref{fig:periodExp}, temos uma função periódica de período $T=2\pi$.
\\
\begin{figure}[H]
    \begin{tikzpicture}
    \begin{axis}[
        Axis Style,
        xtick={
            -6.28318, -4.7123889, -3.14159, -1.5708, 1,
            1.5708, 3.14159, 4.7123889, 6.28318, 7.28318, 
            9.42478, 10.99558, 12.56638
        },
        xticklabels={
            $-2\pi$, $-\frac{3\pi}{2}$, $-\pi$, $-\frac{\pi}{2}$, $a$,
            $\frac{\pi}{2}$, $\pi$, $\frac{3\pi}{2}$, $2\pi$, $a+2\pi$,
            $3\pi$, $\frac{7\pi}{2}$, $4\pi$
        }]
    \addplot [mark=none, thick] {sin(deg(x)) + 1/2*sin(deg(2*x)) + 1/4*sin(deg(3*x))};

    \addplot [name path=border,
            color=gray, thick, dashed]
            coordinates {(1,0) (1,1.35) } ;
    \addplot [name path=border,
            color=gray, thick, dashed]
            coordinates {(1+2*3.14159,0) (1+2*3.14159,1.35) };
    \end{axis}  
    \end{tikzpicture} 
    \caption{Observe que $f(a) = f(a + 2\pi)$}
    \label{fig:periodExp}
\end{figure}

Se $T$ é um período da função periódica $f(x)$, então seus múltiplos $2T$, $3T$, $4T$, etc 
também são períodos da função $f(x)$. Isso é verificado facilmente ao inspecionar 
os gráficos de uma função periódica, ou pela série de igualdades:\\

\begin{equation}
\label{prop2}
    f(x) = f(x + T) = f(x + 2T) = f(x + 4T) = ...
\end{equation} 
\\
Assim, temos\\
\begin{definicao}
    Se uma função f(x) possui um período $T$, então $kT$ também é um período de
    f(x), ou seja \textbf{se um período existe, ele não é único}
\end{definicao}

Vamos mostrar que o resultado da soma de duas funções periódicas de período T
é também uma função de período T. Então, dadas as funções $f(x) = sen(x)$ e $g(x) = sen(2x)$,
seus gráficos são, respectivamente:\\
\begin{figure}[H]
    \begin{tikzpicture}
    \begin{axis}[
        Axis Style,
        xtick={
            -6.28318, -4.7123889, -3.14159, -1.5708,
            1.5708, 3.14159, 4.7123889, 6.28318, 7.85398,
            9.42478, 10.99558, 12.56638
        },
        xticklabels={
            $-2\pi$, $-\frac{3\pi}{2}$, $-\pi$, $-\frac{\pi}{2}$,
            $\frac{\pi}{2}$, $\pi$, $\frac{3\pi}{2}$, $2\pi$,
            $\frac{5\pi}{2}$, $3\pi$, $\frac{7\pi}{2}$, $4\pi$
        }]
    \addplot [mark=none, thick] {sin(deg(x))};
    \end{axis}
    \end{tikzpicture}
    \caption{$f(x)=sen(x)$}
    \label{fig:senx}
\end{figure}

\begin{figure}[H]
    \begin{tikzpicture}
    \begin{axis}[
        Axis Style,
        xtick={
            -6.28318, -4.7123889, -3.14159, -1.5708,
            1.5708, 3.14159, 4.7123889, 6.28318, 7.85398,
            9.42478, 10.99558, 12.56638
        },
        xticklabels={
            $-2\pi$, $-\frac{3\pi}{2}$, $-\pi$, $-\frac{\pi}{2}$,
            $\frac{\pi}{2}$, $\pi$, $\frac{3\pi}{2}$, $2\pi$,
            $\frac{5\pi}{2}$, $3\pi$, $\frac{7\pi}{2}$, $4\pi$
        }]
    \addplot [mark=none, thick] {sin(deg(2*x))};
    \label{sen2x}
    \end{axis}
    \end{tikzpicture}
    \caption{$f(x)=sen(2x)$}
    \label{fig:sen2x}
\end{figure}

Assim, temos duas funções periódicas de período $T = 2\pi$, vale notar que o período mínimo 
de $f(x)$, $T_f = 2\pi$, é maior que o período mínimo de $g(x)$, $T_g = \pi$, mas
que ambas as funções tem o período em comum de $T = 2\pi$. Para somar essas duas
funções, basta somar o valor de $f(x)$ para cada valor de x ao valor de $g(x)$ para 
cada valor de x. Então, teremos o seguinte:
\begin{figure}[H]
    \begin{tikzpicture}
    \begin{axis}[
        Axis Style,
        ymin=-2.5,
        ymax=2.5,
        ytick={-2,-1,0,1,2},
        yticklabels={-2,-1,0,1,2},
        xtick={
            -6.28318, -4.7123889, -3.14159, -1.5708,
            1.5708, 3.14159, 4.7123889, 6.28318, 7.85398,
            9.42478, 10.99558, 12.56638
        },
        xticklabels={
            $-2\pi$, $-\frac{3\pi}{2}$, $-\pi$, $-\frac{\pi}{2}$,
            $\frac{\pi}{2}$, $\pi$, $\frac{3\pi}{2}$, $2\pi$,
            $\frac{5\pi}{2}$, $3\pi$, $\frac{7\pi}{2}$, $4\pi$
        }]
    \addplot [mark=none, thick] {sin(deg(x)) + sin(deg(2*x))};
    \label{sen2x}
    \end{axis}
    \end{tikzpicture}
    \caption{Resultado da soma entre a função $f(x)$ e $g(x)$, uma função $h(x)$ com período $T = 2\pi$}
    \label{fig:addExp}
\end{figure}

É possível ver que, a função $f(x)$ possui um período mínimo maior que a função $g(x)$,
assim, \textbf{a função resultante é uma função de período mínimo $T = 2\pi$}, a 
subtração funciona de forma semelhante.\\

Agora podemos afirmar que a soma e a diferença de duas funções 
periódica de período T é também uma função periódica com período T, onde
T é o maior período entre as funções.\\

E quanto à multiplicação e à divisão? Podemos afirmar o mesmo?\\

A resposta é sim, funciona de forma semelhante da soma e da subtração, multiplicando
os valores $f(x)$ por $g(x)$, para todo x. Ficamos com seguinte:
\begin{figure}[H]
    \begin{tikzpicture}
    \begin{axis}[
        Axis Style,
        ymin=-2.5,
        ymax=2.5,
        ytick={-2,-1,0,1,2},
        yticklabels={-2,-1,0,1,2},
        xtick={
            -6.28318, -4.7123889, -3.14159, -1.5708,
            1.5708, 3.14159, 4.7123889, 6.28318, 7.85398,
            9.42478, 10.99558, 12.56638
        },
        xticklabels={
            $-2\pi$, $-\frac{3\pi}{2}$, $-\pi$, $-\frac{\pi}{2}$,
            $\frac{\pi}{2}$, $\pi$, $\frac{3\pi}{2}$, $2\pi$,
            $\frac{5\pi}{2}$, $3\pi$, $\frac{7\pi}{2}$, $4\pi$
        }]
    \addplot [mark=none, thick] {sin(deg(x)) * sin(deg(2*x))};
    \end{axis}
    \end{tikzpicture}
    \caption{Resultado de uma multiplicação entre a função $f(x)$ e $g(x)$, uma função $h(x)$ com período $T = 2\pi$}
    \label{fig:multExp}
\end{figure}

Por mais esquisito que a função fique, podemos observar que a função resultante
permaneceu com o período $T = 2\pi$. Dessa forma, está claro que operações 
de funções que partilham um mesmo período, terá uma função resultante com o mesmo
período.\\

\begin{definicao}
    Seja $f(x)$ e $g(x)$ duas funções periódicas com período em comum $T$, a soma, subtração,
    multiplicação e divisão das duas funções resulta em uma função periódica de cujo
    período mínimo é o maior período entre $f(x)$ e $g(x)$.
\end{definicao}

Agora, outro aspecto importante de se notar é a integral de uma função periódica.
Lembrando que podemos interpretar a integral de uma função como área, no caso 
a área de uma função periódica é a área entre a curva definida pela função $f(x)$
e o eixo-x, áreas acima do eixo-x são positivas, e abaixo do eixo-x são negativas.\\

Se observarmos a figura \ref{fig:int_area}, é possível concluir que ambas as 
áreas são iguais.a área em \textcolor{blue}{azul} e a
área em \textcolor{red}{vermelho} representam as áreas das integrais da função
periódica $f(x)=sen(4x)+sen(2x)$ de período $T=\pi$ para intervalos de tamanho $\pi$. 
\\

\begin{figure}[H]
    \begin{tikzpicture}
    \begin{axis}[
        Axis Style,
        ymin=-2.5,
        ymax=2.5,
        ytick={-2,-1,0,1,2},
        yticklabels={-2,-1,0,1,2},
        xtick={
            -6.28318, -4.7123889, -3.14159, -1.5708,
            1.5708, 3.14159, 4.7123889, 6.28318, 7.85398,
            9.42478, 10.99558, 12.56638
        },
        xticklabels={
            $-2\pi$, $-\frac{3\pi}{2}$, $-\pi$, $-\frac{\pi}{2}$,
            $\frac{\pi}{2}$, $\pi$, $\frac{3\pi}{2}$, $2\pi$,
            $\frac{5\pi}{2}$, $3\pi$, $\frac{7\pi}{2}$, $4\pi$
        }]
        \addplot[name path=A, mark=none, thick] {sin(deg(4*x)) + sin(deg(2*x))};
        \addplot[name path=B]{0};
        \addplot[blue!40] fill between[of=A and B,
            soft clip={domain=0:3.14159},];
        \addplot[red!40] fill between[of=A and B, 
            soft clip={domain=4.712385:7.853975},];
    \end{axis}
    \end{tikzpicture}
\caption{Observe que ambas as áres são iguais}
\label{fig:int_area}
\end{figure}

Disso temos a seguinte definição\\

\begin{definicao}
\label{def:functPer}
    Se f(x) é integrável em um intervalo de tamanho T,
    então é integrável em qualquer outro intervalo de tamanho T, e o valor da integral
    é o mesmo\\
    \begin{equation}
    \label{int_prop1}
        \int_a^{a+T} \! f(x) \, \mathrm{d}x = \int_b^{b+T} \! f(x) \, \mathrm{d}x.
    \end{equation}
    para qualquer a, b. \\
\end{definicao}


Daqui em diante, quando uma função $f(x)$ de período $T$ for integrável, então
ela será integrável em qualquer intervalo de tamanho $T$.