\section{Polinômios trigonométricos e séries}
Dado o período $T=2l$, considere os harmonicos\\
\begin{equation}
    a_k\cos{\dfrac{\pi kx}{l}} + b_k\sin{\dfrac{\pi kx}{l}},\text{\hspace{5pt}para k = 1,2,3,...}
\end{equation}
\\
Com frequencia $\omega_k = k\pi/l$ e períodos $T_k = \dfrac{2\pi}{\omega_k} = \dfrac{2l}{k}$. 
Uma vez que 
\begin{equation}
    T = 2l = kT_k,
\end{equation}  
\\
\textbf{o número $T=2l$ é simultaneamente o período de todos os harmonicos},
pois um múltiplo de um período é também um período (Sec 1). Então, toda soma na 
forma\\
\begin{equation}
    s_n(n) = A + \sum\limits_{k=1}^{n}(a_k\cos{\dfrac{k\pi x}{l} + b_k\sin{\dfrac{k\pi x}{l}}})
\end{equation}
\\
é uma função de período $2l$, uma vez que é uma soma de funções de período 
$2l$. Vale notar que $A$ é uma constante e não afeta a periodicidade da função,
inclusive é possível considerar que uma constante é uma função periódica, onde 
qualquer valor pode ser um período.\\
\\
Essa função $s_n(x)$ é chamada  de \textbf{polinômio trigonométrico de ordem n}(
e período $2l$).\\
\\
Por mais que seja a soma de vários harmonicos, um polinômio trigonométrico pode 
ser usado para representar uma função de natureza muito mais complexa que a 
de um harmonico. E geralmente é o caso. Escolhendo as constantes corretamente,
podemos formar funções com gráficos bem diferentes de um simples harmonico... ?
\textbf{[colocar algumas aplicações mais recentes que "gráficos"]}.
\\
Na primeira figura \ref{period_ex}, o polínomio que representa aquele gráfico é\\
\begin{equation}
    y = \sin{x} + \dfrac{1}{2}\sin{2x} + \dfrac{1}{4}\sin{3x}
\end{equation}
\\
Se colocar em algum software de plot de função, será idêntica à \ref{period_ex}.\\
\\
A \textbf{série trigonométrica infinita}\\ 
\begin{equation}
\label{serie_inf}
    f(x) = A + \sum\limits_{k=1}^{\infty}(a_k\cos{\dfrac{k\pi x}{l}} + b_k\sin{\dfrac{k\pi x}{l}})
\end{equation}
também representa uma função de período $2l$. As funções como \ref{serie_inf} podemos
ser usadas para representar fenômenos de origem muito mais complexa que um polinomio.

Sendo assim, o gráfico de uma função periódica $f(x)$ pode ser obtido através da 
sobreposição de todos os harmonicos que o compõe, i.e., pode ser representado
como uma soma de harmonicos simples.
Então a pergunta que fica é:\\
\textit{Qualquer função que tenha período 2l pode ser representado por uma soma de séries 
trigonométricas?}\\
\\
A resposta é sim, na realidade, é possível ser usado em grande gama classes de problemas!
Diversos outros fenômenos podem ser ser representado por uma série trigonométrica,
tais como......................\\ 
\\
\\  
Se\\
\begin{equation}
\label{serie_longa}
    f(x) = A + \sum\limits_{k=1}^{\infty}(a_k\cos{\dfrac{k\pi x}{l}} + b_k\sin{\dfrac{k\pi x}{l}})
\end{equation}
Então, podemos definir, por comodidade, que $\dfrac{\pi x}{l} = t$ ou que $x = \dfrac{tl}{\pi}$,
assim teremos\\
\begin{equation}
\label{serie_simples}
    g(t) = f(tl/\pi) = A + \sum\limits_{k=1}^{\infty}(a_k\cos{kt} + b_k\sin{kt})
\end{equation}
\\
onde os harmonicos dessa série tenham período $2\pi$. É possível verificar que 
se a função $f(x)$ de período $2l$ possui a expansão \ref{serie_longa}, então
a função $g(x)$ de período $2\pi$ possui a expansão \ref{serie_simples}, e que
o contrário é verdadeiro também. Por ser mais legível, usaremos a expansão 
\ref{serie_simples} e ao final faremos a tradução para o mais genérico 
\ref{serie_longa}.\\