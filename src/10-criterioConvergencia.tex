\chapter{Um critério de convergência para a série de Fourier}

Agora vamos deixar um pouco mais claro como que ficaria o critério
de convergência para uma série de Fourier quando existe finitos
pontos de descontinuidades em um intervalo.

\begin{definicao}
    A Série de Fourier de uma função semi-suave (contínua ou descontínua)
    $f(x)$ de período $2\pi$ converge para todos os valores de $x$. A soma 
    da série é igual a $f(x)$ eem todo ponto de continuidade é igual à
    \begin{equation}
    \notag
        \dfrac{f(x + 0) + f(x - 0)}{2}
    \end{equation}
    média aritmética dos limites à esquerda e à direita, em todos os 
    pontos de descontinuidades. 
\end{definicao}

A Figura \ref{fig:semiConv} seria um exemplo de convergência de uma função
semi-suave descontínua. 

\begin{figure}[H]
\begin{center}
    \begin{tikzpicture}
    \begin{axis}[
        Axis Style,
        ymin=-2,
        ymax=3,
        xmin=-10,
        xmax=15,
        ytick=\empty,
        xtick={
            -3.14159,
            3.14159,
            9.42478
        },
        xticklabels={
            $-\pi$\ \ \ \ \ \ \ ,
            $\pi$\ \ \ \ \ \ ,
            $3\pi$\ \ \ \ \ \ 
        }]
    \addplot [mark=none, thick, domain=3*pi:5*pi] {sin(deg(x/2 + 2*pi)) + 0.5 };
    \addplot [mark=none, thick, domain=pi:3*pi] {sin(deg(x/2 + pi)) + 0.5 };
    \addplot [
        color=gray, thick, dashed]
        coordinates {(-pi,-0.5) (-pi,0.4) } ;
    \addplot [
        color=gray, thick, dashed]
        coordinates {(-pi,0.58) (-pi,1.5) } ;
    \addplot [mark=none, thick, domain=-pi:pi] {sin(deg(x/2)) + 0.5 };
    \addplot [
        color=gray, thick, dashed]
        coordinates {(pi,-0.5) (pi,0.4) } ;
    \addplot [
        color=gray, thick, dashed]
        coordinates {(pi,0.58) (pi,1.5) } ;
    \addplot [mark=none, thick, domain=-3*pi:-pi] {sin(deg(x/2 - pi)) + 0.5 };
    \addplot [
        color=gray, thick, dashed]
        coordinates {(3*pi,-0.5) (3*pi,0.4) } ;
    \addplot [
        color=gray, thick, dashed]
        coordinates {(3*pi,0.58) (3*pi,1.5) } ;

\addplot[
    scatter,
    only marks,
    point meta=explicit symbolic,
    scatter/classes={
        a={mark=o}},
    ]
    table[meta=label] {
        x y label
        -3.14159 0.5 a
        3.14159 0.5 a
        9.42478 0.5 a
    };
    \end{axis}
    \end{tikzpicture}
\end{center}
\caption{A convergência de uma função semi-suave descontínua}
\label{fig:semiConv}
\end{figure}

\begin{definicao}
    Se $f(x)$ é contínuo em todos os pontos, então a série converge 
    absoluta e uniformemente.
\end{definicao}

Suponha que a função $f(x)$ é definida apenas no intervalo $[-\pi, \pi]$,
e é semi-suave nesse intervalo e contínuo nas extremidades. Como 
mencionado anteriormente na seção 7, a série de Fourier de $f(x)$ coincide
com a série de Fourier da função na qual é a extensão periódica de $f(x)$
em todo eixo-x. Mas nesse caso, tal expansão obviamente levaria a uma função 
$f(x)$ que seria semi-suave em todo eixo-x. Pontanto, o critário que 
formulado implica que a série de Fourier de $f(x)$ irá convergir em 
todos os pontos. Particularmente, para $-\pi < x < \pi$, a série 
irá convergir para $f(x)$ em pontos de continuidade e terá valor de
\begin{equation}
\notag
    \dfrac{f(x + 0) + f(x - 0)}{2}
\end{equation}
nos pontos de descontinuidades. Mas o que acontecerá para extremidades
no intervalo $[-\pi, \pi]$?

Nesse caso, dois casos são possíveis:
\begin{enumerate}
    \item $f(-\pi) = f(\pi)$. Então, a extensão periódica obviamente
    resultará em uma função na qual é contínua nos pontos onde $x = (2k + 1)\pi$
    para $k \in \mathbb{N}$. Daí, pelo critério, a série de Fourier irá
    convergir para $f(x)$ nas extremidades de $[-\pi,\pi]$.

    \item  $f(-\pi)\neq f(\pi)$. Neste caso, a extensão periódica 
    resulta em uma função que é descontínua nos pontos onde $x = (2k + 1)\pi$
    para $k \in \mathbb{N}$. Daí teremos
    \begin{equation}
    \notag
        \begin{split}
            f(-\pi - 0) = f(\pi)\text{,   } & f(-\pi + 0) = f(-\pi) \\
            f(\pi + 0) = f(-\pi)\text{,   } & f(\pi - 0) = f(\pi)             
        \end{split}
    \end{equation}
    Disso ficamos com
    \begin{equation}
    \notag
        \begin{rcases}
            \dfrac{f(-\pi + 0) + f(-\pi - 0)}{2}, \\
            \dfrac{f(\pi + 0) + f(\pi - 0)}{2}
        \end{rcases}
        =\dfrac{f(\pi) + f(-\pi)}{2}
    \end{equation}
\end{enumerate}

