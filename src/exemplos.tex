\chapter{Exemplos de expansões em séries de Fourier}
\section*{Exemplo 1}
\textit{Expandir a função }$f(x) = x^2$\textit{ definida no intervalo }$-\pi < x < \pi$\textit{ em 
Série de Fourier.}\\

A Figura \ref{fig:functIntRepEx} abaixo representa a função
junto com sua extensão periódica ao longo do eixo-x. A função extendida
é contínua e semi-suave. Portanto, pelo critério de convergência do Cap. 10,
a série de Fourier de $f(x) = x^2$ converge em todo intervalo $[-\pi,\pi]$, e
também na sua extensão periódica fora desse intervalo. Além disso, a convergência
é absoluta e uniforme. 

\begin{figure}[H]
    \begin{tikzpicture}
    \begin{axis}[
        Axis Style,
        xmin=-pi,
        ymin=-3,
        ymax=10,
        ytick style={draw=none},
        yticklabels={},
        xtick={
            -3.14159,
            3.14159,
            9.42478
        },
        xticklabels={
            $-\pi$,
            $\pi$, $3\pi$
        }]
        \addplot[name path=A, mark=none, thick, domain=-pi:pi] {x^2};
        \addplot[name path=B, mark=none, thick, domain=pi:3*pi] {x^2 - 4*pi*x + 39.5};
        \addplot[name path=C, mark=none, thick, domain=-pi:5*pi] {x^2 - 8*pi*x + 157.9};
        \addplot [name path=border,
                color=gray, thick, dashed]
                coordinates {(-pi,0) (-pi,pi*pi) } ;
        \addplot [name path=border,
                color=gray, thick, dashed]
                coordinates {(pi,0) (pi,pi*pi) } ;
        \addplot [name path=border,
                color=gray, thick, dashed]
                coordinates {(3*pi,0) (3*pi,pi*pi) };
    \end{axis}
    \end{tikzpicture} 

    \caption{A função $f(x) = x^2$ expandida para todo eixo-x}
    \label{fig:functIntRepEx}
\end{figure}

Como a função $f(x)$ é par, não precisamos obter o coeficiente $b_n$ como mostrado 
no capítulo anterior, assim, temos
\begin{equation}
\notag
\label{eq:exemplo1}
    \begin{split}
        a_0 &= \dfrac{2}{\pi} \int\limits_{0}^{\pi}x^2 dx = \dfrac{2}{\pi}\left[\dfrac{x^3}{2}\right]_{x=0}^{x=\pi} = \dfrac{2\pi^2}{3}\\
        \text{Adiante, temos o }a_n &\\
        a_n &= \dfrac{2}{\pi} \int\limits_{0}^{\pi}x^2 cos(nx) dx\\
        \text{Usando a fórmula \ref{eq:43}}&\\
        &= -\dfrac{2}{\pi}\left( \left[\dfrac{x^2sen(nx)}{n}\right]_{x=0}^{x=\pi} - \dfrac{2}{\pi n}\int_{0}^{\pi}\dfrac{2x sen(nx)}{n}dx \right)\\
        sen(n\pi)\text{ é sempre zero e integral de }& cos(nx) = \dfrac{sen(nx)}{n}\\
        &= -\dfrac{4}{\pi n}\int_{0}^{\pi}x sen(nx)dx \\
        &= -\dfrac{4}{\pi n} \left(\left[\dfrac{x cos(nx)}{n}\right]_{x=0}^{x=\pi} - \int_{0}^{\pi}\dfrac{cos(nx)}{n} \right)\\
        &= -\dfrac{4}{\pi n^2} \left(\left[x cos(nx)\right]_{x=0}^{x=\pi} - \int_{0}^{\pi}cos(nx)\right)\\
        &= -\dfrac{4}{\pi n^2} \left(\left[x cos(nx)\right]_{x=0}^{x=\pi} - \dfrac{sen(\pi n)}{n}\right)\\
        &= -\dfrac{4}{\pi n^2} \left(\left[x cos(nx)\right]_{x=0}^{x=\pi}\right)\\
        &= -\dfrac{4}{\pi n^2} \left( - \pi cos(\pi n) + 0 cos(0) \right)\\
        &= \dfrac{4}{n^2}cos(\pi n) = (-1)^n \dfrac{4}{n^2}\\
        \text{e por fim, }b_n = 0 &
    \end{split}
\end{equation}

Assim temos os coeficientes da expansão de Fourier e usando a fórmula \ref{eq:61}, temos

\begin{equation}
    \label{eq:exemplo1SF}
    x^2 = \dfrac{2\pi^2}{3} - 4\sum\limits_{n=1}^{\infty}(-1)^n \dfrac{cos(nx)}{n^2}
\end{equation}

\section*{Exemplo 2}
\textit{Expandir }$f(x) = |x|$\textit{, para }$-\pi < x < \pi$\textit{ em uma série de Fourier.}\\

A função $f(x)$ é par; a Figura \ref{fig:exemplo2} representa o gráfico da função 
expandida para todo eixo-x. A extensão periódica é contínua e semi-suave, então o critério
do Cap. 10 é aplicável. Portanto, sua série de Fourier converge para $f(x) = |x|$ em 
todo intervalo $[-\pi,\pi]$ e em sua extensão periódica, converge para todo o eixo-x.
Além disso, a convergência é absoluta e uniforme.

\begin{figure}[H]
    \begin{tikzpicture}
    \begin{axis}[
        Axis Style,
        xmin=-pi,
        ymin=-1,
        ymax=4,
        ytick style={draw=none},
        yticklabels={},
        xtick={
            -3.14159,
            3.14159,
            9.42478
        },
        xticklabels={
            $-\pi$,
            $\pi$, $3\pi$
        }]
        \addplot[name path=A, mark=none, thick, domain=-pi:pi] {abs(x)};
        \addplot[name path=B, mark=none, thick, domain=pi:3*pi] {abs(x-2*pi)};
        \addplot[name path=C, mark=none, thick, domain=3*pi:5*pi] {abs(x-4*pi)};
        \addplot [name path=border,
                color=gray, thick, dashed]
                coordinates {(-pi,0) (-pi,pi) } ;
        \addplot [name path=border,
                color=gray, thick, dashed]
                coordinates {(pi,0) (pi,pi) } ;
        \addplot [name path=border,
                color=gray, thick, dashed]
                coordinates {(3*pi,0) (3*pi,pi) };
    \end{axis}
    \end{tikzpicture} 

    \caption{A função $f(x) = |x|$ expandida para todo eixo-x}
    \label{fig:exemplo2}
\end{figure}

Uma vez que $|x| = x$ para $x \geq 0$, temos
\begin{equation}
\notag
\label{eq:exemplo2SF}
    \begin{split}
        \text{O primeiro coeficiente}&\\
        a_0 &= \dfrac{2}{\pi}\int_0^{\pi}x cos(nx)dx = \dfrac{2}{\pi} \left[\dfrac{x^2}{2}\right]_{x=0}^{x=\pi} = \pi\\
        \text{Então o n-ésimo coeficiente}&\\
        a_n &= \dfrac{2}{\pi}\int_{0}^{\pi}x cos(nx)dx\\
        &= \dfrac{2}{\pi}\left[\left(\dfrac{x sen(nx)}{n}\right)_{x=0}^{x=\pi} - \int_{0}^{\pi}\dfrac{sen(nx)}{n}dx\right]\\
        &= -\dfrac{2}{\pi n}\int_{0}^{\pi}sen(nx)dx\\
        &= \dfrac{2}{\pi n^2}[cons(nx)]_{x=0}^{x=\pi}\\
        &= \dfrac{2}{\pi n^2}[(-1)^n - 1]\\
        \text{Como é uma função par}&\\
        b_n &= 0
    \end{split}
\end{equation}

Assim temos os coeficientes da expansão de Fourier e usando a fórmula \ref{eq:61}, temos
\begin{equation}
    |x| = \dfrac{\pi}{2} - \dfrac{2}{\pi}\sum\limits_{n=1}^{\infty} \dfrac{[(-1)^n - 1]cos(nx)}{n^2}
\end{equation}

\section*{Exemplo 3}
\textit{Expandir a função }$f(x) = |sen(x)|$\textit{ em uma série de Fourier.}\\

Essa função é definida para todo x, e representa uma função contínua, semi-suave e par. 
A Figura \ref{fig:exemplo3} representa o gráfico da função. O critério do Cap. 10 é 
aplicável, e portanto $f(x) = |sen(x)|$ é em todo lugar igual a sua série de Fourier,
que é absoluta e uniformemente convergente.

\begin{figure}[H]
    \begin{tikzpicture}
    \begin{axis}[
        Axis Style,
        xmin=-pi,
        ymin=-1,
        ymax=4,
        ytick style={draw=none},
        yticklabels={},
        xtick={
            -3.14159,
            3.14159,
            9.42478
        },
        xticklabels={
            $-\pi$,
            $\pi$, $3\pi$
        }]
        \addplot[name path=A, mark=none, thick, domain=-pi:5*pi] {abs(sin(deg(x)))};
    \end{axis}
    \end{tikzpicture} 

    \caption{A função $f(x) = |sen(x)|$}
    \label{fig:exemplo3}
\end{figure}

Como $|sen(x)| = sen(x)$ para $0 \leq x \leq \pi$, então
\begin{equation}
    \notag
    \label{eq:exemplo3}
    \begin{split}
        a_0 &= \dfrac{2}{\pi} \int_{0}^{\pi}sen(x)dx = \dfrac{4}{\pi}\\
        \text{e}&\\
        a_n &= \dfrac{2}{\pi}\int_{0}^{\pi}sen(x)cos(nx)dx\\
        &= \dfrac{2}{\pi}\int_{0}^{\pi}[sen(n + 1)x - sen(n - 1)x]dx\\
        &= -\dfrac{1}{\pi}\left[\dfrac{cos(n+1)x}{n+1} - \dfrac{cos(n - 1)x}{n - 1}\right]_{x=0}^{x=\pi}\\
        &= -\dfrac{1}{\pi}\left[\dfrac{(-1)^{n+1}-1}{n+1} - \dfrac{(-1)^{n-1}-1}{n - 1}\right]\\
        &= -2\dfrac{(-1)^n + 1}{\pi(n^2 -1)}\\
        \text{para }n \neq 1\text{, para }n=1 &\\
        a_1 &= \dfrac{2}{\pi}\int_{0}^{\pi}sen(x)cos(x)dx =\dfrac{1}{\pi}\int_{0}^{\pi}sen(2x)dx = 0\\
        \text{como é uma função par, temos}&\\
        b_n &= 0
    \end{split} 
\end{equation}

Assim temos os coeficientes da expansão de Fourier e usando a fórmula \ref{eq:61}, temos
\begin{equation}
    |sen(x)| = \dfrac{2}{\pi} - \dfrac{4}{\pi}\sum\limits_{n=2}^{\infty}\dfrac{(-1)^n + 1)cos(nx)}{n^2 - 1}
\end{equation}

\section*{Exemplo 4}
\textit{Expandir }$f(x) = x$\textit{, para }$ -\pi < x < \pi$\textit{ em uma série de Fourier.}\\

A função $f(x)$ é um função \textit{ímpar}; A Figura \ref{fig:exemplo4} representa a extensão 
periódica em todo eixo-x de $f(x)$. A função extendida é semi-suave e descontínua nos pontos 
$x = (2k + 1)\pi$, para $x = 0, \pm 1, \pm 2,\ldots$. O teste do Cap. 10 é aplicável, e a série 
de Fourier de $f(x)$ converge para 0 nos pontos de descontinuidade.

\begin{figure}[H]
    \begin{tikzpicture}
    \begin{axis}[
        Axis Style,
        xmin=-pi,
        ymin=-4,
        ymax=4,
        ytick style={draw=none},
        yticklabels={},
        xtick={
            -3.14159,
            3.14159,
            9.42478
        },
        xticklabels={
            $-\pi$,
            $\pi$, $3\pi$
        }]
        \addplot[name path=A, mark=none, thick, domain=-pi:pi] {x};
        \addplot[name path=B, mark=none, thick, domain=pi:3*pi] {x-2*pi};
        \addplot[name path=C, mark=none, thick, domain=3*pi:5*pi] {x-4*pi};
        \addplot [name path=border,
                color=gray, thick, dashed]
                coordinates {(-pi,-pi) (-pi,pi) } ;
        \addplot [name path=border,
                color=gray, thick, dashed]
                coordinates {(pi,-pi) (pi,pi) } ;
        \addplot [name path=border,
                color=gray, thick, dashed]
                coordinates {(3*pi,-pi) (3*pi,pi) };
    \end{axis}
    \end{tikzpicture} 

    \caption{A função $f(x) = x$ expandida para todo eixo-x}
    \label{fig:exemplo4}
\end{figure}

Como $f(x)$ é ímpar, temos
\begin{equation}
    \notag
    \label{eq:exemplo4}
    \begin{split}
        a_n &= 0\\
        \text{e então}&\\
        b_n &= \dfrac{2}{\pi}\int_{0}^{\pi}x sen(nx)dx\\
        &= \dfrac{2}{\pi}\left[x \dfrac{cos(nx)}{n}\right]_{x=0}^{x=\pi} + \dfrac{2}{\pi}\int_{0}^{\pi}x \dfrac{cos(nx)}{n}dx\\
        &= \dfrac{2}{\pi n} \left(\pi cos(n\pi)\right) + \dfrac{2}{\pi n}\left[-\dfrac{sen(nx)}{n^2}\right]_{x=0}^{x=\pi}\\
        &= \dfrac{2}{n} cos(n\pi) = \dfrac{2}{n} (-1)^{n+1}
    \end{split}
\end{equation}

Assim temos os coeficientes da expansão de Fourier e usando a fórmula \ref{eq:61}, temos
\begin{equation}
    x = 2\sum\limits_{n=1}^{\infty}\dfrac{(-1)^{n + 1})sen(nx)}{n}
\end{equation}

\section*{Exemplo 5}
\textit{Expandir }$f(x) = 1$\textit{, para }$0 < x < \pi$\textit{ em série de senos.}\\

Fazendo a extensão ímpar de $f(x)$ no intervalo $[-\pi, 0]$ produz uma descontinuidade em 
$x = 0$. Figura \ref{fig:exemplo5} representa a extensão ímpar de $f(x)$ ao longo do eixo-x.
O critério de convergência do Cap. 10 é aplicável para essa função extendida. Portanto,
sua série de Fourier para $f(x) = 1$ para $0 < x < \pi$. Fora desse intervalo $0 < x < \pi$,
converge para a função em \ref{fig:exemplo5}, com a soma da série sendo igual a zero nos pontos
$x = kx$ para $x = 0, \pm 1, \pm 2, \ldots$.

\begin{figure}[H]
    \begin{tikzpicture}
    \begin{axis}[
        Axis Style,
        xmin=-pi,
        ymin=-4,
        ymax=4,
        ytick style={draw=none},
        yticklabels={},
        xtick={
            -3.14159,
            3.14159,
            9.42478
        },
        xticklabels={
            $-\pi$,
            $\pi$, $3\pi$
        }]
        \addplot[name path=A, mark=none, thick, domain=-pi:0] {-1};
        \addplot[name path=B, mark=none, thick, domain=0:pi] {1};
        \addplot[name path=C, mark=none, thick, domain=pi:2*pi] {-1};
        \addplot[name path=D, mark=none, thick, domain=2*pi:3*pi] {1};
        \addplot[name path=E, mark=none, thick, domain=3*pi:4*pi] {-1};
        \addplot [name path=border,
                color=gray, thick, dashed]
                coordinates {(-pi,-1) (-pi,1) } ;
        \addplot [name path=border,
                color=gray, thick, dashed]
                coordinates {(pi,-1) (pi,1) } ;
        \addplot [name path=border,
                color=gray, thick, dashed]
                coordinates {(2*pi,-1) (2*pi,1) };
        \addplot [name path=border,
                color=gray, thick, dashed]
                coordinates {(3*pi,-1) (3*pi,1) };
    \end{axis}
    \end{tikzpicture} 

    \caption{A função $f(x) = 1$ definida no intervalo $0 < x < \pi$ expandida para todo eixo-x}
    \label{fig:exemplo5}
\end{figure}

\begin{equation}
    \notag
    \label{eq:exemplo5}
    \begin{split}
        a_n &= 0\\
        \text{e então}&\\
        b_n &= \dfrac{2}{\pi}\int_{0}^{\pi}sen(nx)dx\\
        &= \dfrac{2}{\pi}\left[-\dfrac{cos(nx)}{n}\right]_{x=0}^{x=\pi}\\
        &= \dfrac{2}{\pi n} [1 - (-1)^n]
    \end{split}
\end{equation}

Assim temos os coeficientes da expansão de Fourier e usando a fórmula \ref{eq:61}, temos
\begin{equation}
    1 = \dfrac{2}{\pi}\sum\limits_{n=1}^{\infty}\dfrac{[1 - (-1)^n]}{n}
\end{equation}
para o intervalo $0 < x < \pi$.

\section*{Exemplo 6}

\textit{Expandir }$f(x) = x$\textit{, para }$0 < x < 2\pi$\textit{ em série de Fourier.}\\

Este exemplo é a mesma função do exemplo 4, porém o intervalo é diferente e sua expansão
periódica é representada pela Figura \ref{fig:exemplo6}. O critério de convergência do Cap. 
10 é aplicável. Nos pontos de descontinuidade, a série de Fourier converge para a média aritimética 
do limites à esquerda e à direita, i.e., para $\pi$, neste caso, a função não é par nem ímpar.

\begin{figure}[H]
    \begin{tikzpicture}
    \begin{axis}[
        Axis Style,
        xmin=-1,
        xmax=25,
        ymin=-4,
        ymax=7,
        ytick style={draw=none},
        yticklabels={},
        xtick={
            6.28318,
            12.56638,
            18.84954
        },
        xticklabels={
            $2\pi$, 
            $4\pi$, $6\pi$
        }]
        \addplot[name path=A, mark=none, thick, domain=0:2*pi] {x};
        \addplot[name path=B, mark=none, thick, domain=2*pi:4*pi] {x-2*pi};
        \addplot[name path=C, mark=none, thick, domain=4*pi:6*pi] {x-4*pi};
        \addplot[name path=D, mark=none, thick, domain=6*pi:8*pi] {x-6*pi};
        \addplot [name path=border,
                color=gray, thick, dashed]
                coordinates {(2*pi,0) (2*pi,2*pi) } ;
        \addplot [name path=border,
                color=gray, thick, dashed]
                coordinates {(4*pi,0) (4*pi,2*pi) } ;
        \addplot [name path=border,
                color=gray, thick, dashed]
                coordinates {(6*pi,0) (6*pi,2*pi) } ;
        \addplot [name path=border,
                color=gray, thick, dashed]
                coordinates {(8*pi,0) (8*pi,2*pi) } ;
    \end{axis}
    \end{tikzpicture} 

    \caption{A função $f(x) = x$ expandida para todo eixo-x}
    \label{fig:exemplo6}
\end{figure}

