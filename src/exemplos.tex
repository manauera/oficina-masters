\chapter{Exemplos de expansões em séries de Fourier}
\section*{Exemplo 1}
\textit{Expandir a função }$f(x) = x^2$\textit{ definida no intervalo }$-\pi < x < \pi$\textit{ em 
Série de Fourier.}\\

A Figura \ref{fig:exemplo1} abaixo representa a função
junto com sua extensão periódica ao longo do eixo-x. A função extendida
é contínua e semi-suave. Portanto, pelo critério de convergência do Cap. 10,
a série de Fourier de $f(x) = x^2$ converge em todo intervalo $[-\pi,\pi]$, e
também na sua extensão periódica fora desse intervalo. Além disso, a convergência
é absoluta e uniforme. 

\begin{figure}[H]
    \begin{tikzpicture}
    \begin{axis}[
        Axis Style,
        xmin=-pi,
        ymin=-3,
        ymax=10,
        ytick style={draw=none},
        yticklabels={},
        xtick={
            -3.14159,
            3.14159,
            9.42478
        },
        xticklabels={
            $-\pi$,
            $\pi$, $3\pi$
        }]
        \addplot[name path=A, mark=none, thick, domain=-pi:pi] {x^2};
        \addplot[name path=B, mark=none, thick, domain=pi:3*pi] {x^2 - 4*pi*x + 39.5};
        \addplot[name path=C, mark=none, thick, domain=-pi:5*pi] {x^2 - 8*pi*x + 157.9};
        \addplot [name path=border,
                color=gray, thick, dashed]
                coordinates {(-pi,0) (-pi,pi*pi) } ;
        \addplot [name path=border,
                color=gray, thick, dashed]
                coordinates {(pi,0) (pi,pi*pi) } ;
        \addplot [name path=border,
                color=gray, thick, dashed]
                coordinates {(3*pi,0) (3*pi,pi*pi) };
    \end{axis}
    \end{tikzpicture} 

    \caption{A função $f(x) = x^2$ expandida para todo eixo-x}
    \label{fig:exemplo1}
\end{figure}

Como a função $f(x)$ é par, não precisamos obter o coeficiente $b_n$ como mostrado 
no capítulo anterior, assim, temos
\begin{equation}
\notag
    \begin{split}
        a_0 &= \dfrac{2}{\pi} \int\limits_{0}^{\pi}x^2 dx = \dfrac{2}{\pi}\left[\dfrac{x^3}{2}\right]_{x=0}^{x=\pi} = \dfrac{2\pi^2}{3}\\
        \text{Adiante, temos o }a_n &\\
        a_n &= \dfrac{2}{\pi} \int\limits_{0}^{\pi}x^2 cos(nx) dx\\
        \text{Usando a fórmula \ref{eq:43}}&\\
        &= -\dfrac{2}{\pi}\left( \left[\dfrac{x^2sen(nx)}{n}\right]_{x=0}^{x=\pi} - \dfrac{2}{\pi n}\int_{0}^{\pi}\dfrac{2x sen(nx)}{n}dx \right)\\
        sen(n\pi)\text{ é sempre zero e integral de }& cos(nx) = \dfrac{sen(nx)}{n}\\
        &= -\dfrac{4}{\pi n}\int_{0}^{\pi}x sen(nx)dx \\
        &= -\dfrac{4}{\pi n} \left(\left[\dfrac{x cos(nx)}{n}\right]_{x=0}^{x=\pi} - \int_{0}^{\pi}\dfrac{cos(nx)}{n} \right)\\
        &= -\dfrac{4}{\pi n^2} \left(\left[x cos(nx)\right]_{x=0}^{x=\pi} - \int_{0}^{\pi}cos(nx)\right)\\
        &= -\dfrac{4}{\pi n^2} \left(\left[x cos(nx)\right]_{x=0}^{x=\pi} - \dfrac{sen(\pi n)}{n}\right)\\
        &= -\dfrac{4}{\pi n^2} \left(\left[x cos(nx)\right]_{x=0}^{x=\pi}\right)\\
        &= -\dfrac{4}{\pi n^2} \left( - \pi cos(\pi n) + 0 cos(0) \right)\\
        &= \dfrac{4}{n^2}cos(\pi n) = (-1)^n \dfrac{4}{n^2}\\
        \text{e por fim, }b_n = 0 &
    \end{split}
\end{equation}

Assim temos os coeficientes da expansão de Fourier e usando a fórmula \ref{eq:61}, temos

\begin{equation}
\label{eq:exemplo1SF}
    x^2 = \dfrac{\pi^2}{3} - 4\sum\limits_{n=1}^{\infty}(-1)^n \dfrac{cos(nx)}{n^2}
\end{equation}
para o intervalo $ -\pi < x < \pi$.

\section*{Exemplo 2}
\textit{Expandir }$f(x) = |x|$\textit{, para }$-\pi < x < \pi$\textit{ em uma série de Fourier.}\\

A função $f(x)$ é par; a Figura \ref{fig:exemplo2} representa o gráfico da função 
expandida para todo eixo-x. A extensão periódica é contínua e semi-suave, então o critério
do Cap. 10 é aplicável. Portanto, sua série de Fourier converge para $f(x) = |x|$ em 
todo intervalo $[-\pi,\pi]$ e em sua extensão periódica, converge para todo o eixo-x.
Além disso, a convergência é absoluta e uniforme.

\begin{figure}[H]
    \begin{tikzpicture}
    \begin{axis}[
        Axis Style,
        xmin=-pi,
        ymin=-1,
        ymax=4,
        ytick style={draw=none},
        yticklabels={},
        xtick={
            -3.14159,
            3.14159,
            9.42478
        },
        xticklabels={
            $-\pi$,
            $\pi$, $3\pi$
        }]
        \addplot[name path=A, mark=none, thick, domain=-pi:pi] {abs(x)};
        \addplot[name path=B, mark=none, thick, domain=pi:3*pi] {abs(x-2*pi)};
        \addplot[name path=C, mark=none, thick, domain=3*pi:5*pi] {abs(x-4*pi)};
        \addplot [name path=border,
                color=gray, thick, dashed]
                coordinates {(-pi,0) (-pi,pi) } ;
        \addplot [name path=border,
                color=gray, thick, dashed]
                coordinates {(pi,0) (pi,pi) } ;
        \addplot [name path=border,
                color=gray, thick, dashed]
                coordinates {(3*pi,0) (3*pi,pi) };
    \end{axis}
    \end{tikzpicture} 

    \caption{A função $f(x) = |x|$ expandida para todo eixo-x}
    \label{fig:exemplo2}
\end{figure}

Uma vez que $|x| = x$ para $x \geq 0$, temos
\begin{equation}
\notag
    \begin{split}
        \text{O primeiro coeficiente}&\\
        a_0 &= \dfrac{2}{\pi}\int_0^{\pi}x cos(nx)dx = \dfrac{2}{\pi} \left[\dfrac{x^2}{2}\right]_{x=0}^{x=\pi} = \pi\\
        \text{Então o n-ésimo coeficiente}&\\
        a_n &= \dfrac{2}{\pi}\int_{0}^{\pi}x cos(nx)dx\\
        &= \dfrac{2}{\pi}\left[\left(\dfrac{x sen(nx)}{n}\right)_{x=0}^{x=\pi} - \int_{0}^{\pi}\dfrac{sen(nx)}{n}dx\right]\\
        &= -\dfrac{2}{\pi n}\int_{0}^{\pi}sen(nx)dx\\
        &= \dfrac{2}{\pi n^2}[cons(nx)]_{x=0}^{x=\pi}\\
        &= \dfrac{2}{\pi n^2}[(-1)^n - 1]\\
        \text{Como é uma função par}&\\
        b_n &= 0
    \end{split}
\end{equation}

Assim temos os coeficientes da expansão de Fourier e usando a fórmula \ref{eq:61}, temos
\begin{equation}
\label{eq:exemplo2SF}
    |x| = \dfrac{\pi}{2} - \dfrac{2}{\pi}\sum\limits_{n=1}^{\infty} \dfrac{[(-1)^n - 1]cos(nx)}{n^2}
\end{equation}
para o intervalo $ -\pi < x < \pi$.

\section*{Exemplo 3}
\textit{Expandir a função }$f(x) = |sen(x)|$\textit{ em uma série de Fourier.}\\

Essa função é definida para todo x, e representa uma função contínua, semi-suave e par. 
A Figura \ref{fig:exemplo3} representa o gráfico da função. O critério do Cap. 10 é 
aplicável, e portanto $f(x) = |sen(x)|$ é em todo lugar igual a sua série de Fourier,
que é absoluta e uniformemente convergente.

\begin{figure}[H]
    \begin{tikzpicture}
    \begin{axis}[
        Axis Style,
        xmin=-pi,
        ymin=-1,
        ymax=4,
        ytick style={draw=none},
        yticklabels={},
        xtick={
            -3.14159,
            3.14159,
            9.42478
        },
        xticklabels={
            $-\pi$,
            $\pi$, $3\pi$
        }]
        \addplot[name path=A, mark=none, thick, domain=-pi:5*pi] {abs(sin(deg(x)))};
    \end{axis}
    \end{tikzpicture} 

    \caption{A função $f(x) = |sen(x)|$}
    \label{fig:exemplo3}
\end{figure}

Como $|sen(x)| = sen(x)$ para $0 \leq x \leq \pi$, então
\begin{equation}
    \notag
    \begin{split}
        a_0 &= \dfrac{2}{\pi} \int_{0}^{\pi}sen(x)dx = \dfrac{4}{\pi}\\
        \text{e}&\\
        a_n &= \dfrac{2}{\pi}\int_{0}^{\pi}sen(x)cos(nx)dx\\
        &= \dfrac{2}{\pi}\int_{0}^{\pi}[sen(n + 1)x - sen(n - 1)x]dx\\
        &= -\dfrac{1}{\pi}\left[\dfrac{cos(n+1)x}{n+1} - \dfrac{cos(n - 1)x}{n - 1}\right]_{x=0}^{x=\pi}\\
        &= -\dfrac{1}{\pi}\left[\dfrac{(-1)^{n+1}-1}{n+1} - \dfrac{(-1)^{n-1}-1}{n - 1}\right]\\
        &= -2\dfrac{(-1)^n + 1}{\pi(n^2 -1)}\\
        \text{para }n \neq 1\text{, para }n=1 &\\
        a_1 &= \dfrac{2}{\pi}\int_{0}^{\pi}sen(x)cos(x)dx =\dfrac{1}{\pi}\int_{0}^{\pi}sen(2x)dx = 0\\
        \text{como é uma função par, temos}&\\
        b_n &= 0
    \end{split} 
\end{equation}

Assim temos os coeficientes da expansão de Fourier e usando a fórmula \ref{eq:61}, temos
\begin{equation}
\label{eq:exemplo3SF}
    |sen(x)| = \dfrac{2}{\pi} - \dfrac{4}{\pi}\sum\limits_{n=2}^{\infty}\dfrac{(-1)^n + 1)cos(nx)}{n^2 - 1}
\end{equation}

\section*{Exemplo 4}
\textit{Expandir }$f(x) = x$\textit{, para }$ -\pi < x < \pi$\textit{ em uma série de Fourier.}\\

A função $f(x)$ é um função \textit{ímpar}; A Figura \ref{fig:exemplo4} representa a extensão 
periódica em todo eixo-x de $f(x)$. A função extendida é semi-suave e descontínua nos pontos 
$x = (2k + 1)\pi$, para $x = 0, \pm 1, \pm 2,\ldots$. O teste do Cap. 10 é aplicável, e a série 
de Fourier de $f(x)$ converge para 0 nos pontos de descontinuidade.

\begin{figure}[H]
    \begin{tikzpicture}
    \begin{axis}[
        Axis Style,
        xmin=-pi,
        ymin=-4,
        ymax=4,
        ytick style={draw=none},
        yticklabels={},
        xtick={
            -3.14159,
            3.14159,
            9.42478
        },
        xticklabels={
            $-\pi$,
            $\pi$, $3\pi$
        }]
        \addplot[name path=A, mark=none, thick, domain=-pi:pi] {x};
        \addplot[name path=B, mark=none, thick, domain=pi:3*pi] {x-2*pi};
        \addplot[name path=C, mark=none, thick, domain=3*pi:5*pi] {x-4*pi};
        \addplot [name path=border,
                color=gray, thick, dashed]
                coordinates {(-pi,-pi) (-pi,pi) } ;
        \addplot [name path=border,
                color=gray, thick, dashed]
                coordinates {(pi,-pi) (pi,pi) } ;
        \addplot [name path=border,
                color=gray, thick, dashed]
                coordinates {(3*pi,-pi) (3*pi,pi) };
                                color=gray, thick, dashed]
                coordinates {(3*pi,-1) (3*pi,1) };
                        \addplot [name path=border,
                color=gray, thick, dashed]
                coordinates {(8*pi,0) (8*pi,2*pi) } ;
                        \addplot[
            scatter,
            only marks,
            point meta=explicit symbolic,
            color=white,
            scatter/classes={
                b={mark=o, draw=black}},
            ]
            table[meta=label] {
                x y label
                -3.14159 0 a
                3.14159 0 a
                9.42478 0 a
                -3.14159 0 b
                3.14159 0 b
                9.42478 0 b
            };
    \end{axis}
    \end{tikzpicture} 

    \caption{A função $f(x) = x$ expandida para todo eixo-x}
    \label{fig:exemplo4}
\end{figure}

Como $f(x)$ é ímpar, temos
\begin{equation}
    \notag
    \begin{split}
        a_n &= 0\\
        \text{e então}&\\
        b_n &= \dfrac{2}{\pi}\int_{0}^{\pi}x sen(nx)dx\\
        &= \dfrac{2}{\pi}\left[x \dfrac{cos(nx)}{n}\right]_{x=0}^{x=\pi} + \dfrac{2}{\pi}\int_{0}^{\pi}x \dfrac{cos(nx)}{n}dx\\
        &= \dfrac{2}{\pi n} \left(\pi cos(n\pi)\right) + \dfrac{2}{\pi n}\left[-\dfrac{sen(nx)}{n^2}\right]_{x=0}^{x=\pi}\\
        &= \dfrac{2}{n} cos(n\pi) = \dfrac{2}{n} (-1)^{n+1}
    \end{split}
\end{equation}

Assim temos os coeficientes da expansão de Fourier e usando a fórmula \ref{eq:61}, temos
\begin{equation}
\label{eq:exemplo4SF}
    x = 2\sum\limits_{n=1}^{\infty}\dfrac{(-1)^{n + 1}sen(nx)}{n}
\end{equation}
para o intervalo $ -\pi < x < \pi$.


\section*{Exemplo 5}
\textit{Expandir }$f(x) = 1$\textit{, para }$0 < x < \pi$\textit{ em série de senos.}\\

Fazendo a extensão ímpar de $f(x)$ no intervalo $[-\pi, 0]$ produz uma descontinuidade em 
$x = 0$. Figura \ref{fig:exemplo5} representa a extensão ímpar de $f(x)$ ao longo do eixo-x.
O critério de convergência do Cap. 10 é aplicável para essa função extendida. Portanto,
sua série de Fourier para $f(x) = 1$ para $0 < x < \pi$. Fora desse intervalo $0 < x < \pi$,
converge para a função em \ref{fig:exemplo5}, com a soma da série sendo igual a zero nos pontos
$x = kx$ para $x = 0, \pm 1, \pm 2, \ldots$.

\begin{figure}[H]
    \begin{tikzpicture}
    \begin{axis}[
        Axis Style,
        xmin=-pi,
        ymin=-4,
        ymax=4,
        ytick style={draw=none},
        yticklabels={},
        xtick={
            -3.14159,
            3.14159,
            9.42478
        },
        xticklabels={
            $-\pi$,
            $\pi$, $3\pi$
        }]
        \addplot[name path=A, mark=none, thick, domain=-pi:0] {-1};
        \addplot[name path=B, mark=none, thick, domain=0:pi] {1};
        \addplot[name path=C, mark=none, thick, domain=pi:2*pi] {-1};
        \addplot[name path=D, mark=none, thick, domain=2*pi:3*pi] {1};
        \addplot[name path=E, mark=none, thick, domain=3*pi:4*pi] {-1};
        \addplot[name path=E, mark=none, thick, domain=4*pi:5*pi] {1};
        \addplot [name path=border,
                color=gray, thick, dashed]
                coordinates {(-pi,-1) (-pi,1) } ;
        \addplot [name path=border,
                color=gray, thick, dashed]
                coordinates {(pi,-1) (pi,1) } ;
        \addplot [name path=border,
                color=gray, thick, dashed]
                coordinates {(2*pi,-1) (2*pi,1) };
        \addplot [name path=border,
                color=gray, thick, dashed]
                coordinates {(3*pi,-1) (3*pi,1) };
        \addplot [name path=border,
                color=gray, thick, dashed]
                coordinates {(4*pi,-1) (4*pi,1) } ;
        \addplot[
            scatter,
            only marks,
            point meta=explicit symbolic,
            color=white,
            scatter/classes={
                b={mark=o, draw=black}},
            ]
            table[meta=label] {
                x y label
                -3.14159 0 a
                3.14159 0 a
                6.28318 0 a
                9.42478 0 a
                12.56638 0 a
                -3.14159 0 b
                3.14159 0 b
                6.28318 0 b
                9.42478 0 b
                12.56638 0 b
            };
    \end{axis}
    \end{tikzpicture} 

    \caption{A função $f(x) = 1$ definida no intervalo $0 < x < \pi$ expandida para todo eixo-x}
    \label{fig:exemplo5}
\end{figure}

\begin{equation}
    \notag
    \begin{split}
        a_n &= 0\\
        \text{e então}&\\
        b_n &= \dfrac{2}{\pi}\int_{0}^{\pi}sen(nx)dx\\
        &= \dfrac{2}{\pi}\left[-\dfrac{cos(nx)}{n}\right]_{x=0}^{x=\pi}\\
        &= \dfrac{2}{\pi n} [1 - (-1)^n]
    \end{split}
\end{equation}

Assim temos os coeficientes da expansão de Fourier e usando a fórmula \ref{eq:61}, temos
\begin{equation}
\label{eq:exemplo5SF}
    1 = \dfrac{2}{\pi}\sum\limits_{n=1}^{\infty}\dfrac{[1 - (-1)^n]}{n}
\end{equation}
para o intervalo $0 < x < \pi$.

\section*{Exemplo 6}

\textit{Expandir }$f(x) = x$\textit{, para }$0 < x < 2\pi$\textit{ em série de Fourier.}\\

Este exemplo é a mesma função do exemplo 4, porém o intervalo é diferente e sua expansão
periódica é representada pela Figura \ref{fig:exemplo6}. O critério de convergência do Cap. 
10 é aplicável. Nos pontos de descontinuidade, a série de Fourier converge para a média aritimética 
do limites à esquerda e à direita, i.e., para $\pi$, neste caso, a função não é par nem ímpar.

\begin{figure}[H]
    \begin{tikzpicture}
    \begin{axis}[
        Axis Style,
        xmin=-1,
        xmax=25,
        ymin=-4,
        ymax=7,
        ytick style={draw=none},
        yticklabels={},
        xtick={
            6.28318,
            12.56638,
            18.84954
        },
        xticklabels={
            $2\pi$, 
            $4\pi$, $6\pi$
        }]
        \addplot[name path=A, mark=none, thick, domain=0:2*pi] {x};
        \addplot[name path=B, mark=none, thick, domain=2*pi:4*pi] {x-2*pi};
        \addplot[name path=C, mark=none, thick, domain=4*pi:6*pi] {x-4*pi};
        \addplot[name path=D, mark=none, thick, domain=6*pi:8*pi] {x-6*pi};
        \addplot [name path=border,
                color=gray, thick, dashed]
                coordinates {(2*pi,0) (2*pi,2*pi) } ;
        \addplot [name path=border,
                color=gray, thick, dashed]
                coordinates {(4*pi,0) (4*pi,2*pi) } ;
        \addplot [name path=border,
                color=gray, thick, dashed]
                coordinates {(6*pi,0) (6*pi,2*pi) } ;
        \addplot [name path=border,
                color=gray, thick, dashed]
                coordinates {(8*pi,0) (8*pi,2*pi) } ;
                        \addplot[
            scatter,
            only marks,
            point meta=explicit symbolic,
            color=white,
            scatter/classes={
                b={mark=o, draw=black}},
            ]
            table[meta=label] {
                x y label
                6.28318 3.14159 a
                12.56638 3.14159 a
                18.84954 3.14159 a
                6.28318 3.14159 b
                12.56638 3.14159 b
                18.84954 3.14159 b
            };
    \end{axis}
    \end{tikzpicture} 

    \caption{A função $f(x) = x$ expandida para todo eixo-x}
    \label{fig:exemplo6}
\end{figure}

\begin{equation}
    \notag
    \begin{split}
        a_0 &= \dfrac{2}{2\pi} \int_{0}^{2\pi}x dx = \dfrac{1}{\pi}\left[\dfrac{x^2}{2}\right]_{x=0}^{x=2\pi}=2\pi\\
        \text{e}&\\
        a_n &= \dfrac{1}{\pi}\int_{0}^{2\pi}x cos(nx)dx\\
        &= \dfrac{1}{\pi n}[x sen(nx)]_{x=0}^{x=2\pi} - \dfrac{1}{\pi n}\int_{0}^{2\pi}sen(nx)dx = 0\\
        b_n &= \dfrac{1}{\pi}\int_{0}^{2\pi}x sen(nx)dx\\
        &= - \dfrac{1}{\pi n}[x cos(nx)]_{x=0}^{x=2\pi} + \dfrac{1}{\pi n}\int_{0}^{2\pi}cos(nx)dx = -\dfrac{2}{n}
    \end{split} 
\end{equation}

Assim temos os coeficientes da expansão de Fourier e usando a fórmula \ref{eq:61}, temos
\begin{equation}
\label{eq:exemplo6SF}
    x = \pi - 2\sum\limits_{n=1}^{\infty}\dfrac{sen(nx)}{n}
\end{equation}
para o intervalo $0 < x < 2\pi$.

\section*{Exemplo 7}
\textit{Expandir }$f(x) = x^2$\textit{, para }$0 < x < 2\pi$\textit{ em série de Fourier.}\\

Como no exemplo 1, porém em um intervalo diferente, a Figura \ref{fig:exemplo7} representa a 
extensão periódica de $f(x)$. O critério de convergência do Cap. 10 é aplicável, e nos pontos 
de descontinuidade a série converge para a média dos limites à esquerda e à direita, i.e., 
para $x = 2\pi^2$. Neste caso, a função não é par, nem ímpar.

\begin{figure}[H]
    \begin{tikzpicture}
    \begin{axis}[
        Axis Style,
        xmin=-1,
        xmax=25,
        ymin=-3,
        ymax=40,
        ytick style={draw=none},
        yticklabels={},
        xtick={
            6.28318,
            12.56638,
            18.84954
        },
        xticklabels={
            $2\pi$,
            $4\pi$, $6\pi$
        }]
        \addplot[name path=A, mark=none, thick, domain=0:2*pi] {x^2};
        \addplot[name path=B, mark=none, thick, domain=2*pi:4*pi] {x^2 - 4*pi*x + 39.5};
        \addplot[name path=C, mark=none, thick, domain=4*pi:6*pi] {x^2 - 8*pi*x + 157.9};
        \addplot[name path=D, mark=none, thick, domain=6*pi:8*pi] {x^2 - 12*pi*x + 355.3};
        \addplot [name path=border,
                color=gray, thick, dashed]
                coordinates {(2*pi,0) (2*pi,4*pi*pi) } ;
        \addplot [name path=border,
                color=gray, thick, dashed]
                coordinates {(4*pi,0) (4*pi,4*pi*pi) } ;
        \addplot [name path=border,
                color=gray, thick, dashed]
                coordinates {(6*pi,0) (6*pi,4*pi*pi) };
        \addplot[
            scatter,
            only marks,
            point meta=explicit symbolic,
            color=white,
            scatter/classes={
                b={mark=o, draw=black}},
            ]
            table[meta=label] {
                x y label
                6.28318 19.73917 a
                12.56638 19.73917 a
                18.84954 19.73917 a
                6.28318 19.73917 b
                12.56638 19.73917 b
                18.84954 19.73917 b
            };
    \end{axis}
    \end{tikzpicture} 

    \caption{A função $f(x) = x^2$ expandida para todo eixo-x}
    \label{fig:exemplo7}
\end{figure}


\begin{equation}
    \notag
    \begin{split}
        a_0 &= \dfrac{2}{2\pi} \int_{0}^{2\pi}x^2 dx = \dfrac{1}{\pi}\left[\dfrac{x^3}{3}\right]_{x=0}^{x=2\pi}=\dfrac{8\pi^2}{3}\\
        \text{e}&\\
        a_n &= \dfrac{1}{\pi}\int_{0}^{2\pi}x^2 cos(nx)dx\\
        &= - \dfrac{2}{\pi n}\int_{0}^{2\pi}x^2 sen(nx)dx\\
        &= \dfrac{2}{\pi n}[x cos(nx)]_{x=0}^{x=2\pi} - \dfrac{2}{\pi n^2}\int_{0}^{2\pi}cos(nx)dx = \dfrac{4}{n^2}\\
        b_n &= \dfrac{1}{\pi}\int_{0}^{2\pi}x^2 sen(nx)dx\\
        &= - \dfrac{1}{\pi n}[x cos(nx)]_{x=0}^{x=2\pi} + \dfrac{2}{\pi n}\int_{0}^{2\pi}x cos(nx)dx\\
        &= - \dfrac{4\pi}{n} - \dfrac{2}{\pi n^2} \int_{0}^{2\pi}sen(nx)dx = - \dfrac{4\pi}{n}
    \end{split} 
\end{equation}

Assim temos os coeficientes da expansão de Fourier e usando a fórmula \ref{eq:61}, temos
\begin{equation}
\label{eq:exemplo7SF}
    x^2 = \dfrac{4\pi^2}{3} - 4\sum\limits_{n=1}^{\infty}\dfrac{cos(nx)}{n^2} - 4\sum\limits_{n=1}^{\infty}\dfrac{sen(nx)}{n}
\end{equation}
para o intervalo $0 < x < 2\pi$.

\section*{Exemplo 8}

\textit{Expandir }$f(x) = Ax^2 + Bx + C$\textit{, para }$-\pi < x < \pi$\textit{ em série de Fourier, onde A, B e C
são constantes.}\\

O gráfico de $f(x)$ é um parábola. Pela extensão periódica, podemos obter uma função contínua ou descontínua 
dependendo da escolha das constantes A, B e C. 

Podemos calcular os coeficientes de Fourier a partir das fórmulas que foram dadas, mas não 
é necessário. É possível utilizar as expansões das funções $x^2$ e $x$ ($-\pi < x < \pi$), 
dos exemplos 1 e 4, fica assim

\begin{equation}
\notag
    Ax^2 + Bx + C = \dfrac{A\pi^2}{3} + C + 4A\sum\limits_{n=1}^{\infty}(-1)^n \dfrac{cos(nx)}{n^2} - \sum\limits_{n=1}^{\infty}\dfrac{(-1)^{n + 1}sen(nx)}{n}
\end{equation}
para o intervalo $ -\pi < x < \pi$.


\section*{Exemplo 9}
\textit{Expandir }$f(x) = Ax^2 + Bx + C$\textit{, para }$0 < x < 2\pi$\textit{ em série de Fourier, onde A, B e C
são constantes.}\\

Assim como no exemplo anterior, podemos utilizar os exemplos 6 e 7 para obter a série de Fourier 
de $f(x)$, fica assim

\begin{equation}
\notag
    Ax^2 + Bx + C = \dfrac{4A\pi^2}{3} + B\pi + C - 4A\sum\limits_{n=1}^{\infty}\dfrac{cos(nx)}{n^2} - (4\pi A - 2B)\sum\limits_{n=1}^{\infty}\dfrac{sen(nx)}{n}
\end{equation}
para o intervalo $0 < x < 2\pi$.

\section*{Inferências}

É possível utilizar os exemplos deste capítulo para calcular a soma de algumas séries trigonométricas 
importantes. Por exemplo, \ref{eq:exemplo5SF} nos dá imediatamente
\begin{equation}
\label{eq:137}
    \sum\limits_{n=1}^{\infty}\dfrac{sen(nx)}{n} = \dfrac{\pi - x}{2}\text{ }(0 < x < 2\pi)
\end{equation}
e a partir de \ref{eq:exemplo5SF} e \ref{eq:exemplo6SF}, temos
\begin{equation}
\label{eq:138}
    \sum\limits_{n=1}^{\infty}\dfrac{cos(nx)}{n^2} = \dfrac{3x^2 - 6\pi x + 2\pi^2}{12}\text{ }(0 < x < 2\pi)
\end{equation}

Uma vez que os termos da série na esquerda não excede o valor $1/n^2$ em valor absoluto, a série é
uniformemente convergente, o que significa que a soma é contínua para todo $x$ (Cap. 4). Portanto,
\ref{eq:138} é válido para $0 \leq x \leq 2\pi$, e não só $0 < x < 2\pi$.

De forma semelhante, \ref{eq:exemplo3SF} nos dá
\begin{equation}
    \label{eq:139}
    \sum\limits_{n=1}^{\infty}(-1)^{n+1}\dfrac{sen(nx)}{n} = \dfrac{x}{2}\text{ }(-\pi < x < \pi)
\end{equation}

\ref{eq:exemplo1SF} nos dá
\begin{equation}
    \label{eq:1310}
    \sum\limits_{n=0}^{\infty}(-1)^{n+1}\dfrac{cos(nx)}{n^2} = \dfrac{\pi^2 - 3x^2}{12}\text{ }(-\pi \leq x \leq \pi)
\end{equation}

\ref{eq:exemplo4SF} nos dá
\begin{equation}
    \label{eq:1311}
    \sum\limits_{n=0}^{\infty}\dfrac{cos([2n+1] x)}{(2n+1)^2} = \dfrac{\pi^2 - 2\pi x}{8}\text{ }(0 \leq x \leq \pi)
\end{equation}


Além disso, subtraindo \ref{eq:1311} de \ref{eq:137}, obtemos
\begin{equation}
    \label{eq:1313}
    \sum\limits_{n=0}^{\infty}\dfrac{sen(nx)}{2n} = \dfrac{\pi - 2x}{4}\text{ }(0 < x < \pi)
\end{equation}
e subtraindo \ref{eq:exemplo2SF} de \ref{eq:138}, temos
\begin{equation}
    \label{eq:1314}
    \sum\limits_{n=0}^{\infty}\dfrac{cos(2nx)}{4n^2} = \dfrac{6x^2 - 6\pi x + \pi^2}{24}\text{ }(0 \leq x \leq \pi)
\end{equation}

Estas fórmulas também nos permitem calcular a soma de algumas séries numéricas. 
Por exemplo, se setarmos $x = 0$, \ref{eq:138} e \ref{eq:1310} se transformam 
em
\begin{equation}
\notag
    \dfrac{\pi^2}{6} = 1 + \dfrac{1}{2^2} + \dfrac{1}{3^2} + \dfrac{1}{4^2} + \ldots, \dfrac{\pi^2}{12} = 1 - \dfrac{1}{2^2} - \dfrac{1}{3^2} - \dfrac{1}{4^2} - \ldots    
\end{equation}
enquanto se setarmos $x = \dfrac{\pi}{2}$, \ref{eq:1311} se transforma em
\begin{equation}
\notag
    \dfrac{\pi}{4} = 1 - \dfrac{1}{3} + \dfrac{1}{5} - \dfrac{1}{7} + \ldots
\end{equation}