\chapter{Exemplos de expansões em séries de Fourier}
\section*{Exemplo 1}
Expandir a função $f(x) = x^2$ definida no intervalo $-\pi < x < \pi$ em 
Série de Fourier. A Figura \ref{fig:functIntRepEx} abaixo representa a função
junto com sua extensão periódica ao longo do eixo-x. A função extendida
é contínua e semi-suave. Portanto, pelo critério de convergência do Cap. 10,
a série de Fourier de $f(x) = x^2$ converge em todo intervalo $[-\pi,\pi]$, e
também na sua extensão periódica fora desse intervalo. Além disso, a convergência
é absoluta e uniforme. 

\begin{figure}[H]
    \begin{tikzpicture}
    \begin{axis}[
        Axis Style,
        xmin=-pi,
        ymin=-3,
        ymax=10,
        ytick style={draw=none},
        yticklabels={},
        xtick={
            -3.14159,
            3.14159,
            9.42478
        },
        xticklabels={
            $-\pi$,
            $\pi$, $3\pi$
        }]
        \addplot[name path=A, mark=none, thick, domain=-pi:pi] {x^2};
        \addplot[name path=B, mark=none, thick, domain=pi:3*pi] {x^2 - 4*pi*x + 39.5};
        \addplot[name path=C, mark=none, thick, domain=-pi:5*pi] {x^2 - 8*pi*x + 157.9};
        \addplot [name path=border,
                color=gray, thick, dashed]
                coordinates {(-pi,0) (-pi,pi*pi) } ;
        \addplot [name path=border,
                color=gray, thick, dashed]
                coordinates {(pi,0) (pi,pi*pi) } ;
        \addplot [name path=border,
                color=gray, thick, dashed]
                coordinates {(3*pi,0) (3*pi,pi*pi) };
    \end{axis}
    \end{tikzpicture} 

    \caption{A mesma função expandida para todo eixo-x}
    \label{fig:functIntRepEx}
\end{figure}

Como a função $f(x)$ é par, não precisamos obter o coeficiente $b_n$ como mostrado 
no capítulo anterior, assim, temos
\begin{equation}
\notag
\label{eq:exemplo1}
    \begin{split}
        a_0 &= \dfrac{2}{\pi} \int\limits_{0}^{\pi}x^2 dx = \dfrac{2}{\pi}\left[\dfrac{x^3}{2}\right]_{x=0}^{x=\pi} = \dfrac{2\pi^2}{3}\\
        \text{Adiante, temos o }a_n &\\
        a_n &= \dfrac{2}{\pi} \int\limits_{0}^{\pi}x^2 cos(nx) dx\\
        \text{Usando a fórmula \ref{eq:43}}&\\
        &= -\dfrac{2}{\pi}\left( \left[\dfrac{x^2sen(nx)}{n}\right]_{x=0}^{x=\pi} - \dfrac{2}{\pi n}\int_{0}^{\pi}\dfrac{2x sen(nx)}{n}dx \right)\\
        sen(n\pi)\text{ é sempre zero e integral de }& cos(nx) = \dfrac{sen(nx)}{n}\\
        &= -\dfrac{4}{\pi n}\int_{0}^{\pi}x sen(nx)dx \\
        &= -\dfrac{4}{\pi n} \left(\left[\dfrac{x cos(nx)}{n}\right]_{x=0}^{x=\pi} - \int_{0}^{\pi}\dfrac{cos(nx)}{n} \right)\\
        &= -\dfrac{4}{\pi n^2} \left(\left[x cos(nx)\right]_{x=0}^{x=\pi} - \int_{0}^{\pi}cos(nx)\right)\\
        &= -\dfrac{4}{\pi n^2} \left(\left[x cos(nx)\right]_{x=0}^{x=\pi} - \dfrac{sen(\pi n)}{n}\right)\\
        &= -\dfrac{4}{\pi n^2} \left(\left[x cos(nx)\right]_{x=0}^{x=\pi}\right)\\
        &= -\dfrac{4}{\pi n^2} \left( - \pi cos(\pi n) + 0 cos(0) \right)\\
        &= \dfrac{4}{n^2}cos(\pi n) = (-1)^n \dfrac{4}{n^2}\\
        \text{e por fim, }b_n = 0 &
    \end{split}
\end{equation}

Assim temos os coeficientes da expansão de Fourier e usando a fórmula \ref{eq:61}, temos

\begin{equation}
    \label{eq:exemplo1SF}
    f(x) = \dfrac{2\pi^2}{3} - 4\sum\limits_{n=1}^{\infty}(-1)^n \dfrac{cos(nx)}{n^2}
\end{equation}