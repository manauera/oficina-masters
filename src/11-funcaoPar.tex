\chapter{Funções par e ímpar}

Seja uma função $f(x)$, definida em um intervalo ou em todo eixo-x, 
tal que ela seja simétrica em relação a origem das coordenadas.
\\

\begin{definicao}
\label{def:funcPar}
    Dizemos que uma função é \textit{par}, tal que 
    \begin{equation}
    \notag
        f(-x) = f(x)
    \end{equation} 
    para todo x.    
\end{definicao}

Por exemplo, a função $f(x) = cos(x)$ é uma função par

\begin{figure}[H]
\begin{center}
    \begin{tikzpicture}
    \begin{axis}[
        Axis Style,
        xmin=-2*pi,
        xmax=2*pi,
        width=10cm,
        ticks=none]
    \addplot [mark=none, thick, domain=-(2*pi)+0.5:(2*pi)-0.5] {sin(deg(x+pi/2))};
    \label{senoide}
    \end{axis}
    \end{tikzpicture}
    \caption{Uma função par}
    \label{fig:funcPar}
\end{center}
\end{figure}
ou seja, uma função par é qualquer função que seja simétrica em 
relação ao eixo-y
\\

Portanto, se interpretarmos a integral como sendo uma área, temos

\begin{equation}
\label{eq:intPar}
    \int_{-l}^{l} f(x)dx = 2\int_{0}^{l} f(x)dx
\end{equation}

para qualquer $l$, dado que $f(x)$ é integrável nesse intervalo.

Vale ressaltar que a equação \ref{eq:intPar} é derivada da definição \ref{def:funcPar}.
No exemplo ilustrado na Figura \ref{fig:funcPar}, a integral $\int_{0}^{l} f(x)dx = 0$,
porém existem outras funções como por exemplo $f(x) = cos^2(x)$, que é uma função par
e sua integral não é zero.

\begin{definicao}
\label{def:funcImpar}
    Dizemos que uma função é \textit{ímpar}, tal que 
    \begin{equation}
    \notag
        f(-x) = - f(x)
    \end{equation} 
    para todo x.    
\end{definicao}

Como exemplo de uma função ímpar, temos a própria senóide
\begin{figure}[H]
\begin{center}
    \begin{tikzpicture}
    \begin{axis}[
        Axis Style,
        xmin=-2*pi,
        xmax=2*pi,
        width=10cm,
        ticks=none]
    \addplot [mark=none, thick, domain=-(2*pi)+0.5:(2*pi)-0.5] {sin(deg(x))};
    \label{senoide}
    \end{axis}
    \end{tikzpicture}
    \caption{Uma função ímpar}
    \label{fig:funcImpar}
\end{center}
\end{figure}

Para a função ímpar, sua integral em um intervalo $[-l,l]$ é

\begin{equation}
\label{eq:intImpar}
    \int_{-l}^{l} f(x)dx = 0
\end{equation}

Sendo assim, podemos definir as seguintes propriedades \\
\begin{definicao}
\label{def:opPar}
    \begin{enumerate}
        \item[(a)] O produto de duas funções pares ou duas funções ímpares 
        é uma função par;
        \item[(b)] O produto entre uma função ímpar e uma função par é uma 
        função ímpar;
    \end{enumerate}
\end{definicao}

Para $(a)$, podemos verificar com as funções $f(x) = sen(x) *  sen(2x)$ e 
$g(x) = cos(x) * cos(2x)$, representados pela Figura \ref{fig:multImpar}
e pela Figura \ref{fig:multPar}, respectivamente. Para $(b)$, podemos 
verificar com a função $h(x) = sen(x) * sen(2x\pi/2)$ representado na 
Figura \ref{fig:multImparPar}


\begin{figure}[H]
\centering
\begin{subfigure}{.5\textwidth}
  \centering
    \begin{tikzpicture}
    \begin{axis}[
        Axis Style,
        xmin=-2*pi,
        xmax=2*pi,
        ymin=-3,
        ymax=3,
        width=7cm,
        ticks=none]
    \addplot [mark=none, thick, domain=-(2*pi)+0.5:(2*pi)-0.5] {sin(deg(x)) * sin(deg(2*x))};
    \end{axis}
    \end{tikzpicture}
  \caption{Gráfico de $f(x)$}
  \label{fig:multImpar}
\end{subfigure}%
\begin{subfigure}{.5\textwidth}
  \centering

    \begin{tikzpicture}
    \begin{axis}[
        Axis Style,
        xmin=-2*pi,
        xmax=2*pi,
        ymin=-3,
        ymax=3,
        width=7cm,
        ticks=none]
    \addplot [mark=none, thick, domain=-(2*pi)+0.5:(2*pi)-0.5] {sin(deg(x+pi/2)) * sin(deg(2*x+pi/2))};
    \end{axis}
    \end{tikzpicture}
  \caption{Gráfico de $g(x)$}
  \label{fig:multPar}
\end{subfigure}
\label{fig:multFunc}
\end{figure}

\begin{figure}[H]
  \centering
    \begin{tikzpicture}
    \begin{axis}[
        Axis Style,
        xmin=-2*pi,
        xmax=2*pi,
        ymin=-3,
        ymax=3,
        width=7cm,
        ticks=none]
    \addplot [mark=none, thick, domain=-(2*pi)+0.5:(2*pi)-0.5] {sin(deg(x)) * sin(deg(2*x+pi/2))};
    \end{axis}
    \end{tikzpicture}
  \caption{Gráfico de $h(x)$}
  \label{fig:multImparPar}
\end{figure}


