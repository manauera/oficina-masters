\chapter{Séries de Fourier para funções de período $2\pi$}
Suponha que a função $f(x)$ de período $2\pi$ tenha a seguinte expansão:\\
\begin{equation}
    f(x) = \dfrac{a_0}{2} + \sum\limits_{k=1}^{\infty}(a_k\cos{kx} + b_k\sin{kx})
\label{eq:61}
\end{equation}
\\
onde, para simplificar para próximas fórmulas, vamos denotar a constante da 
expansão como sendo $\dfrac{a_0}{2}$.\\
\\
Agora, vamos resolver o problema para achar os valores de $a_0, a_k, b_k$, para
\mbox{$k \in \mathbb{N}$}, por um conhecimento em $f(x)$.\\
\\
Para isso, vamos fazer a seguinte suposição:\\
\\
$\to$ A série \ref{eq:61} e a série a seguir podem ser integradas termo a termo, ou seja
a integral das somas é igual a soma das integrais.
\\
\\
Então, integrando \ref{eq:61} no intervalo $[0, 2\pi]$, ficamos com:\\
\begin{equation}
\begin{split}
    \int_{0}^{2\pi} f(x)\hspace{5pt}dx = &\\
     &\dfrac{a_0}{2}\int_{0}^{2\pi}dx + \sum\limits_{k=1}^{\infty}(a_k\int_{0}^{2\pi}\cos{kx}dx + b_k\int_{0}^{2\pi}\sin{kx}dx)
\end{split}
\end{equation}

Pela Definição \ref{def:52}, todas as integrais somem, ficando apenas com a parte 
constante:

\begin{equation}
    \int_{0}^{2\pi}f(x) dx = \pi a_0
    \label{eq:62}
\end{equation}

Agora, se multiplicarmos os dois lados por $cos(nx)$ e integrar o resultado no 
intervalo $[0, 2\pi]$ como antes, desta vez teremos:\\

\begin{equation}
\begin{split}
    \int_{0}^{2\pi} f(x)\cos{nx}\hspace{5pt}dx = &\dfrac{a_0}{2}\int_{0}^{2\pi}\cos{nx}dx + \\
    &\sum\limits_{k=1}^{\infty}(a_k\int_{0}^{2\pi}\cos{kx}\cos{nx}dx + b_k\int_{0}^{2\pi}\sin{kx}\cos{nx}dx)
\end{split}
\end{equation}
\\
Pela Definição \ref{def:52}, todas as integrais desaparecem, com exceção de uma, a 
de coeficiente $a_n$.

\begin{equation}
\notag
    \int_{0}^{2\pi}cos^2(nx)dx = \pi
\end{equation}
\\ 
E disso, temos\\
\\
\begin{equation}
\label{eq:63}
    \int_{0}^{2\pi}f(x)cos(nx)dx = a_n\pi
\end{equation}
\\
De mesmo modo\\
\\
\begin{equation}
\label{eq:64}
    \int_{0}^{2\pi}f(x)\sin{nx}dx = b_n\pi
\end{equation}
\\
Então, dado \ref{eq:63} e \ref{eq:64}, temos\\
\begin{equation}
\label{eq:65}
    \begin{split}
        a_n &= \dfrac{1}{\pi}\int_{0}^{2\pi}f(x)cos(nx)dx\\
        b_n &= \dfrac{1}{\pi}\int_{0}^{2\pi}f(x)sen(nx)dx
    \end{split}
\end{equation}
\\
Finalmente, se $f(x)$ é integrável e pode ser expandido em uma série trigonométrica,
e se essa série e a série obtida multiplicando por $\cos{nx}$ e $\sin{nx}$ ($n = 1, 2, 3, ...$)
pode ser integrada termo a termo, então os coeficientes $a_n$ e $b_n$ são dados pela
fórmula \ref{eq:65}. Estes coeficientes são conhecidos como \textit{coeficientes de Fourier}
da função $f(x)$, que representa a série trigonométrica conhecida como \textit{Série de
Fourier} de $f(x)$.\\

\begin{teorema}
    Se uma função $f(x)$ de período $2\pi$ pode ser expandida
    em uma série trigonométrica na qual converge uniformemente 
    em todo eixo-x, então essa é uma Série de Fourier de $f(x)$.
\end{teorema}
\begin{proof}
    Supondo que $f(x)$ satisfaz \ref{eq:61}, onde as séries são
    uniformemente convergente. \\
    \\
    Temos então, pelo Teorema \ref{teo:unifConv}, que $f(x)$ é
    contínuo e integrável termo a termo. Isso nos dá a fórmula
    \ref{eq:62}.\\

    Analisando a igualdade
    \begin{equation}
        f(x)cos(nx) = \dfrac{a_0 cos(nx)}{2} + \sum\limits_{k=1}^{\infty}(a_kcos(kx)cos(nx) + b_ksen(kx)cos(nx))
    \label{eq:67}
    \end{equation} 
    e mostrar que a série à direita é uniformemente convergente setando
    \begin{equation}
        s_m(x) = \dfrac{a_0}{2} + \sum\limits_{k=1}^{\infty}(a_kcos(kx)+b_ksen(kx))
    \end{equation}
    e seja $\epsilon$ um número arbitrário positivo.

    Se a série \ref{eq:61} converge uniformemente, então existe
    um número $\mathcal{N}$ tal que 
    \begin{equation}
        |f(x) - s_m(x)| \leq \epsilon
    \end{equation}
    para todo $m \geq \mathcal{N}$. 

    O produto $s_m(x)cos(nx)$ é obviamente a m-ésima soma parcial
    da série \ref{eq:67}, então a desigualdade
    \begin{equation}
        |f(x)cos(nx) - s_m(x)cos(nx)| = |f(x) - s_m(x)||cos(nx)| \leq \epsilon
    \end{equation}
    é verdade para todo $m \geq \mathcal{N}$.

    Com isso, podemos dizer que a série \ref{eq:67} converge uniformemente.
    E disso temos que a série pode ser integrada termo a termo, o resultado
    disso é a equação \ref{eq:63}. De modo semelhante, provamos a fórmula
    \ref{eq:64}.

    Finalmente, as fórmulas \ref{eq:65} são válidas para os coeficientes $a_n$
    e $b_n$, que significa que a fórmula \ref{eq:61} é a Série de Fourier de $f(x)$.
\end{proof}