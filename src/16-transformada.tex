\chapter{Transformada de Fourier}


A transformada de Fourier é um conceito simples, porém que requer bastante
atenção. Assim como qualquer outro conceito matemático, a quantidade de 
detalhes nos faz pensar que é algo de outro planeta. Nesta sessão, quero
simplificar um pouco o que é a transformada de Fourier para que seja
mais compreensível e, ao final, explicar em termos formais assustadores.


\section{Analogia}
Até agora vimos extensivamente o que é expandir uma função $f(x)$ periódica
ou apenas definida em um intervalo, em uma série de Fourier. 

Agora, vamos deixar mais claro o que é a \textit{transformada de Fourier}.
Para isso, vamo fazer uma analogia para se entender a idéia e assim,
entraremos em mais detalhes formais.\\

Então, vamos imaginar uma sopa. Isso mesmo, uma sopa.\\

Esta sopa é composta de ingredientes. Para efeitos práticos, vamos supor
que toda sopa tenha um número finito de ingredientes e que você saiba 
quais são eles.\\

Dito isso, vamos imaginar que um belo dia, você experimentou \textbf{A SOPA},
tão saborosa e cremosa que fez você queer descobrir como foi feita.\\

Pois bem, sabemos do que a sopa pode ser composta, mas não sabemos quanto de cada
ingrediente aquela sopa possui.\\

Então, se possuirmos um jeito de ``filtrar" os ingredientes e obter apenas 
suas quantidades, conseguiríamos fazer a mesma sopa! \\

Bom, para isso ser possível, vamos criar um \textit{filtro} para cada 
ingrediente e despejar a sopa em cada um desses filtros. Se despejarmos 
a sopa em cada um dos filtros, um cálculo simples de quanto tinha de massa 
antes e depois, conseguimos obter quanto tinha no total de cada ingrediente.\\

É assim que a transformada de Fourier funciona. Podemos dizer que a sopa
está para um sinal, assim como os ingredientes estão para os harmonicos 
(Definição \ref{def:harmonico}, do Cap. \ref{cap:harm}) que compõe o 
sinal. Ou seja, a transformada de Fourier é, basicamente, \textit{transformar} 
um sinal em diversos harmonicos. \\

Vamos com calma, primeiro temos que associar com aquilo que já vimos.\\

Se supormos que $f(x)$ é uma função que representa um sinal, podemos 
afirmar que ele é composto de diversos harmonicos. Como já vimos antes,
podemos calcular o coeficiente de cada harmonico separadamente, dado
que a função satisfaça todos os requisitos (são ortogonais, etc NÃO ME 
LEMBRO DE TODOS).
