\chapter{ Uma terminologia mais precisa}
Agora vamos introduzir a uma terminologia mais precisa e relembrar alguns fatos
de cálculo integral e diferencial. Quando dizemos que $f(x)$ é integrável no 
intervalo [a,b], siginifica que a integral\\
\begin{equation}
    \int_{a}^{b}f(x)dx 
\label{eq:int}
\end{equation}
(que pode ser imprópria) existe no sentido elementar. Portanto, nossas funções 
integráveis $f(x)$ sempre serão contínuas ou com finitas descontinuidades no 
intervalo [a,b], no qual a função pode ser limitada ou não.\\
\\
Em cursos de cálculo integral, primeiro se prova que a função possui um número
finito de descontinuidades dentro de um intervalo, então se a integral\\
\\
\begin{equation}
\notag
    \int_{a}^{b}|f(x)|dx 
\end{equation} 
\\
existe, então \ref{eq:int} também existe. Neste caso, a função $f(x)$ é tida como
uma função \textbf{absolutamente integrável}. (Vale notar que o inverso pode não
ser verdadeiro).\\
\\
\textbf{Propriedade 1. }
Se $f(x)$ é uma função absolutamente integrável e $g(x)$ é uma função integrável
limitada, então o produto $f(x)g(x)$ é uma função absolutamente integrável também.
\\
\\
A seguinte regra de integração por partes é válida:\\
\\
\textit{Seja f(x) e g(x) contínuas em [a,b], mas talvez não diferençiavel
em um número finito de pontos. Portanto se f'(x) e g'(x) são absolutamente 
integráveis, então temos:}\\
\begin{equation}
    \int_{a}^{b}f(x)g'(x) dx = [f(x)g(x)]_{a}^{b} - \int_{a}^{b}f'(x)g(x) dx
\end{equation}

Outro resultado familiar é o fato que se as funções $f_1(x), f_2(x), ..., f_n(x)$
são integráveis no intervalo [a,b], então a soma deles também é integrável.\\
\begin{equation}
    \int_{a}^{b}\left[\sum\limits_{k=1}^{n}f(x)\right]dx = \sum\limits_{k=1}^{n}\int_{a}^{b}f(x)dx
\label{eq:43}
\end{equation}

Agora considere uma série infinita de funções:\\
\begin{equation}
    f_1(x), f_2(x), f_3(x), ... = \sum\limits_{k=1}^{\infty}f_k(x)
\label{eq:44}
\end{equation}
\\
Uma série desse tipo é dita convergente se para um dado valor de x,
suas somas parciais:\\
\begin{equation}
    s_n(x) = \sum\limits_{k=1}^{n}f_k(x)\text{, para n = 1, 2, 3, ...}
\end{equation}
\\
tiverem um limite finito:\\
\begin{equation}
    s(x) = \lim_{x\to\infty} s_n(x)
\end{equation}
\\
$s(x)$ é dita ser a soma da série e, obviamente, é uma função de $x$.\\
\\
Se a série converge para todo x no intervalo [a,b], então sua soma $s(x)$
é definida em todo o intervalo [a,b].\\
\\
Agora perguntamos se a formula \ref{eq:43} pode ser extendida para o caso de 
uma série convergente de funções integráveis no intervalo [a,b], i.e.,
a fórmula abaixo é válida?\\
\begin{equation}
    \int_{a}^{b}\left[\sum\limits_{k=1}^{\infty}f(x)\right]dx = \int_{a}^{b}s(x) dx = \sum\limits_{k=1}^{\infty}\int_{a}^{b}f_k(x) dx
\label{eq:45}
\end{equation}
\\
Em outras palavras, a série pode ser integrada termo a termo?\\
\\
A equação \ref{eq:45} nem sempre é válida, simplesmente pela série de funções integráveis,
ou até mesmo contínua pode, se quer, ter uma soma integrável. Um problema parecido 
surge com relação a possibilidade de diferenciação termo a termo da série.
E agora vamos descartar uma importante classe de séries de função para o qual 
essas operações podem ser aplicadas.\\
\\
A série \ref{eq:44} é dita ser uniformemente convergente em um intervalo [a,b] se
para um número positivo qualquer $\varepsilon$, existe um número $N$ tal que a
desigualdade abaixo seja verdadeira para todo $n \geq N$ e para todo x no 
intervalo [a,b].\\
\begin{equation}
\notag
    |s(x) - s_n(x)| \leq \varepsilon
\end{equation}
\\
Portanto, convergência uniforme significa que para um $n$ suficientemente 
grande e para todo $x$ no intervalo, o gráfico da soma de séries $s(x)$ e o 
gráfico da soma parcial $s_n(x)$, estão $\varepsilon$ distantes uma da outra,
desse jeito, ambas as curvas estarão \textit{uniformemente perto} uma da outra.\\
\\
\textit{Importante Notar:}
Não é qualquer série que converge no intervalo [a,b] e, também, converge uniformemente
no mesmo intervalo.\\

O teste a seguir é um teste muito útil e simples para a convergencia uniforme
de uma série de funções (Weierstrass M-Test):\\
\textit{
    Se a série infinita de números \\
    \begin{equation}
        M_1 + M_2 + M_3 + ... + M_k + ...
    \end{equation}
    \\
    \\
    convergir e, se para algum x no intervalo [a,b], nós tivermos $|f_k(x)| \leq M_k$
    de um certo k em diante, então a série \ref{eq:43} converge uniformemente ( e 
    absolutamente) no intervalo [a,b].
} 
\\
\\

Os seguintes teoremas são válidos:

\begin{teorema}
    Se os termos da série \ref{eq:44} são contínuos em [a,b] e se a
    série converge uniformemente em [a,b], então:\\
    a) A soma da série é contínua\\
    b) A soma pode ser integrada termo a termo
\label{teo:unifConv}
\end{teorema}

\begin{teorema}
    Se a série \ref{eq:44} converge, e se os termos da série são diferenciáveis
    e se a série:\\
    \begin{equation}
        f_1^{'}(x) + f_2^{'}(x) + f_3^{'}(x) + ... = \sum\limits_{k=1}^{\infty}f_k^{'}(x)
    \end{equation}
    é uniformemente convergente em [a,b], então\\
    \begin{equation}
        (\sum\limits_{k=1}^{\infty} f(x))^{'} = s^{'}(x) = \sum\limits_{k=1}^{\infty}f_k^{'}(x)
    \end{equation}
    i.e., a série \ref{eq:44} pode ser diferenciada termo a termo.
\end{teorema}