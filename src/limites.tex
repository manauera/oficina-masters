\chapter{Limites à esquerda, à direita e descontinuidades}

Neste seção, vamos revisar os conceitos de limites à direita
e à esquerda, assim como introduzir a seguinte notação:
\begin{equation}
    \lim_{x\to x_0\text{, }x < x_0} = f(x_0 - 0)\hspace{10pt},\hspace{10pt}\lim_{x\to x_0\text{, }x > x_0} = f(x_0 + 0)
\label{not:limites}
\end{equation}
dado que estes limites existem e são finitos.

O primeiro destes limites é chamado de \textit{limite à esquerda}
de $f(x)$ no ponto $x_0$, e o segundo é chamado de \textit{limite à 
direita} de $f(x)$ no ponto $x_0$.

Se estes limites existem em um ponto $x_0$ onde $f(x)$ é contínuo, 
então
\begin{equation}
\notag
    f(x_0 - 0) = f(x_0) = f(x_0 + 0)
\end{equation}

Caso $x_0$ seja um ponto de descontinuidade de $f(x)$, então os 
limites \ref{not:limites} podem ou não existir (um deles ou ambos).

Se ambos os limites existem, dizemos que o ponto $x_0$ é um ponto
de \textit{descontinuidade de primeira ordem}, ou simplemente, um 
``salto descontínuo".

Se ao menos um desses limites não existir, então o ponto $x_0$ é 
dito ponto de \textit{descontinuidade de segunda ordem}.

Para este curso, estaremos apenas interessados no primeiro caso.
Então, se $x_0$ é um $salto$, então
\begin{equation}
\notag
    \delta = f(x_0 + 0) - f(x_0 - 0)
\end{equation}
é chamado de ``salto" da função $f(x)$ no ponto $x_0$.

Para ilustrar essa situação, suponha que 
\begin{equation}
\notag
    f(x)=\begin{cases}
        -x^3 &\text{ , se }x < 1\\
        0 &\text{ , se }x = 1\\
        \sqrt{x} &\text{ , se }x > 1
    \end{cases}
\end{equation}
representado pela Figura \ref{fig:saltoEx}.

\begin{figure}[H]
    \begin{center}
        \begin{tikzpicture}
        \begin{axis}[
            Axis Style,
            xmin=-2,
            xmax=3,
            width=8cm,
            xtick={
                1
            },
            xticklabels={
                \ \ \ 1
            }]
            \addplot[name path=A, mark=none, thick, domain=-2:1] {-x^3};
            \addplot [name path=border,
                    color=gray, thick, dashed]
                    coordinates {(1,-1) (1,1) } ;
            \addplot[name path=B, mark=none, thick, domain=1:3] {sqrt(x)};
            
        \end{axis}
        \end{tikzpicture}   
    \end{center}
    \caption{Descontinuidade em $x = 1$}
    \label{fig:saltoEx}
\end{figure}

Os limites à direita e à esquerda são, respectivamente:
\begin{equation}
\notag
    f(1 - 0) = -1\text{  e  }f(1 + 0) = 1
\end{equation} 

Portanto, o salto da função em $x = 1$ é
\begin{equation}
\notag
    \delta = f(1 + 0) - f(1 - 0) = 2
\end{equation} 

Disso, podemos supor o seguinte:
\\

Se $f(x)$ é uma função contínua no intervalo $[0, 2\pi]$ e $f(0) \neq f(2\pi)$,
descontinuidades aparecem ao realizar a extensão periódica de $f(x)$ de $[0, 2\pi]$
para todo o eixo-x, e todo os saltos descontínuos serão de tamanho
\begin{equation}
\notag
\begin{split}
    \delta & = f(0 + 0) - f(0 - 0)\\
    &\text{    ou}\\
    \delta & = f(2\pi + 0) - f(2\pi - 0)
\end{split}
\end{equation} 