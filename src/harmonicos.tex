\chapter{Harmonicos}

A função periódica mais simples é $y = \sin{x}$ e se imaginar constanstes $\mathcal{A} = 1$,
$\omega = 1$ e $\varphi = 0$, podemos reescrever a mesma função da seguinte forma:
\begin{equation}
\label{eq:harm}
    \senoide
\end{equation} 
\\
onde $\mathcal{A}$, $\omega$ e $\varphi$ são constantes. Essa função é chamada de função 
\textit{harmonica} de amplitude $\mathcal{A}$, frequência $\omega$ e fase
inicial $\varphi$. Neste caso, o período dessa harmonica é $T = 2\pi / \omega$
\begin{equation}
\label{harm_ex}
    \mathcal{A}\sin{\left[\omega\left(x+\dfrac{2\pi}{\omega}\right) + \varphi\right]} = \mathcal{A}\sin{[(\omega x + \varphi) + 2\pi]} = \mathcal{A}\sin{(\omega x + \varphi)}
\end{equation}


Agora, vamos definir daqui em diante $\omega > 0$, uma vez que, pela propriedade do seno, 
\mbox{$sen(-a) = - sen (a)$} e examinar o comportamento da função \ref{eq:harm} para 
diferentes valores de amplitude, frequência e fase inicial.
\\
\\


Para $\mathcal{A}=1$, $\omega = 1$ e $\varphi = 0$, temos a curva senóide comum $y = sen x$\\
\\

\begin{figure}[H]
    \begin{tikzpicture}
    \begin{axis}[
        Axis Style,
        xtick={
            -6.28318, -4.7123889, -3.14159, -1.5708,
            1.5708, 3.14159, 4.7123889, 6.28318, 7.85398,
            9.42478, 10.99558, 12.56638
        },
        xticklabels={
            $-2\pi$, $-\frac{3\pi}{2}$, $-\pi$, $-\frac{\pi}{2}$,
            $\frac{\pi}{2}$, $\pi$, $\frac{3\pi}{2}$, $2\pi$,
            $\frac{5\pi}{2}$, $3\pi$, $\frac{7\pi}{2}$, $4\pi$
        }]
    \addplot [mark=none, thick] {sin(deg(x))};
    \label{senoide}
    \end{axis}
    \end{tikzpicture}
    \caption{Uma senóide comum}
    \label{fig:senoide}
\end{figure}

Agora considere o seguinte harmonico $y = sen(wx)$ e definir $\omega x = z$, teremos
$y = sen(z)$ cujo gráfico é a curva senóide normal. Portanto, o gráfico de 
$y = sen(\omega x)$ é obtido deformando o gráfico da senóide comum. Por exemplo,
se atribuirmos um $\omega > 1$, teremos uma \textit{compressão} do gráfico da
senóide, então se tivermos $\mathcal{A} = 1$, $\omega = 3$ e $\varphi = 0$, o gráfico 
desse harmonico seria como \ref{fig:compSen} abaixo.

\begin{figure}[H]
    \begin{tikzpicture}
    \begin{axis}[
        Axis Style,
        xtick={
            -6.28318, -4.7123889, -3.14159, -1.5708,
            1.5708, 3.14159, 4.7123889, 6.28318, 7.85398,
            9.42478, 10.99558, 12.56638
        },
        xticklabels={
            $-2\pi$, $-\frac{3\pi}{2}$, $-\pi$, $-\frac{\pi}{2}$,
            $\frac{\pi}{2}$, $\pi$, $\frac{3\pi}{2}$, $2\pi$,
            $\frac{5\pi}{2}$, $3\pi$, $\frac{7\pi}{2}$, $4\pi$
        }]
    \addplot [mark=none, thick] {sin(deg(2*x))};
    \end{axis}
    \end{tikzpicture}
    \caption{Compressão de uma senóide}
    \label{fig:compSen}
\end{figure}

Ou seja, tivemos uma ``compressão" da curva senóide original ao setar $\omega = 2$,
sendo assim sempre que tivermos um $\omega > 1$, teremos uma propocionalmente
\textbf{menor}, neste caso $T = 2\pi / \omega = 2\pi / 3$.

Por outro lado, se atribuirmos um $\omega < 1$, teríamos uma \textit{expansão}
do gráfico da senóide, então se para $\mathcal{A} = 1$, $\omega = 1/4$ e $\varphi = 0$,
o gráfico seria:
\begin{figure}[H]
    \begin{tikzpicture}
    \begin{axis}[
        Axis Style,
        xtick={
            -6.28318, -4.7123889, -3.14159, -1.5708,
            1.5708, 3.14159, 4.7123889, 6.28318, 7.85398,
            9.42478, 10.99558, 12.56638
        },
        xticklabels={
            $-2\pi$, $-\frac{3\pi}{2}$, $-\pi$, $-\frac{\pi}{2}$,
            $\frac{\pi}{2}$, $\pi$, $\frac{3\pi}{2}$, $2\pi$,
            $\frac{5\pi}{2}$, $3\pi$, $\frac{7\pi}{2}$, $4\pi$
        }]
    \addplot [mark=none, thick] {sin(deg((x/2))};
    \end{axis}
    \end{tikzpicture}
    \caption{Expansão de uma senóide}
    \label{fig:expSen}
\end{figure}

Assim, teremos uma "expansão" do gráfico da curva senóide original ao setar $\omega < 1$,
e uma função com período \textbf{maior} com $\omega < 1$, neste caso 
\mbox{$T = 2\pi/\omega = \dfrac{2\pi}{1/2} = 4\pi$}.

Agora considere o harmonico $y = sen(\omega x + \varphi)$ e definir $\omega x + \varphi = \omega z$,
para que $x = z - \varphi/\omega$. Como já sabemos o gráfico de $sen(\omega z)$, o gráfico 
de $y = sen(\omega x + \varphi)$ é obtido deslocando o gráfico de $y = sen(\omega x)$ ao
longo do eixo x por $-\varphi/\omega$. Então, dado $\mathcal{A} = 1$, $\omega = 1$ e $\varphi = 1/2$,
teremos a curva que representa o $cos(x)$

\begin{figure}[H]
    \begin{tikzpicture}
    \begin{axis}[
        Axis Style,
        xtick={
            -6.28318, -4.7123889, -3.14159, -1.5708,
            1.5708, 3.14159, 4.7123889, 6.28318, 7.85398,
            9.42478, 10.99558, 12.56638
        },
        xticklabels={
            $-2\pi$, $-\frac{3\pi}{2}$, $-\pi$, $-\frac{\pi}{2}$,
            $\frac{\pi}{2}$, $\pi$, $\frac{3\pi}{2}$, $2\pi$,
            $\frac{5\pi}{2}$, $3\pi$, $\frac{7\pi}{2}$, $4\pi$
        }]
    \addplot [mark=none, thick] {sin(deg(x+(pi/2)))};
    \end{axis}
    \end{tikzpicture}
    \caption{Deslocamento da senóide}
    \label{fig:deslocSen}
\end{figure}

que nada mais é que a curva senóide deslocada para esquerda.\\
\\
Finalmente, o harmonico \senoide é obtido do harmonico $y = sen(\omega x + \varphi)$
multiplicando todas as ordenadas por $\mathcal{A}$, então dado $\mathcal{A} = 2$, $\omega = 1$ e $\varphi = 0$,
temos

\begin{figure}[H]
    \begin{tikzpicture}
    \begin{axis}[
        Axis Style,
        ymin=-2.5,
        ymax=2.5,
        ytick={-2,-1,0,1,2},
        xtick={
            -6.28318, -4.7123889, -3.14159, -1.5708,
            1.5708, 3.14159, 4.7123889, 6.28318, 7.85398,
            9.42478, 10.99558, 12.56638
        },
        xticklabels={
            $-2\pi$, $-\frac{3\pi}{2}$, $-\pi$, $-\frac{\pi}{2}$,
            $\frac{\pi}{2}$, $\pi$, $\frac{3\pi}{2}$, $2\pi$,
            $\frac{5\pi}{2}$, $3\pi$, $\frac{7\pi}{2}$, $4\pi$
        }]
    \addplot [mark=none, thick] {2*sin(deg(x+(pi/2)))};
    \end{axis}
    \end{tikzpicture}
    \caption{Ampliação da senóide}
    \label{fig:ampSen}
\end{figure}

Portanto, podemos resumir tudo isso no seguinte:\\
\begin{definicao}
    O gráfico de uma harmonica é obtido do gráfico da curva senóide 
    comum por uma compressão (ou expansão) uniforme ao longo dos eixos,
    mais um deslocamento ao longo do eixo x, e é dado pela equação \ref{eq:harm}
\end{definicao}


Assim, podemos utilizar uma conhecida fórmula matemática para derivar
o seguinte:\\
\begin{equation}
    \mathcal{A}sen(\omega x + \varphi) = \mathcal{A}(cos(\omega x)sen(\varphi) + sen(\omega x)cos(\varphi)
\end{equation}
Disso, temos\\
\begin{equation}
    a = \mathcal{A}sen(\varphi)\text{\hspace{10pt},\hspace{10pt}}b = \mathcal{A}cos(\varphi)
\label{eq:ab_harm}
\end{equation}
e então podemos dizer que todo harmonico pode ser representado na forma
\begin{equation}
    y = a \hspace{1pt}cos(\omega x) + b\hspace{1pt}sen(\omega x)
\label{eq:harmSimpl}
\end{equation}
\\
Do mesmo jeito que uma função com a forma \ref{eq:harmSimpl} é um harmonico também. 
Para provar isso, basta resolver \ref{eq:ab_harm} para $a$ e $b$. Temos
\begin{equation}
    \begin{split}
        A = \sqrt{a^2 + b^2}\hspace{5pt},\hspace{10pt} &sen(\varphi) = \dfrac{a}{A} = \dfrac{a}{\sqrt{a^2 + b^2}}\\
        e\hspace{10pt} & cos(\varphi) = \dfrac{b}{A} = \dfrac{b}{\sqrt{a^2 + b^2}}
    \end{split}
\end{equation} 
do qual $\varphi$ pode ser encontrado.\\

Assim, podemos escrever os harmonicos na forma \ref{eq:harmSimpl}. Na Figura \ref{fig:ampSen},
o harmonico pode ser escrito na forma\\
\begin{equation}
    y = \sqrt{2}\cos{3x}+\sin{3x}
\end{equation} 
e a notação dada pela equação \ref{eq:harmSimpl} será usada daqui em diante.\\
\\
Também será conviniente explicitar o período $T$ em \ref{eq:harmSimpl}. Se definirmos
$T = 2l$, então, como $T = 2\pi/\omega$, temos
\begin{equation}
    \notag
    \omega = \dfrac{2\pi}{T}=\dfrac{\pi}{l}
\end{equation}
e assim, o harmonico com período $T=2l$ pode ser escrito da seguinte forma\\
\begin{equation}
    a\cos{\dfrac{\pi x}{l}} + b\sin{\dfrac{\pi x}{l}}
\end{equation}