\section{Harmonicos}
A função periódica mais simples é $y = \sin{x}$ e pode ser escrito da forma:
\begin{equation}
\label{harm}
    \senoide
\end{equation} 
\\
onde $A$, $\omega$ e $\varphi$ são constantes. Essa função é chamada de função 
de função \textit{harmonica} de amplitude $A$, frequência $\omega$ e fase
inicial $\varphi$. Neste caso, o período dessa harmonica é $T = 2\pi / \omega$
\begin{equation}
\label{harm_ex}
    A\sin{\left[\omega\left(x+\dfrac{2\pi}{\omega}\right) + \varphi\right]} = A\sin{[(\omega x + \varphi) + 2\pi]} = A\sin{(\omega x + \varphi)}
\end{equation}
\\
{{daí tem uma historinha de onde surgiu esses termos, spoiler: de um problema 
oscilatório.}}\\
\\
Vamos assumir que $\omega > 0$, uma vez que, pela propriedade do seno, $\sin{-a} = - \sin{a}$

Vamos examinar o comportamento da função 
\begin{equation}
\label{harm1}
    \senoide
\end{equation}
\\
Para $A=1$, $\omega = 1$ e $\varphi = 0$, temos a curva senóide comum $y = \sin{x}$\\
\\
\begin{tikzpicture}
\begin{axis}[
    Axis Style,
    xtick={
        -6.28318, -4.7123889, -3.14159, -1.5708,
        1.5708, 3.14159, 4.7123889, 6.28318, 7.85398,
        9.42478, 10.99558, 12.56638
    },
    xticklabels={
        $-2\pi$, $-\frac{3\pi}{2}$, $-\pi$, $-\frac{\pi}{2}$,
        $\frac{\pi}{2}$, $\pi$, $\frac{3\pi}{2}$, $2\pi$,
        $\frac{5\pi}{2}$, $3\pi$, $\frac{7\pi}{2}$, $4\pi$
    }]
\addplot [mark=none, thick] {sin(deg(x))};
\label{senoide}
\end{axis}
\end{tikzpicture}
\\
Agora considere o seguinte harmonico $y = \sin{wx}$ e definir $\omega x = z$, teremos
$y = \sin{z}$ cujo gráfico é a curva senóide normal. Portanto, o gráfico de 
$y = \sin{\omega x}$ é obtido deformando o gráfico da senóide comum. Por exemplo,
se atribuirmos um $\omega > 1$, teremos uma \textit{compressão} do gráfico da
senóide, então se tivermos $A = 1$, $\omega = 3$ e $\varphi = 0$, o gráfico 
desse harmonico seria como \ref{freq_ex1} abaixo.
\begin{tikzpicture}
\begin{axis}[
    Axis Style,
    xtick={
        -6.28318, -4.7123889, -3.14159, -1.5708,
        1.5708, 3.14159, 4.7123889, 6.28318, 7.85398,
        9.42478, 10.99558, 12.56638
    },
    xticklabels={
        $-2\pi$, $-\frac{3\pi}{2}$, $-\pi$, $-\frac{\pi}{2}$,
        $\frac{\pi}{2}$, $\pi$, $\frac{3\pi}{2}$, $2\pi$,
        $\frac{5\pi}{2}$, $3\pi$, $\frac{7\pi}{2}$, $4\pi$
    }]
\addplot [mark=none, thick] {sin(deg(2*x))};
\label{freq_ex1}
\end{axis}
\end{tikzpicture}

Ou seja, tivemos uma "compressão" da curva senóide original ao setar $\omega = 2$,
sendo assim sempre que tivermos um $\omega > 1$, teremos uma propocionalmente
\textbf{menor}, neste caso $T = 2\pi / \omega = 2\pi / 3$.

Por outro lado, se atribuirmos um $\omega < 1$, teríamos uma \textit{expansão}
do gráfico da senóide, então se para $A = 1$, $\omega = 1/4$ e $\varphi = 0$,
o gráfico seria:\\
\begin{tikzpicture}
\begin{axis}[
    Axis Style,
    xtick={
        -6.28318, -4.7123889, -3.14159, -1.5708,
        1.5708, 3.14159, 4.7123889, 6.28318, 7.85398,
        9.42478, 10.99558, 12.56638
    },
    xticklabels={
        $-2\pi$, $-\frac{3\pi}{2}$, $-\pi$, $-\frac{\pi}{2}$,
        $\frac{\pi}{2}$, $\pi$, $\frac{3\pi}{2}$, $2\pi$,
        $\frac{5\pi}{2}$, $3\pi$, $\frac{7\pi}{2}$, $4\pi$
    }]
\addplot [mark=none, thick] {sin(deg((x/2))};
\label{freq_ex2}
\end{axis}
\end{tikzpicture}
\\
Assim, teremos uma "expansão" do gráfico da curva senóide original ao setar $\omega < 1$,
e uma função com período \textbf{maior} com $\omega < 1$, neste caso 
\mbox{$T = 2\pi/\omega = \dfrac{2\pi}{1/2} = 4\pi$}.

Agora considere o harmonico $y = \sin{(\omega x + \varphi)}$ e definir $\omega x + \varphi = \omega z$,
para que $x = z - \varphi/\omega$. Como já sabemos o gráfico de $\sin{\omega z}$, o gráfico 
de $y = \sin{(\omega x + \varphi)}$ é obtido deslocando o gráfico de $y = \sin{\omega x}$ ao
longo do eixo x por $-\varphi/\omega$. Então, dado $A = 1$, $\omega = 1$ e $\varphi = 1/2$,
teremos a curva que representa o $\cos{x}$\\
\begin{tikzpicture}
\begin{axis}[
    Axis Style,
    xtick={
        -6.28318, -4.7123889, -3.14159, -1.5708,
        1.5708, 3.14159, 4.7123889, 6.28318, 7.85398,
        9.42478, 10.99558, 12.56638
    },
    xticklabels={
        $-2\pi$, $-\frac{3\pi}{2}$, $-\pi$, $-\frac{\pi}{2}$,
        $\frac{\pi}{2}$, $\pi$, $\frac{3\pi}{2}$, $2\pi$,
        $\frac{5\pi}{2}$, $3\pi$, $\frac{7\pi}{2}$, $4\pi$
    }]
\addplot [mark=none, thick] {sin(deg(x+(pi/2)))};
\label{fase_ex}
\end{axis}
\end{tikzpicture}
\\
que nada mais é que a curva senóide deslocada para esquerda.\\
\\
Finalmente, o harmonico \senoide é obtido do harmonico $y = \sin{(\omega x + \varphi)}$
multiplicando todas as ordenadas por $A$, então dado $A = 2$, $\omega = 1$ e $\varphi = 0$,
temos\\
\begin{tikzpicture}
\begin{axis}[
    Axis Style,
    ymin=-2.5,
    ymax=2.5,
    ytick={-2,-1,0,1,2},
    xtick={
        -6.28318, -4.7123889, -3.14159, -1.5708,
        1.5708, 3.14159, 4.7123889, 6.28318, 7.85398,
        9.42478, 10.99558, 12.56638
    },
    xticklabels={
        $-2\pi$, $-\frac{3\pi}{2}$, $-\pi$, $-\frac{\pi}{2}$,
        $\frac{\pi}{2}$, $\pi$, $\frac{3\pi}{2}$, $2\pi$,
        $\frac{5\pi}{2}$, $3\pi$, $\frac{7\pi}{2}$, $4\pi$
    }]
\addplot [mark=none, thick] {2*sin(deg(x+(pi/2)))};
\label{amp_ex}
\end{axis}
\end{tikzpicture}
\\
Portanto, podemos resumir tudo isso no seguinte:\\
\textit{ O gráfico da harmonica \senoide é obtido do gráfico da curva
senóide comum por uma compressão (ou expansão) uniforme ao longo dos eixos,
mais um deslocamento ao longo do eixo x.}\\
\\
Com isso, podemos utilizar uma conhecida fórmula matemática para derivar
o seguinte:\\
\begin{equation}
    A\sin{(\omega x + \varphi)} = A(\cos{\omega x}\sin{\varphi} + \sin{\omega x}\cos{\varphi})
\end{equation}
Então, temos\\
\begin{equation}
\label{ab_harm}
    a = A\sin{\varphi}\text{\hspace{10pt},\hspace{10pt}}b = A\cos{\varphi}
\end{equation}
podemos dizer que todo harmonico pode ser representado na forma
\begin{equation}
\label{eq_harm}
    y = a\cos{\omega x} + b\sin{\omega x}
\end{equation}
\\
Do mesmo jeito que uma função com a forma \ref{eq_harm} é um harmonico também. 
Para provar isso, basta resolver \ref{ab_harm} para $a$ e $b$. O resultado é
\begin{equation}
    \begin{split}
        A = \sqrt{a^2 + b^2}\hspace{5pt},\hspace{10pt} &\sin{\varphi} = \dfrac{a}{A} = \dfrac{a}{\sqrt{a^2 + b^2}}\\
        e\hspace{10pt} & \cos{\varphi} = \dfrac{b}{A} = \dfrac{b}{\sqrt{a^2 + b^2}}
    \end{split}
\end{equation} 
do qual $\varphi$ pode ser facilmente encontrado.\\
\\
Assim, podemos escrever os harmonicos na forma \ref{eq_harm}. No exemplo \ref{amp_ex},
o harmonico pode ser escrito na forma\\
\begin{equation}
    y = \sqrt{2}\cos{3x}+\sin{3x}
\end{equation} 
e essa notação será usada daqui em diante.\\
\\
Também será conviniente explicitar o período $T$ em \ref{eq_harm}. Se definirmos
$T = 2l$, então, como $T = 2\pi/\omega$, temos
\begin{equation}
    \omega = \dfrac{2\pi}{T}=\dfrac{\pi}{l}
\end{equation}

e assim, o harmonico com período $T=2l$ pode ser escrito da seguinte forma\\
\begin{equation}
    a\cos{\dfrac{\pi x}{l}} + b\sin{\dfrac{\pi x}{l}}
\end{equation}