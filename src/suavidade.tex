\chapter{Funções suaves e semi-suaves}

A função $f(x)$ é dita suave se no intervalo $[a, b]$, $f(x)$
possui derivada contínua em $[a,b]$. Em linguagem geométrica,
isso significa que a direção da tangente muda continuamente,
ou seja sem saltos enquanto se move pela curva de $f(x)$.

Portanto, o gráfico de uma função suave é uma curva ``suave",
i.e. não possui ``cantos" ou ``pontas".

Uma função $f(x)$ é dita ``semi-suave" no intervalo $[a,b]$
caso $f(x)$ e suas derivadas são ambas contínuas em $[a,b]$,
ou se possui um número finito de descontinuidades em $[a,b]$.
A Figura \ref{fig:suave} seria um exemplo de uma função suave.

\begin{figure}[H]
\begin{center}
    
    \begin{tikzpicture}
    \begin{axis}[
        Axis Style,
        ymin=-0.3,
        ymax=3,
        ticks=none,
        width=7cm]
    \addplot [mark=none, thick, domain=pi:3*pi] {sin(deg(x/2)) + 1.5 +sin(deg(x))};
    \end{axis}
    \end{tikzpicture}
\end{center}
    \caption{Uma função dita suave}
    \label{fig:suave}
\end{figure}



\begin{figure}
\centering
\begin{subfigure}{.5\textwidth}
  \centering
    \begin{tikzpicture}
    \begin{axis}[
        Axis Style,
        xmin=-pi,
        ymin=-0.3,
        ymax=4,
        width=7cm,
        ticks=none]
        \addplot[name path=A, mark=none, thick, domain=pi:2*pi] {sin(deg(x)) + 2};      
        \addplot [name path=A1,
            color=gray, thick, dashed]
            coordinates {(pi,0) (pi,2) } ;
        \addplot[name path=B, mark=none, thick, domain=2*pi:3*pi] {sin(deg(x))+2 + sin(deg(2*x))/2};
        \addplot [name path=B1,
            color=gray, thick, dashed]
            coordinates {(2*pi,0) (2*pi,2) } ;
        \addplot[name path=C, mark=none, thick, domain=3*pi:4*pi] {2};
        \addplot [name path=C1,
            color=gray, thick, dashed]
            coordinates {(3*pi,0) (3*pi,2) } ;      
    \end{axis}
    \end{tikzpicture} 
  \caption{Semi-suave contínua}
  \label{fig:semiCont}
\end{subfigure}%
\begin{subfigure}{.5\textwidth}
  \centering
    \begin{tikzpicture}
    \begin{axis}[
        Axis Style,
        xmin=-pi,
        ymin=-0.3,
        ymax=4,
        width=7cm,
        ticks=none]
        \addplot[name path=A, mark=none, thick, domain=pi:2*pi] {sin(deg(x)) + 3};      
        \addplot [name path=A1,
            color=gray, thick, dashed]
            coordinates {(pi,0) (pi,3) } ;
        \addplot[name path=B, mark=none, thick, domain=2*pi:3*pi] {sin(deg(x))+1 + sin(deg(2*x))/2};
        \addplot [name path=B1,
            color=gray, thick, dashed]
            coordinates {(2*pi,0) (2*pi,3) } ;
        \addplot[name path=C, mark=none, thick, domain=3*pi:4*pi] {3};
        \addplot [name path=C1,
            color=gray, thick, dashed]
            coordinates {(3*pi,0) (3*pi,3) } ;      
    \end{axis}
    \end{tikzpicture} 
  \caption{Semi-suave descontínua}
  \label{fig:semiDesc}
\end{subfigure}
\caption{Os dois tipos de funções semi-suaves}
\label{fig:test}
\end{figure}


\begin{definicao}
    Uma função $f(x)$ é \textbf{semi-suave contínua} se possuir
    um número finito de ``pontas" em um intervalo $[a,b]$.
\end{definicao}

\begin{definicao}
    Uma função $f(x)$ é \textbf{semi-suave descontínua} se possuir
    um número finito de descontinuidades em um intervalo $[a,b]$.
\end{definicao}



A função da Figura \ref{fig:semiCont} representa uma função semi-suave
contínua, e a Figura \ref{fig:semiDesc} representa uma função dita 
semi-suave descontínua.