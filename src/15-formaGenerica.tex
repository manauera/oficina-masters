\chapter{Funções de período 2l}

Caso seja necessário expandir uma função $f(x)$ de período $2l$ em uma série de Fourier,
setamos $x = lt/\pi$, e portanto obtendo a função $\phi(t) = f(lt/\pi)$ de período $2\pi$
(Cap. 4). Para $\phi(t)$ podemos formar a série de Fourier

\begin{equation}
    \label{eq:151}
     \phi(t) ~ \dfrac{a_0}{2} + \sum\limits_{n=1}^{\infty}(a_n cos(nt) + b_n sen(nt))
\end{equation}
onde
\begin{equation}
\notag
    \begin{split}
        a_n &= \dfrac{1}{l}\int_{-l}^{l}\phi(t)cos(nt)dx = \dfrac{1}{l}\int_{-l}^{l}f\left(\dfrac{lt}{\pi}\right)cos(nt)dx, (n = 0, 1, 2, \ldots)\\
        b_n &= \dfrac{1}{l}\int_{-l}^{l}\phi(t)sen(nt)dx = \dfrac{1}{l}\int_{-l}^{l}f\left(\dfrac{lt}{\pi}\right)sen(nt)dx, (n = 1, 2, \ldots)
    \end{split}
\end{equation}

Retornando ao valor original da variável $x$ setando $t = \pi x/l$, temos

\begin{equation}
\label{eq:152}
    \phi(t) ~ \dfrac{a_0}{2} + \sum\limits_{n=1}^{\infty}\left(a_n cos\left(\dfrac{\pi nx}{l}\right) + b_n sen\left(\dfrac{\pi nx}{l}\right)\right)
\end{equation}
onde 
\begin{equation}
\label{eq:153}
    \begin{split}
        a_n &= \dfrac{1}{l}\int_{-l}^{l}f(x)cos\left(\dfrac{\pi nx}{l}\right)dx, (n = 0, 1, 2, \ldots)\\
        b_n &= \dfrac{1}{l}\int_{-l}^{l}f(x)sen\left(\dfrac{\pi nx}{l}\right)dx, (n = 1, 2, \ldots)
    \end{split}
\end{equation}

Os coeficientes \ref{eq:153} ainda são chamados de coeficientes de Fourier de $f(x)$, e a série 
\ref{eq:152} continua sendo a série de Fourier de $f(x)$. Se a igualdade permanece em \ref{eq:151},
então também serve para \ref{eq:152}, o inverso é verdadeiro também.

Seria possível construir toda a teoria que vimos até aqui nesse formato diretamente, começando
de um sistema trigonométrico na forma 

\begin{equation}
\label{eq:154}
    1, cos\left(\dfrac{\pi x}{l}\right), sen\left(\dfrac{\pi x}{l}\right), \ldots, cos\left(\dfrac{\pi nx}{l}\right), sen\left(\dfrac{\pi nx}{l}\right), \ldots
\end{equation} 

como fizemos em \ref{eq:51}. O sistema \ref{eq:154} consiste de funções com um período comum
de $2l$, e é facilmente verificado que essas funções são ortogonais em todo intervalo $2l$.
As considerações dos Cap. 6, 7, 10, 12 e 14 podem ser adaptadas para o sistema \ref{eq:154},
e o resultado é uma formulação análoga, trocando apenas $\pi$ por $l$.

Particularmente, ao invés de uma função $f(x)$ de período $2l$, podemos considerar a 
função definida somente no intervalo $[-l, l]$ (ou qualquer outro intervalo de tamanho $2l$).

A série de Fourier de tal função é identica à sua extensão periódica em todo o eixo-x. O
critério de convergência do Cap. 10 continua "valendo", se substituirmos o período $2\pi$ por
$2l$.

Se $f(x)$ é par, as fórmulas \ref{eq:153} ficam
\begin{equation}
\label{eq:155}
    \begin{split}
        a_n &= \dfrac{2}{l}\int_{0}^{l}f(x)cos\left(\dfrac{\pi nx}{l}\right)dx (n = 0, 1, 2, \ldots)\\
        b_n &= 0
    \end{split}
\end{equation}
enquanto se $f(x)$ é ímpar, ficam
\begin{equation}
\label{eq:156}
    \begin{split}
        a_n &= 0\\
        b_n &= \dfrac{2}{l}\int_{0}^{l}f(x)sen\left(\dfrac{\pi nx}{l}\right)dx (n = 1, 2, \ldots)
    \end{split}
\end{equation}

Como no Cap. 12, podemos usar este fato para expandir a função $f(x)$ definida somente no
intervalo $[0, l]$ em série de cossenos ou série de senos (fazendo a extensão ímpar ou par
da função no intervalo $[-l, 0]$).\\

A forma complexa da série \ref{eq:152} fica assim
\begin{equation}
    \notag
    f(x) ~ \sum\limits_{n=-\infty}^{\infty}c_n e^{i\pi nx/l}
\end{equation}
onde 
\begin{equation}
\notag
    c_n = \dfrac{1}{2l}\int_{-l}^{l}f(x)e^{-i\pi nx/l}dx\text{, para }n = 0, \pm 1, \pm 2, \ldots
\end{equation} 
ou
\begin{equation}
\notag
    c_0 = \dfrac{a_0}{2}, c_n = \dfrac{a_n - ib_n}{2}, c_{-n}=\dfrac{a_n + ib_n}{2}\text{, para }(n = 1, 2, \ldots)
\end{equation} 