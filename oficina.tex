\documentclass{report}

\renewcommand{\thesection}{}
\renewcommand{\thesubsection}{\arabic{section}.\arabic{subsection}}
\makeatletter
\def\@seccntformat#1{\csname #1ignore\expandafter\endcsname\csname the#1\endcsname\quad}
\let\sectionignore\@gobbletwo
\let\latex@numberline\numberline
\def\numberline#1{\if\relax#1\relax\else\latex@numberline{#1}\fi}
\makeatother

\newcommand{\senoide}{\mbox{$y = \mathcal{A}\sin{(\omega x + \varphi)}$}}

\usepackage[utf8]{inputenc}

\title{Série de Fourier}
\author{Gustavo Higuchi}
\date{\today}

\usepackage{natbib}
\usepackage{graphicx}
\usepackage{amssymb}
\usepackage{amsthm}
\usepackage{amsmath}
\usepackage{color}   %May be necessary if you want to color links
\usepackage[portuguese, ruled, linesnumbered]{algorithm2e}
\usepackage{float}
\usepackage{pgfplots}
\usepackage{caption}
\usepackage{subcaption}
\usepackage[portuguese]{babel}
\usepackage{mwe}
\usepackage{subfig}

\usepgfplotslibrary{fillbetween}
\pgfkeys{/pgfplots/Axis Style/.style={
    width=13.5cm, height=6cm,
    axis x line=center, 
    axis y line=middle, 
    samples=200,
    ymin=-1.5, ymax=1.5,
    xmin=0, xmax=13.0,
    domain=0:4*pi
}}

\newtheorem{definicao}{Definição}
\newtheorem{teorema}{Teorema}

\theoremstyle{definition} 
\newtheorem{propriedade}{Propriedade}
\newtheorem*{exemplo}{Exemplo}


\usepackage{mathtools}
\DeclarePairedDelimiter\ceil{\lceil}{\rceil}
\DeclarePairedDelimiter\floor{\lfloor}{\rfloor}

% usado para linkar cada section na tabela de conteúdo com a respectiva
% página no documento
\usepackage{hyperref}
\hypersetup{
    colorlinks,
    citecolor=black,
    filecolor=black,
    linkcolor=black,
    urlcolor=black,
    linktoc=all
}
% o começo do documento
\begin{document}

% compila o título
\maketitle

% compila a tabela de conteúdos
\tableofcontents
\listoffigures
\newpage

%
\chapter{Funções periódicas}

\begin{definicao}
\label{def1}
    
Uma função $f(x)$ é dita periódica se existe uma constante $T > 0$, tal que 
\begin{equation}
    f(x + T) = f(x)
\end{equation}
para qualquer $T \in \mathbb{R}$. 
\end{definicao}
Essa constante T é chamada de período da função $f(x)$. As funções periódicas 
mais comuns são $\sin{x}$, $\cos{x}$, $\tan{x}$, etc. Funções periódicas surgem
em muitas aplicações matemáticas e em problemas de física e engenharia. 
\\
Se plotarmos o gráfico da função $y=f(x)$ em qualquer intervalo fechado 
\mbox{$a \leq x \leq a + T$}, é possível obter o gráfico de $f(x)$ através da 
repitição periódica da porção do gráfico correspondente a \mbox{$a \leq x \leq a + T$}.
Na figura ~\ref{fig:periodExp}, temos uma função periódica de período $T=2\pi$.
\\
\begin{figure}[H]
    \begin{tikzpicture}
    \begin{axis}[
        Axis Style,
        xtick={
            -6.28318, -4.7123889, -3.14159, -1.5708, 1,
            1.5708, 3.14159, 4.7123889, 6.28318, 7.28318, 
            9.42478, 10.99558, 12.56638
        },
        xticklabels={
            $-2\pi$, $-\frac{3\pi}{2}$, $-\pi$, $-\frac{\pi}{2}$, $a$,
            $\frac{\pi}{2}$, $\pi$, $\frac{3\pi}{2}$, $2\pi$, $a+2\pi$,
            $3\pi$, $\frac{7\pi}{2}$, $4\pi$
        }]
    \addplot [mark=none, thick] {sin(deg(x)) + 1/2*sin(deg(2*x)) + 1/4*sin(deg(3*x))};

    \addplot [name path=border,
            color=gray, thick, dashed]
            coordinates {(1,0) (1,5) } ;
    \addplot [name path=border,
            color=gray, thick, dashed]
            coordinates {(1+2*3.14159,0) (1+2*3.14159,5) };
    \end{axis}  
    \end{tikzpicture} 
    \caption{Observe que $f(a) = f(a + 2\pi)$}
    \label{fig:periodExp}
\end{figure}

Se $T$ é um período da função periódica $f(x)$, então seus múltiplos $2T$, $3T$, $4T$, etc 
também são períodos da função $f(x)$. Isso é verificado facilmente ao inspecionar 
os gráficos de uma função periódica, ou pela série de igualdades:\\
\begin{equation}
\label{prop2}
    f(x) = f(x + T) = f(x + 2T) = f(x + 4T) = ...
\end{equation} 
\\
Assim, temos\\
\begin{definicao}
    Se uma função f(x) possui um período $T$, então $kT$ também é um período de
    f(x), ou seja \textbf{se um período existe, ele não é único}
\end{definicao}

Vamos mostrar que o resultado da soma de duas funções periódicas de período T
é também uma função de período T. Então, dadas as funções $f(x) = sen(x)$ e $g(x) = sen(2x)$,
seus gráficos são, respectivamente:\\
\begin{figure}[H]
    \begin{tikzpicture}
    \begin{axis}[
        Axis Style,
        xtick={
            -6.28318, -4.7123889, -3.14159, -1.5708,
            1.5708, 3.14159, 4.7123889, 6.28318, 7.85398,
            9.42478, 10.99558, 12.56638
        },
        xticklabels={
            $-2\pi$, $-\frac{3\pi}{2}$, $-\pi$, $-\frac{\pi}{2}$,
            $\frac{\pi}{2}$, $\pi$, $\frac{3\pi}{2}$, $2\pi$,
            $\frac{5\pi}{2}$, $3\pi$, $\frac{7\pi}{2}$, $4\pi$
        }]
    \addplot [mark=none, thick] {sin(deg(x))};
    \end{axis}
    \end{tikzpicture}
    \caption{$f(x)=sen(x)$}
    \label{fig:senx}
\end{figure}

\begin{figure}[H]
    \begin{tikzpicture}
    \begin{axis}[
        Axis Style,
        xtick={
            -6.28318, -4.7123889, -3.14159, -1.5708,
            1.5708, 3.14159, 4.7123889, 6.28318, 7.85398,
            9.42478, 10.99558, 12.56638
        },
        xticklabels={
            $-2\pi$, $-\frac{3\pi}{2}$, $-\pi$, $-\frac{\pi}{2}$,
            $\frac{\pi}{2}$, $\pi$, $\frac{3\pi}{2}$, $2\pi$,
            $\frac{5\pi}{2}$, $3\pi$, $\frac{7\pi}{2}$, $4\pi$
        }]
    \addplot [mark=none, thick] {sin(deg(2*x))};
    \label{sen2x}
    \end{axis}
    \end{tikzpicture}
    \caption{$f(x)=sen(2x)$}
    \label{fig:sen2x}
\end{figure}

Assim, temos duas funções periódicas de período $T = 2\pi$, vale notar que o período mínimo 
de $f(x)$, $T_f = 2\pi$, é maior que o período mínimo de $g(x)$, $T_g = \pi$, mas
que ambas as funções tem o período em comum de $T = 2\pi$. Para somar essas duas
funções, basta somar o valor de $f(x)$ para cada valor de x ao valor de $g(x)$ para 
cada valor de x. Então, teremos o seguinte:
\begin{figure}[H]
    \begin{tikzpicture}
    \begin{axis}[
        Axis Style,
        ymin=-2.5,
        ymax=2.5,
        ytick={-2,-1,0,1,2},
        yticklabels={-2,-1,0,1,2},
        xtick={
            -6.28318, -4.7123889, -3.14159, -1.5708,
            1.5708, 3.14159, 4.7123889, 6.28318, 7.85398,
            9.42478, 10.99558, 12.56638
        },
        xticklabels={
            $-2\pi$, $-\frac{3\pi}{2}$, $-\pi$, $-\frac{\pi}{2}$,
            $\frac{\pi}{2}$, $\pi$, $\frac{3\pi}{2}$, $2\pi$,
            $\frac{5\pi}{2}$, $3\pi$, $\frac{7\pi}{2}$, $4\pi$
        }]
    \addplot [mark=none, thick] {sin(deg(x)) + sin(deg(2*x))};
    \label{sen2x}
    \end{axis}
    \end{tikzpicture}
    \caption{Somando duas funções de mesmo período}
    \label{fig:addExp}
\end{figure}

É possível ver que, a função $f(x)$ possui um período mínimo maior que a função $g(x)$,
assim, \textbf{a função resultante é uma função de período mínimo $T = 2\pi$}, a 
subtração funciona de forma semelhante.\\

Agora podemos afirmar que a soma e a diferença de duas funções 
periódica de período T é também uma função periódica com período T, onde
T é o maior período entre as funções.\\

E quanto à multiplicação e à divisão? Podemos afirmar o mesmo?\\

A resposta é sim, funciona de forma semelhante da soma e da subtração, multiplicando
os valores $f(x)$ por $g(x)$, para todo x. Ficamos com seguinte:
\begin{figure}[H]
    \begin{tikzpicture}
    \begin{axis}[
        Axis Style,
        ymin=-2.5,
        ymax=2.5,
        ytick={-2,-1,0,1,2},
        yticklabels={-2,-1,0,1,2},
        xtick={
            -6.28318, -4.7123889, -3.14159, -1.5708,
            1.5708, 3.14159, 4.7123889, 6.28318, 7.85398,
            9.42478, 10.99558, 12.56638
        },
        xticklabels={
            $-2\pi$, $-\frac{3\pi}{2}$, $-\pi$, $-\frac{\pi}{2}$,
            $\frac{\pi}{2}$, $\pi$, $\frac{3\pi}{2}$, $2\pi$,
            $\frac{5\pi}{2}$, $3\pi$, $\frac{7\pi}{2}$, $4\pi$
        }]
    \addplot [mark=none, thick] {sin(deg(x)) * sin(deg(2*x))};
    \end{axis}
    \end{tikzpicture}
    \caption{Multiplicando funções de mesmo período}
    \label{fig:multExp}
\end{figure}

Por mais esquisito que a função fique, podemos observar que a função resultante
permaneceu com o período $T = 2\pi$. Dessa forma, está claro que operações 
de funções que partilham um mesmo período, terá uma função resultante com o mesmo
período.\\

\begin{definicao}
    Seja $f(x)$ e $g(x)$ duas funções periódicas com período em comum $T$, a soma, subtração,
    multiplicação e divisão das duas funções resulta em uma função periódica de 
    mesmo período mínimo é o maior período entre $f(x)$ e $g(x)$.
\end{definicao}

Mais adiante, temos a seguinte propriedade de qualquer função periódica $f(x)$
com período $T$:\\
\begin{definicao}
Se f(x) é integrável em um intervalo de tamanho T,
então é integrável em qualquer outro intervalo de tamanho T, e o valor da integral
é o mesmo\\
\begin{equation}
\label{int_prop1}
    \int_a^{a+T} \! f(x) \, \mathrm{d}x = \int_b^{b+T} \! f(x) \, \mathrm{d}x.
\end{equation}
para qualquer a, b. \\
\label{def:functPer}
\end{definicao}

Essa propriedade é uma consequência imediata da interpretação da integral como
área. Cada integral é igual a área incluida entre a curva $y=f(x)$, o eixo x e
as ordenadas desenhadas nos limites do intervalo, onde áreas acima do eixo x
são tidas como positivas, e áreas abaixo são tidas como negativas. No caso,
as áreas representadas pelas duas integrais são a mesma por causa da propriedade 
de \ref{def:functPer}. A figura \ref{fig:int_area}, a área em \textcolor{blue}{azul} e a
área em \textcolor{red}{vermelho} representam as áreas das integrais da função
periódica $f(x)=sen(4x)+sen(2x)$ de período $T=\pi$ para intervalos de tamanho $\pi$. 
\\
\begin{figure}[H]
    \begin{tikzpicture}
    \begin{axis}[
        Axis Style,
        ymin=-2.5,
        ymax=2.5,
        ytick={-2,-1,0,1,2},
        yticklabels={-2,-1,0,1,2},
        xtick={
            -6.28318, -4.7123889, -3.14159, -1.5708,
            1.5708, 3.14159, 4.7123889, 6.28318, 7.85398,
            9.42478, 10.99558, 12.56638
        },
        xticklabels={
            $-2\pi$, $-\frac{3\pi}{2}$, $-\pi$, $-\frac{\pi}{2}$,
            $\frac{\pi}{2}$, $\pi$, $\frac{3\pi}{2}$, $2\pi$,
            $\frac{5\pi}{2}$, $3\pi$, $\frac{7\pi}{2}$, $4\pi$
        }]
        \addplot[name path=A, mark=none, thick] {sin(deg(4*x)) + sin(deg(2*x))};
        \addplot[name path=B]{0};
        \addplot[blue!40] fill between[of=A and B,
            soft clip={domain=0:3.14159},];
        \addplot[red!40] fill between[of=A and B, 
            soft clip={domain=4.712385:7.853975},];
    \end{axis}
    \end{tikzpicture}
\caption{Observe que ambas as áres são iguais}
\label{fig:int_area}
\end{figure}

Daqui em diante, quando uma função $f(x)$ de período $T$ for integrável, então
ela será integrável em qualquer intervalo de tamanho $T$.
%\chapter{Harmonicos}

A função periódica mais simples é $y = \sin{x}$ e se imaginar constanstes $\mathcal{A} = 1$,
$\omega = 1$ e $\varphi = 0$, podemos reescrever a mesma função da seguinte forma:
\begin{equation}
\label{eq:harm}
    \senoide
\end{equation} 
\\
onde $\mathcal{A}$, $\omega$ e $\varphi$ são constantes. Essa função é chamada de função 
\textit{harmonica} de amplitude $\mathcal{A}$, frequência $\omega$ e fase
inicial $\varphi$. Neste caso, o período dessa harmonica é $T = 2\pi / \omega$
\begin{equation}
\label{harm_ex}
    \mathcal{A}\sin{\left[\omega\left(x+\dfrac{2\pi}{\omega}\right) + \varphi\right]} = \mathcal{A}\sin{[(\omega x + \varphi) + 2\pi]} = \mathcal{A}\sin{(\omega x + \varphi)}
\end{equation}


Agora, vamos definir daqui em diante $\omega > 0$, uma vez que, pela propriedade do seno, 
\mbox{$sen(-a) = - sen (a)$} e examinar o comportamento da função \ref{eq:harm} para 
diferentes valores de amplitude, frequência e fase inicial.
\\
\\


Para $\mathcal{A}=1$, $\omega = 1$ e $\varphi = 0$, temos a curva senóide comum $y = sen x$\\
\\

\begin{figure}[H]
    \begin{tikzpicture}
    \begin{axis}[
        Axis Style,
        xtick={
            -6.28318, -4.7123889, -3.14159, -1.5708,
            1.5708, 3.14159, 4.7123889, 6.28318, 7.85398,
            9.42478, 10.99558, 12.56638
        },
        xticklabels={
            $-2\pi$, $-\frac{3\pi}{2}$, $-\pi$, $-\frac{\pi}{2}$,
            $\frac{\pi}{2}$, $\pi$, $\frac{3\pi}{2}$, $2\pi$,
            $\frac{5\pi}{2}$, $3\pi$, $\frac{7\pi}{2}$, $4\pi$
        }]
    \addplot [mark=none, thick] {sin(deg(x))};
    \label{senoide}
    \end{axis}
    \end{tikzpicture}
    \caption{Uma senóide comum}
    \label{fig:senoide}
\end{figure}

Agora considere o seguinte harmonico $y = sen(wx)$ e definir $\omega x = z$, teremos
$y = sen(z)$ cujo gráfico é a curva senóide normal. Portanto, o gráfico de 
$y = sen(\omega x)$ é obtido deformando o gráfico da senóide comum. Por exemplo,
se atribuirmos um $\omega > 1$, teremos uma \textit{compressão} do gráfico da
senóide, então se tivermos $\mathcal{A} = 1$, $\omega = 3$ e $\varphi = 0$, o gráfico 
desse harmonico seria como \ref{fig:compSen} abaixo.

\begin{figure}[H]
    \begin{tikzpicture}
    \begin{axis}[
        Axis Style,
        xtick={
            -6.28318, -4.7123889, -3.14159, -1.5708,
            1.5708, 3.14159, 4.7123889, 6.28318, 7.85398,
            9.42478, 10.99558, 12.56638
        },
        xticklabels={
            $-2\pi$, $-\frac{3\pi}{2}$, $-\pi$, $-\frac{\pi}{2}$,
            $\frac{\pi}{2}$, $\pi$, $\frac{3\pi}{2}$, $2\pi$,
            $\frac{5\pi}{2}$, $3\pi$, $\frac{7\pi}{2}$, $4\pi$
        }]
    \addplot [mark=none, thick] {sin(deg(2*x))};
    \end{axis}
    \end{tikzpicture}
    \caption{Compressão de uma senóide}
    \label{fig:compSen}
\end{figure}

Ou seja, tivemos uma ``compressão" da curva senóide original ao setar $\omega = 2$,
sendo assim sempre que tivermos um $\omega > 1$, teremos uma propocionalmente
\textbf{menor}, neste caso $T = 2\pi / \omega = 2\pi / 3$.

Por outro lado, se atribuirmos um $\omega < 1$, teríamos uma \textit{expansão}
do gráfico da senóide, então se para $\mathcal{A} = 1$, $\omega = 1/4$ e $\varphi = 0$,
o gráfico seria:
\begin{figure}[H]
    \begin{tikzpicture}
    \begin{axis}[
        Axis Style,
        xtick={
            -6.28318, -4.7123889, -3.14159, -1.5708,
            1.5708, 3.14159, 4.7123889, 6.28318, 7.85398,
            9.42478, 10.99558, 12.56638
        },
        xticklabels={
            $-2\pi$, $-\frac{3\pi}{2}$, $-\pi$, $-\frac{\pi}{2}$,
            $\frac{\pi}{2}$, $\pi$, $\frac{3\pi}{2}$, $2\pi$,
            $\frac{5\pi}{2}$, $3\pi$, $\frac{7\pi}{2}$, $4\pi$
        }]
    \addplot [mark=none, thick] {sin(deg((x/2))};
    \end{axis}
    \end{tikzpicture}
    \caption{Expansão de uma senóide}
    \label{fig:expSen}
\end{figure}

Assim, teremos uma "expansão" do gráfico da curva senóide original ao setar $\omega < 1$,
e uma função com período \textbf{maior} com $\omega < 1$, neste caso 
\mbox{$T = 2\pi/\omega = \dfrac{2\pi}{1/2} = 4\pi$}.

Agora considere o harmonico $y = sen(\omega x + \varphi)$ e definir $\omega x + \varphi = \omega z$,
para que $x = z - \varphi/\omega$. Como já sabemos o gráfico de $sen(\omega z)$, o gráfico 
de $y = sen(\omega x + \varphi)$ é obtido deslocando o gráfico de $y = sen(\omega x)$ ao
longo do eixo x por $-\varphi/\omega$. Então, dado $\mathcal{A} = 1$, $\omega = 1$ e $\varphi = 1/2$,
teremos a curva que representa o $cos(x)$

\begin{figure}[H]
    \begin{tikzpicture}
    \begin{axis}[
        Axis Style,
        xtick={
            -6.28318, -4.7123889, -3.14159, -1.5708,
            1.5708, 3.14159, 4.7123889, 6.28318, 7.85398,
            9.42478, 10.99558, 12.56638
        },
        xticklabels={
            $-2\pi$, $-\frac{3\pi}{2}$, $-\pi$, $-\frac{\pi}{2}$,
            $\frac{\pi}{2}$, $\pi$, $\frac{3\pi}{2}$, $2\pi$,
            $\frac{5\pi}{2}$, $3\pi$, $\frac{7\pi}{2}$, $4\pi$
        }]
    \addplot [mark=none, thick] {sin(deg(x+(pi/2)))};
    \end{axis}
    \end{tikzpicture}
    \caption{Deslocamento da senóide}
    \label{fig:deslocSen}
\end{figure}

que nada mais é que a curva senóide deslocada para esquerda.\\
\\
Finalmente, o harmonico \senoide é obtido do harmonico $y = sen(\omega x + \varphi)$
multiplicando todas as ordenadas por $\mathcal{A}$, então dado $\mathcal{A} = 2$, $\omega = 1$ e $\varphi = 0$,
temos

\begin{figure}[H]
    \begin{tikzpicture}
    \begin{axis}[
        Axis Style,
        ymin=-2.5,
        ymax=2.5,
        ytick={-2,-1,0,1,2},
        xtick={
            -6.28318, -4.7123889, -3.14159, -1.5708,
            1.5708, 3.14159, 4.7123889, 6.28318, 7.85398,
            9.42478, 10.99558, 12.56638
        },
        xticklabels={
            $-2\pi$, $-\frac{3\pi}{2}$, $-\pi$, $-\frac{\pi}{2}$,
            $\frac{\pi}{2}$, $\pi$, $\frac{3\pi}{2}$, $2\pi$,
            $\frac{5\pi}{2}$, $3\pi$, $\frac{7\pi}{2}$, $4\pi$
        }]
    \addplot [mark=none, thick] {2*sin(deg(x+(pi/2)))};
    \end{axis}
    \end{tikzpicture}
    \caption{Ampliação da senóide}
    \label{fig:ampSen}
\end{figure}

Portanto, podemos resumir tudo isso no seguinte:\\
\begin{definicao}
    O gráfico de uma harmonica é obtido do gráfico da curva senóide 
    comum por uma compressão (ou expansão) uniforme ao longo dos eixos,
    mais um deslocamento ao longo do eixo x, e é dado pela equação \ref{eq:harm}
\end{definicao}


Assim, podemos utilizar uma conhecida fórmula matemática para derivar
o seguinte:\\
\begin{equation}
    \mathcal{A}sen(\omega x + \varphi) = \mathcal{A}(cos(\omega x)sen(\varphi) + sen(\omega x)cos(\varphi)
\end{equation}
Disso, temos\\
\begin{equation}
    a = \mathcal{A}sen(\varphi)\text{\hspace{10pt},\hspace{10pt}}b = \mathcal{A}cos(\varphi)
\label{eq:ab_harm}
\end{equation}
e então podemos dizer que todo harmonico pode ser representado na forma
\begin{equation}
    y = a \hspace{1pt}cos(\omega x) + b\hspace{1pt}sen(\omega x)
\label{eq:harmSimpl}
\end{equation}
\\
Do mesmo jeito que uma função com a forma \ref{eq:harmSimpl} é um harmonico também. 
Para provar isso, basta resolver \ref{eq:ab_harm} para $a$ e $b$. Temos
\begin{equation}
    \begin{split}
        A = \sqrt{a^2 + b^2}\hspace{5pt},\hspace{10pt} &sen(\varphi) = \dfrac{a}{A} = \dfrac{a}{\sqrt{a^2 + b^2}}\\
        e\hspace{10pt} & cos(\varphi) = \dfrac{b}{A} = \dfrac{b}{\sqrt{a^2 + b^2}}
    \end{split}
\end{equation} 
do qual $\varphi$ pode ser encontrado.\\

Assim, podemos escrever os harmonicos na forma \ref{eq:harmSimpl}. Na Figura \ref{fig:ampSen},
o harmonico pode ser escrito na forma\\
\begin{equation}
    y = \sqrt{2}\cos{3x}+\sin{3x}
\end{equation} 
e a notação dada pela equação \ref{eq:harmSimpl} será usada daqui em diante.\\
\\
Também será conviniente explicitar o período $T$ em \ref{eq:harmSimpl}. Se definirmos
$T = 2l$, então, como $T = 2\pi/\omega$, temos
\begin{equation}
    \notag
    \omega = \dfrac{2\pi}{T}=\dfrac{\pi}{l}
\end{equation}
e assim, o harmonico com período $T=2l$ pode ser escrito da seguinte forma\\
\begin{equation}
    a\cos{\dfrac{\pi x}{l}} + b\sin{\dfrac{\pi x}{l}}
\end{equation}
%\chapter{Polinômios trigonométricos e séries}
Dado o período $T=2l$, considere os harmonicos\\
\begin{equation}
    a_k\cos{\dfrac{\pi kx}{l}} + b_k\sin{\dfrac{\pi kx}{l}},\text{\hspace{5pt}para k = 1,2,3,...}
\end{equation}
\\
Com frequencia $\omega_k = k\pi/l$ e períodos $T_k = \dfrac{2\pi}{\omega_k} = \dfrac{2l}{k}$. 
Uma vez que 
\begin{equation}
\notag
    T = 2l = kT_k
\end{equation}  
\\
\textbf{o número $T=2l$ é simultaneamente o período de todos os harmonicos},
pois um múltiplo de um período é também um período (Sec 1). Então, toda soma na 
forma\\
\begin{equation}
    s_n(n) = \mathcal{A} + \sum\limits_{k=1}^{n}(a_k\cos{\dfrac{k\pi x}{l} + b_k\sin{\dfrac{k\pi x}{l}}})
\end{equation}
\\
é uma função de período $2l$, uma vez que é uma soma de funções de período 
$2l$. Vale notar que $A$ é uma constante e não afeta a periodicidade da função,
inclusive é possível considerar que uma constante é uma função periódica, onde 
qualquer valor pode ser um período.\\
\\
Essa função $s_n(x)$ é chamada  de \textbf{polinômio trigonométrico de ordem n}(
e período $2l$).\\
\\
Por mais que seja a soma de vários harmonicos, um polinômio trigonométrico pode 
ser usado para representar uma função de natureza muito mais complexa que a 
de um harmonico. E geralmente é o caso. Escolhendo as constantes corretamente,
podemos formar funções com gráficos bem diferentes de um simples harmonico.
\\
Na primeira Figura \ref{fig:periodExp}, o polínomio que representa aquele gráfico é\\
\begin{equation}
    y = \sin{x} + \dfrac{1}{2}\sin{2x} + \dfrac{1}{4}\sin{3x}
\end{equation}
\\
Facilmente verificado com algum software de plot de função.\\
\\

A \textbf{série trigonométrica infinita}\\ 
\begin{equation}
    f(x) = A + \sum\limits_{k=1}^{\infty}(a_k\cos{\dfrac{k\pi x}{l}} + b_k\sin{\dfrac{k\pi x}{l}})
\label{eq:serie_inf}
\end{equation}
também representa uma função de período $2l$. As funções como \ref{eq:serie_inf} podem
ser usadas para representar fenômenos de origem muito mais complexa que um polinomio.

Sendo assim, o gráfico de uma função periódica $f(x)$ pode ser obtido através da 
sobreposição de todos os harmonicos que o compõe, i.e., pode ser representado
como uma soma de harmonicos simples.
Então a pergunta que fica é:\\
\textit{Qualquer função que tenha período 2l pode ser representado por uma soma de séries 
trigonométricas?}\\
\\
A resposta é sim, e na realidade, é possível ser usado em grande quantidade de problemas!
Diversos outros fenômenos podem ser ser representado por uma série trigonométrica.
\\
\\  

\textbf{Se}
\begin{equation}
    f(x) = A + \sum\limits_{k=1}^{\infty}(a_k\cos{\dfrac{k\pi x}{l}} + b_k\sin{\dfrac{k\pi x}{l}})
\label{eq:serieLonga}
\end{equation}

\textbf{Então}, podemos definir, por comodidade, que $\dfrac{\pi x}{l} = t$ ou que $x = \dfrac{tl}{\pi}$,
assim teremos\\
\begin{equation}
    g(t) = f(tl/\pi) = A + \sum\limits_{k=1}^{\infty}(a_k\cos{kt} + b_k\sin{kt})
\label{eq:serieSimples}
\end{equation}
\\
onde os harmonicos dessa série tenham período $2\pi$. 

É possível verificar que se a função $f(x)$ de período $2l$ possui a expansão 
\ref{eq:serieLonga}, então a função $g(x)$ de período $2\pi$ possui a expansão 
\ref{eq:serieSimples}, e que o contrário é verdadeiro também. 

Por ser mais legível, daqui em diante usaremos a expansão \ref{eq:serieSimples} 
e ao final faremos a tradução para o mais genérico \ref{eq:serieLonga}.\\
%\section{ Uma terminologia mais precisa}
Agora vamos introduzir a uma terminologia mais precisa e relembrar alguns fatos
de cálculo integral e diferencial. Quando dizemos que $f(x)$ é integrável no 
intervalo [a,b], siginifica que a integral\\
\begin{equation}
\label{int}
    \int_{a}^{b}f(x)dx 
\end{equation}
(que pode ser imprópria) existe no sentido elementar. Portanto, nossas funções 
integráveis $f(x)$ sempre serão contínuas ou com finitas descontinuidades no 
intervalo [a,b], no qual a função pode ser limitada ou não.\\
\\
Em cursos de cálculo integral, primeiro se prova que a função possui um número
finito de descontinuidades dentro de um intervalo, então se a integral\\
\\
\begin{equation}
    \int_{a}^{b}|f(x)|dx 
\end{equation} 
\\
existe, então \ref{int} também existe. Neste caso, a função $f(x)$ é tida como
uma função \textbf{absolutamente integrável}. (Vale notar que o inverso pode não
ser verdadeiro).\\
\\
\textbf{Propriedade 1}\\
Se $f(x)$ é uma função absolutamente integrável e $g(x)$ é uma função integrável
limitada, então o produto $f(x)g(x)$ é uma função absolutamente integrável também.
\\
\\
A seguinte regra de integração por partes é válida:\\
\\
\textit{Seja f(x) e g(x) contínuas em [a,b], mas talvez não diferençiavel
em um número finito de pontos. Portanto se f'(x) e g'(x) são absolutamente 
integráveis, então temos:}\\
\begin{equation}
    \int_{a}^{b}f(x)g'(x) dx = [f(x)g(x)]_{a}^{b} - \int_{a}^{b}f'(x)g(x) dx
\end{equation}

Outro resultado familiar é o fato que se as funções $f_1(x), f_2(x), ..., f_n(x)$
são integráveis no intervalo [a,b], então a soma deles também é integrável.\\
\begin{equation}
\label{43}
    \int_{a}^{b}[\sum\limits_{k=1}^{n}f(x)]dx = \sum\limits_{k=1}^{n}\int_{a}^{b}f(x)dx
\end{equation}

Agora considere uma série infinita de funções:\\
\begin{equation}
\label{44}
    f_1(x), f_2(x), f_3(x), ... = \sum\limits_{k=1}^{\infty}f_k(x)
\end{equation}
\\
Uma série desse tipo é dita convergente se para um dado valor de x,
suas somas parciais:\\
\begin{equation}
    s_n(x) = \sum\limits_{k=1}^{n}f_k(x)\text{, para n = 1, 2, 3, ...}
\end{equation}
\\
tiverem um limite finito:\\
\begin{equation}
    s(x) = \lim_{x\to\infty} s_n(x)
\end{equation}
\\
$s(x)$ é dita ser a soma da série e, obviamente, é uma função de $x$.\\
\\
Se a série converge para todo x no intervalo [a,b], então sua soma $s(x)$
é definida em todo o intervalo [a,b].\\
\\
Agora perguntamos se a formula \ref{43} pode ser extendida para o caso de 
uma série convergente de funções integráveis no intervalo [a,b], i.e.,
a fórmula abaixo é válida?\\
\begin{equation}
\label{45}
    \int_{a}^{b}[\sum\limits_{k=1}^{\infty}f(x)]dx = \int_{a}^{b}s(x) dx = \sum\limits_{k=1}^{\infty}\int_{a}^{b}f_k(x) dx
\end{equation}
\\
Em outras palavras, a série pode ser integrada termo a termo?\\
\\
\ref{45} nem sempre é válida, simplesmente pela série de funções integráveis,
ou até mesmo contínua pode, se quer, ter uma soma integrável. Um problema parecido 
surge com relação a possibilidade de diferenciação termo a termo da série.
E agora vamos descartar uma importante classe de séries de função para o qual 
essas operações podem ser aplicadas.\\
\\
A série \ref{44} é dita ser uniformemente convergente em um intervalo [a,b] se
para um número positivo qualquer $\varepsilon$, existe um número $N$ tal que a
desigualdade abaixo seja verdadeira para todo $n \geq N$ e para todo x no 
intervalo [a,b].\\
\begin{equation}
    |s(x) - s_n(x)| \leq \varepsilon
\end{equation}
\\
Portanto, convergência uniforme significa que para um $n$ suficientemente 
grande e para todo $x$ no intervalo, o gráfico da soma de séries $s(x)$ e o 
gráfico da soma parcial $s_n(x)$, estão $\varepsilon$ distantes uma da outra,
desse jeito, ambas as curvas estarão \textit{uniformemente perto} uma da outra.\\
\\
\textit{Importante Notar:}
Não é qualquer série que converge no intervalo [a,b] e, também, convere uniformemente
no mesmo intervalo.\\
\\
\textbf{gap} 
\\
O teste a seguir é um teste muito útil e simples para a convergencia uniforme
de uma série de funções (Weierstrass M-Test):\\
\textit{
    Se a série infinita de números \\
    \begin{equation}
        M_1 + M_2 + M_3 + ... + M_k + ...
    \end{equation}
    \\
    \\
    convergir e, se para algum x no intervalo [a,b], nós tivermos $|f_k(x)| \leq M_k$
    de um certo k em diante, então a série \ref{43} converge uniformemente ( e 
    absolutamente) no intervalo [a,b].
} 
\\
\textbf{gap}\\
\\
Os seguintes teoremas são válidos:

\newtheorem{teo1}{Teorema}
\begin{teo1}
    Se os termos da série \ref{44} são contínuos em [a,b] e se a
    série converge uniformemente em [a,b], então:\\
    a) A soma da série é contínua\\
    b) A soma pode ser integrada termo a termo
\end{teo1}

\newtheorem{teo2}{Teorema}
\begin{teo2}
    Se a série \ref{44} converge, e se os termos da série são diferenciáveis
    e se a série:\\
    \begin{equation}
        f_1^{'}(x) + f_2^{'}(x) + f_3^{'}(x) + ... = \sum\limits_{k=1}^{\infty}f_k^{'}(x)
    \end{equation}
    é uniformemente convergente em [a,b], então\\
    \begin{equation}
        (\sum\limits_{k=1}^{\infty} f(x))^{'} = s^{'}(x) = \sum\limits_{k=1}^{\infty}f_k^{'}(x)
    \end{equation}
    i.e., a série \ref{44} pode ser diferenciada termo a termo.
\end{teo2}
%\chapter{Revisão de trigonometria}

Um \textit{sistema básico trigonométrico} significa um sistema de funções como 
as de \ref{eq:51} que possuem um período $T=2\pi$\\
\begin{equation}
    1, cos(x), sen(x), cos(2x), sen(2x), ... 
\label{eq:51}
\end{equation}
\\
Com base nessas funções, vamos mostrar algumas funções que irão auxiliar mais
adiante.\\


Para um inteiro $n \neq 0$, temos\\
\begin{definicao}
    \label{def:52}
    A integral definida em um intervalo de tamanho $T=2\pi$ da função $f(x)=sen(x)$ é sempre 
    igual a zero. O mesmo vale para $g(x)=cos(x)$.
    \begin{equation}
        \begin{split}
            \int_{0}^{2\pi}cos(nx)dx & = \left[\dfrac{sen(nx)}{n}\right]_{0}^{2\pi} = 0\\
            \int_{0}^{2\pi}sen(nx)dx & = \left[\dfrac{-cos(nx)}{n}\right]_{0}^{2\pi} = 0
        \end{split}
    \end{equation}
\end{definicao}

Se desenharmos o gráfico das funções $sen(x)$ e $cos(x)$, fica evidente que sua 
integral é zero. Na figura \ref{fig:intSen}, a parte em \textcolor{red}{vermelho} 
é positiva e a parte em \textcolor{blue}{azul} é negativa, e como ambas são iguais,
a área total das funções $sen(x)$ e $cos(x)$ no intervalo $[0:2\pi]$ é zero.
\begin{figure}[H]
    \begin{tikzpicture}
    \begin{axis}[
        Axis Style,
        xtick={
            -6.28318, -4.7123889, -3.14159, -1.5708,
            1.5708, 3.14159, 4.7123889, 6.28318, 7.85398,
            9.42478, 10.99558, 12.56638
        },
        xticklabels={
            $-2\pi$, $-\frac{3\pi}{2}$, $-\pi$, $-\frac{\pi}{2}$,
            $\frac{\pi}{2}$, $\pi$, $\frac{3\pi}{2}$, $2\pi$,
            $\frac{5\pi}{2}$, $3\pi$, $\frac{7\pi}{2}$, $4\pi$
        }]
        \addplot[name path=A, mark=none, thick] {sin(deg(x))};
        \addplot[name path=B]{0};
        \addplot[red!40] fill between[of=A and B,
            soft clip={domain=0:3.14159},];
        \addplot[blue!40] fill between[of=A and B, 
            soft clip={domain=3.14159:3.14159+3.14159},];
    \end{axis}
    \end{tikzpicture}
    \begin{tikzpicture}
    \begin{axis}[
        Axis Style,
        xtick={
            -6.28318, -4.7123889, -3.14159, -1.5708,
            1.5708, 3.14159, 4.7123889, 6.28318, 7.85398,
            9.42478, 10.99558, 12.56638
        },
        xticklabels={
            $-2\pi$, $-\frac{3\pi}{2}$, $-\pi$, $-\frac{\pi}{2}$,
            $\frac{\pi}{2}$, $\pi$, $\frac{3\pi}{2}$, $2\pi$,
            $\frac{5\pi}{2}$, $3\pi$, $\frac{7\pi}{2}$, $4\pi$
        }]
        \addplot[name path=C, mark=none, thick] {sin(deg(x +1.5708))};
        \addplot[name path=D]{0};
        \addplot[red!40] fill between[of=C and D,
            soft clip={domain=0:1.5708},];
        \addplot[blue!40] fill between[of=C and D, 
            soft clip={domain=1.5708:1.5708+3.14159},];
        \addplot[red!40] fill between[of=C and D,
            soft clip={domain=1.5708+3.14159:6.28318},];            
    \end{axis}
    \end{tikzpicture}    
\caption{A integral como área das funções $f(x)=sen(x)$ e $g(x)=cos(x)$}
\label{fig:intSen}
\end{figure}

Também podemos afirmar que\\
\begin{definicao}
    \label{def:53}
    A integral definida em um intervalo de tamanho $T=2\pi$ da função $f(x)=sen^2(x)$
    é sempre a metade do período. O mesmo vale para $g(x)=cos(x)$.
    \begin{equation}
        \begin{split}
            \int_{0}^{2\pi}cos^2(nx)dx & = \int_{0}^{2\pi}\dfrac{1 + cos(2nx)}{2} dx = \pi\\
            \int_{0}^{2\pi}sen^2(nx) dx & = \int_{0}^{2\pi}\dfrac{1 - cos(2nx)}{2} dx = \pi
        \end{split}
    \end{equation}
\end{definicao}

Olhando para o gráfico das funções $f(x)=sen^2(x)$ e $g(x)=cos^2(x)$, representado na figura
\ref{fig:intSenQuad} abaixo, fica claro que a parte que antes era negativa, torna-se positiva
e a parte positiva permanece igual.

\begin{figure}[H]
    \begin{tikzpicture}
    \begin{axis}[
        Axis Style,
        xtick={
            -6.28318, -4.7123889, -3.14159, -1.5708,
            1.5708, 3.14159, 4.7123889, 6.28318, 7.85398,
            9.42478, 10.99558, 12.56638
        },
        xticklabels={
            $-2\pi$, $-\frac{3\pi}{2}$, $-\pi$, $-\frac{\pi}{2}$,
            $\frac{\pi}{2}$, $\pi$, $\frac{3\pi}{2}$, $2\pi$,
            $\frac{5\pi}{2}$, $3\pi$, $\frac{7\pi}{2}$, $4\pi$
        }]
        \addplot[name path=A, mark=none, thick] {sin(deg(x))*sin(deg(x))};
        \addplot[name path=B]{0};
        \addplot[red!40] fill between[of=A and B,
            soft clip={domain=0:6.28318},];
    \end{axis}
    \end{tikzpicture}
    \begin{tikzpicture}
    \begin{axis}[
        Axis Style,
        xtick={
            -6.28318, -4.7123889, -3.14159, -1.5708,
            1.5708, 3.14159, 4.7123889, 6.28318, 7.85398,
            9.42478, 10.99558, 12.56638
        },
        xticklabels={
            $-2\pi$, $-\frac{3\pi}{2}$, $-\pi$, $-\frac{\pi}{2}$,
            $\frac{\pi}{2}$, $\pi$, $\frac{3\pi}{2}$, $2\pi$,
            $\frac{5\pi}{2}$, $3\pi$, $\frac{7\pi}{2}$, $4\pi$
        }]
        \addplot[name path=C, mark=none, thick] {sin(deg(x +1.5708))*sin(deg(x +1.5708))};
        \addplot[name path=D]{0};
        \addplot[red!40] fill between[of=C and D,
            soft clip={domain=0:6.28318},];           
    \end{axis}
    \end{tikzpicture}    
\caption{A integral como área das funções $f(x)=sen^2(x)$ e $g(x)=cos^2(x)$}
\label{fig:intSenQuad}
\end{figure}

 
Mais adiante, usando fórmulas trigonométricas conhecidas, temos\\
\\
\begin{equation}
\label{eq:54}
    \begin{split}
        cos(\alpha)cos(\beta) & = \dfrac{1}{2}[cos(\alpha + \beta) + cos(\alpha - \beta)]\\
        sen(\alpha)sen(\beta) & = \dfrac{1}{2}[cos(\alpha - \beta) - cos(\alpha + \beta)]
    \end{split}
\end{equation}
\\
para qualquer $n$, $m$.\\
\\
Finalmente, usando a fórmula \\
\begin{equation}
        sen(\alpha)cos(\beta) = \dfrac{1}{2}[\sin{(\alpha + \beta)} + \sin{(\alpha - \beta)}]    
\end{equation}
\\
tiramos o seguinte:\\
\\
\begin{definicao}
    A integral definida em um intervalo de tamanho $T=2\pi$ da função $f(x)=sen(nx)*cos(mx)$
    é sempre igual a zero, para qualquer n, m.
    \begin{equation}
        \int_{0}^{2\pi}\sin{nx}\cos{mx}dx = 0
    \end{equation}
    \label{def:55}
\end{definicao}

A assim como em \ref{def:53}, fica evidente que a área do total da função 
$f(x)=sen(x)*cos(x)$ é igual a zero.
\begin{figure}[H]
    \begin{tikzpicture}
    \begin{axis}[
        Axis Style,
        xtick={
            -6.28318, -4.7123889, -3.14159, -1.5708,
            1.5708, 3.14159, 4.7123889, 6.28318, 7.85398,
            9.42478, 10.99558, 12.56638
        },
        xticklabels={
            $-2\pi$, $-\frac{3\pi}{2}$, $-\pi$, $-\frac{\pi}{2}$,
            $\frac{\pi}{2}$, $\pi$, $\frac{3\pi}{2}$, $2\pi$,
            $\frac{5\pi}{2}$, $3\pi$, $\frac{7\pi}{2}$, $4\pi$
        }]
        \addplot[name path=A, mark=none, thick] {sin(deg(x))*sin(deg(x+1.5708))};
        \addplot[name path=B]{0};
        \addplot[red!40] fill between[of=A and B,
            soft clip={domain=0:1.5708},];
        \addplot[blue!40] fill between[of=A and B,
            soft clip={domain=1.5708:3.14159},];
        \addplot[red!40] fill between[of=A and B,
            soft clip={domain=3.14159:4.7123889},];
        \addplot[blue!40] fill between[of=A and B,
            soft clip={domain=4.7123889:6.28318},];            
    \end{axis}
    \end{tikzpicture}
\caption{A integral como área da função $f(x)=sen(x)*cos(x)$}
\label{fig:intSenCos}
\end{figure}

As fórmulas de \ref{def:52} e \ref{eq:54}, mostram que a integral sobre um intervalo $[0, 2\pi]$
do produto de qualquer duas funções diferentes do sistema \ref{eq:51} desaparece.

Finalmente, definimos
\begin{definicao}
\label{def:56}
    Duas funções $f(x)$ e $g(x)$ são ortogonais em um intervalo $[a,b]$ se
    \begin{equation}
        \int_{a}^{b} f(x)g(x)\hspace{4pt}dx = 0
    \end{equation}
\end{definicao}

Com a definição \ref{def:56}, podemos dizer que os pares de funções em \ref{eq:51} são
ortogonais no intervalo $[0, 2\pi]$.

Como já foi dito na Sec. 1, a integral de uma função periódica é a mesma em
qualquer outro intervalo de mesmo tamanho que o período, no caso $2\pi$.


%\section{Séries de Fourier para funções de período $2\pi$}
Suponha que a função $f(x)$ de período $2\pi$ tenha a seguinte expansão:\\
\begin{equation}
    \label{61}
    f(x) = \dfrac{a_0}{2} + \sum\limits_{k=1}^{\infty}(a_k\cos{kx} + b_k\sin{kx})
\end{equation}
\\
onde, para simplificar para próximas fórmulas, vamos denotar a constante da 
expansão como sendo $\dfrac{a_0}{2}$.\\
\\
Agora, vamos resolver o problema para achar os valores de $a_0, a_k, b_k$, para
k = 1, 2, 3, ..., por um conhecimento em $f(x)$.\\
\\
Para isso, vamos fazer a seguinte suposição:\\
\\
$\to$ A série \ref{61} e a série a seguir podem ser integradas termo a termo, ou seja
a integral das somas é igual a soma das integrais.
\\
\\
Então, integrando \ref{61} no intervalo $[-\pi, \pi]$, ficamos com:\\
\begin{equation}
    \int_{-\pi}^{\pi} f(x)\hspace{5pt}dx = \dfrac{a_0}{2}\int_{-\pi}^{\pi}dx + \sum\limits_{k=1}^{\infty}(a_k\int_{-\pi}^{\pi}\cos{kx}dx + b_k\int_{-\pi}^{\pi}\sin{kx}dx)
\end{equation}
\\
Pela \ref{52}, todas as integrais somem\\
\\
\begin{equation}
    \label{62}
    \int_{-\pi}^{\pi}f(x) dx = \pi a_0
\end{equation}
\\
Agora, vamos multiplicaros dois lados por $\cos{nx}$ e integrar o resultado no 
intervalo $[-\pi, \pi]$, como antes, desta vez temos:\\
\\
\begin{equation}
    \int_{-\pi}^{\pi} f(x)\cos{nx}\hspace{5pt}dx = \dfrac{a_0}{2}\int_{-\pi}^{\pi}\cos{nx}dx + \sum\limits_{k=1}^{\infty}(a_k\int_{-\pi}^{\pi}\cos{kx}\cos{nx}dx + b_k\int_{-\pi}^{\pi}\sin{kx}\cos{nx}dx)
\end{equation}
\\
Pela \ref{52}, todas as integrais desaparecem, com exceção de uma, a de coeficiente $a_n$.\\
\\
\begin{equation}
    \int_{-\pi}^{\pi}\cos^2{nx}dx = \pi
\end{equation}
\\ 
E disso, temos\\
\\
\begin{equation}
\label{63}
    \int_{-\pi}^{\pi}f(x)\cos{nx}dx = a_n\pi
\end{equation}
\\
De mesmo modo\\
\\
\begin{equation}
\label{64}
    \int_{-\pi}^{\pi}f(x)\sin{nx}dx = b_n\pi
\end{equation}
\\
Então, dado \ref{63} e \ref{64}, temos\\
\begin{equation}
\label{eq:65}
    \begin{split}
        a_n &= \dfrac{1}{\pi}\int_{-\pi}^{\pi}f(x)\cos{nx}dx\\
        b_n &= \dfrac{1}{\pi}\int_{-\pi}^{\pi}f(x)\sin{nx}dx
    \end{split}
\end{equation}
\\
Finalmente, se $f(x)$ é integrável e pode ser expandido em uma série trigonométrica,
e se essa série e a série obtida multiplicando por $\cos{nx}$ e $\sin{nx}$ ($n = 1, 2, 3, ...$)
pode ser integrada termo a termo, então os coeficientes $a_n$ e $b_n$ são dados pela
fórmula \ref{eq:65}. Estes coeficientes são conhecidos como \textit{coeficientes de Fourier}
da função $f(x)$, que representa a série trigonométrica conhecida como \textit{Série de
Fourier}.\\
%\chapter{Séries de Fourier para funções definidas em um intervalo de tamanho $2\pi$}

Existem funções que são definidas apenas em um intervalo. Por questão de 
simplicidade, imagina que esse intervalo seja $[-\pi, \pi]$, ou qualquer
outro intervalo de tamanho $2\pi$.

Neste caso, não estamos falando de uma funções periódica, e sim de uma 
função definida em um intervalo fechado, mas ainda assim, é possível
escrever a Série de Fourier para tal funções.

Para isso, temos que observar ao fato de que as funções \ref{eq:65}
envolvem apenas um intervalo de tamanho $2\pi$. Então, podemos 
interpretar a série de Fourier para uma função $f(x)$ definida 
em um intervalo como sendo a repetição de $f(x)$ daquele intervalo
para todo o eixo-x.
\\

Isso pode dar um nó na cabeça, então vamos por partes!
\\

Dado a função $f(x) = x^2$, para $x \in [-\pi,\pi]$. Temos o gráfico da 
função representado pela Figura \ref{fig:functInt}.

\begin{figure}[H]
    \begin{tikzpicture}
    \begin{axis}[
        Axis Style,
        xmin=-pi,
        ymin=-10,
        ymax=10,
        xtick={
            -6.28318, -4.7123889, -3.14159, -1.5708,
            1.5708, 3.14159, 4.7123889, 6.28318, 7.85398,
            9.42478, 10.99558, 12.56638
        },
        xticklabels={
            $-2\pi$, $-\frac{3\pi}{2}$, $-\pi$, $-\frac{\pi}{2}$,
            $\frac{\pi}{2}$, $\pi$, $\frac{3\pi}{2}$, $2\pi$,
            $\frac{5\pi}{2}$, $3\pi$, $\frac{7\pi}{2}$, $4\pi$
        }]
        \addplot[name path=A, mark=none, thick, domain=-pi:pi] {x^2};      
    \end{axis}
    \end{tikzpicture} 

    \caption{Uma função definida em um intervalo de $[-\pi, \pi]$}
    \label{fig:functInt}
\end{figure}

Se expardirmos a função para todo eixo-x, teremos exatamente a série
de Fourier daquela função, ficaria parecido com o gráfico representado 
pela Figura \ref{fig:functIntRep}.

\begin{figure}[H]
    \begin{tikzpicture}
    \begin{axis}[
        Axis Style,
        xmin=-pi,
        ymin=-10,
        ymax=10,
        xtick={
            -6.28318, -4.7123889, -3.14159, -1.5708,
            1.5708, 3.14159, 4.7123889, 6.28318, 7.85398,
            9.42478, 10.99558, 12.56638
        },
        xticklabels={
            $-2\pi$, $-\frac{3\pi}{2}$, $-\pi$, $-\frac{\pi}{2}$,
            $\frac{\pi}{2}$, $\pi$, $\frac{3\pi}{2}$, $2\pi$,
            $\frac{5\pi}{2}$, $3\pi$, $\frac{7\pi}{2}$, $4\pi$
        }]
        \addplot[name path=A, mark=none, thick, domain=-pi:pi] {x^2};
        \addplot[name path=B, mark=none, thick, domain=pi:3*pi] {x^2 - 4*pi*x + 39.5};        
        \addplot[name path=C, mark=none, thick, domain=-pi:5*pi] {x^2 - 8*pi*x + 157.9};        
    \end{axis}
    \end{tikzpicture} 

    \caption{A mesma função expandida para todo eixo-x}
    \label{fig:functIntRep}
\end{figure}

Este caso seria o mais simples, onde $f(-\pi) = f(\pi)$ e portanto
uma expansão periódica dela seria contínua. 

Agora considera a função $g(x) = x$ no mesmo intervalo, teríamos um
gráfico como na Figura \ref{fig:functImpar}

\begin{figure}[H]
    \begin{tikzpicture}
    \begin{axis}[
        Axis Style,
        xmin=-pi,
        ymin=-4,
        ymax=4,
        xtick={
            -6.28318, -4.7123889, -3.14159, -1.5708,
            1.5708, 3.14159, 4.7123889, 6.28318, 7.85398,
            9.42478, 10.99558, 12.56638
        },
        xticklabels={
            $-2\pi$, $-\frac{3\pi}{2}$, $-\pi$, $-\frac{\pi}{2}$,
            $\frac{\pi}{2}$, $\pi$, $\frac{3\pi}{2}$, $2\pi$,
            $\frac{5\pi}{2}$, $3\pi$, $\frac{7\pi}{2}$, $4\pi$
        }]
        \addplot[name path=A, mark=none, thick, domain=-pi:pi] {x};      
    \end{axis}
    \end{tikzpicture} 

    \caption{Uma função definida em um intervalo de $[-\pi, \pi]$}
    \label{fig:functImpar}
\end{figure}

Se simplesmente extender a função como fizemos com $f(x)$, teremos 
um gráfico representado pela Figura \ref{fig:functImparRep}

\begin{figure}[H]
    \begin{tikzpicture}
    \begin{axis}[
        Axis Style,
        xmin=-pi,
        ymin=-4,
        ymax=4,
        xtick={
            -6.28318, -4.7123889, -3.14159, -1.5708,
            1.5708, 3.14159, 4.7123889, 6.28318, 7.85398,
            9.42478, 10.99558, 12.56638
        },
        xticklabels={
            $-2\pi$, $-\frac{3\pi}{2}$, $-\pi$, $-\frac{\pi}{2}$,
            $\frac{\pi}{2}$, \ \ \ $\pi$, $\frac{3\pi}{2}$, $2\pi$,
            $\frac{5\pi}{2}$, \ \ \ \ \ \ $3\pi$, $\frac{7\pi}{2}$, $4\pi$
        }]
        \addplot[name path=A, mark=none, thick, domain=-pi:pi] {x};      
        \addplot [name path=A1,
            color=gray, thick, dashed]
            coordinates {(pi,-pi) (pi,pi) } ;
        \addplot[name path=B, mark=none, thick, domain=pi:3*pi] {x - 2*pi};
        \addplot [name path=B1,
            color=gray, thick, dashed]
            coordinates {(3*pi,-pi) (3*pi,pi) } ;
        \addplot[name path=C, mark=none, thick, domain=3*pi:5*pi] {x - 4*pi};      
    \end{axis}
    \end{tikzpicture} 

    \caption{Uma função definida em um intervalo de $[-\pi, \pi]$}
    \label{fig:functImparRep}
\end{figure}

Então, temos que $g(-\pi) \neq g(\pi)$ e ao extender periodicamente
a função $g(x)$, teríamos descontinuidades nesses pontos. Então, para 
esses valores coincidirem, é preciso alterar os valores da função $g(x)$
nos pontos $x = -\pi$ e $x = \pi$.
\\

\begin{enumerate}
    \item[(i)] Podemos ignorar os valores de $g(x)$ em $x = -\pi$ e 
    $x = \pi$, tornando a função indefinida nesses pontos, e assim
    indefinida nos pontos $x = (2k + 1)\pi$, para $k \in \mathbb{N}$
    \item[(ii)] Podemos modificar os valores de $g(x)$ em $x = -\pi$ e
    $x = \pi$ para que satisfaça $g(-\pi) = g(\pi)$.
\end{enumerate}

Apenas como exemplo, a série de Fourier da função $g(x)$ no intervalo
de $[-\pi,\pi]$ é dada exatamente por 
\begin{equation}
    g(x) = 2\sum\limits_{n=1}^{\infty}\dfrac{(-1)^{n+1}sen(nx)}{n}
\end{equation}
e será explicada com detalhes mais adiante.

Isso acontece que a série de Fourier necessita apenas dos coeficientes,
e estes, por sua vez, necessita apenas de um intervalo definido, e 
portanto, qualquer uma das duas alternativas que escolhermos, a série
de Fourier será a mesma. 
 


Pela definição de funções periódicas \ref{def:functPer}, não teríamos uma 
função periódica válida. 


Temos uma função com período $T = 2\pi$ e 
$g(\pi) \neq g(\pi + 2\pi)$, onde temos a descontinuidade. Então, para estes 
casos, temos que alterar o valor da função $g(x)$ para que a igualdade 
$g(x) = g(x + 2\pi)$ seja válida.


%\chapter{Limites à esquerda, à direita e descontinuidades}

Neste seção, vamos revisar os conceitos de limites à direita
e à esquerda, assim como introduzir a seguinte notação:
\begin{equation}
    \lim_{x\to x_0\text{, }x < x_0} = f(x_0 - 0)\hspace{10pt},\hspace{10pt}\lim_{x\to x_0\text{, }x > x_0} = f(x_0 + 0)
\label{not:limites}
\end{equation}
dado que estes limites existem e são finitos.

O primeiro destes limites é chamado de \textit{limite à esquerda}
de $f(x)$ no ponto $x_0$, e o segundo é chamado de \textit{limite à 
direita} de $f(x)$ no ponto $x_0$.

Se estes limites existem em um ponto $x_0$ onde $f(x)$ é contínuo, 
então
\begin{equation}
\notag
    f(x_0 - 0) = f(x_0) = f(x_0 + 0)
\end{equation}

Caso $x_0$ seja um ponto de descontinuidade de $f(x)$, então os 
limites \ref{not:limites} podem ou não existir (um deles ou ambos).

Se ambos os limites existem, dizemos que o ponto $x_0$ é um ponto
de \textit{descontinuidade de primeira ordem}, ou simplemente, um 
``salto descontínuo".

Se ao menos um desses limites não existir, então o ponto $x_0$ é 
dito ponto de \textit{descontinuidade de segunda ordem}.

Para este curso, estaremos apenas interessados no primeiro caso.
Então, se $x_0$ é um $salto$, então
\begin{equation}
\notag
    \delta = f(x_0 + 0) - f(x_0 - 0)
\end{equation}
é chamado de ``salto" da função $f(x)$ no ponto $x_0$.

Para ilustrar essa situação, suponha que 
\begin{equation}
\notag
    f(x)=\begin{cases}
        -x^3 &\text{ , se }x < 1\\
        0 &\text{ , se }x = 1\\
        \sqrt{x} &\text{ , se }x > 1
    \end{cases}
\end{equation}
representado pela Figura \ref{fig:saltoEx}.

\begin{figure}[H]
    \begin{center}
        \begin{tikzpicture}
        \begin{axis}[
            Axis Style,
            xmin=-2,
            xmax=3,
            width=8cm,
            xtick={
                1
            },
            xticklabels={
                \ \ \ 1
            }]
            \addplot[name path=A, mark=none, thick, domain=-2:1] {-x^3};
            \addplot [name path=border,
                    color=gray, thick, dashed]
                    coordinates {(1,-1) (1,1) } ;
            \addplot[name path=B, mark=none, thick, domain=1:3] {sqrt(x)};
            
        \end{axis}
        \end{tikzpicture}   
    \end{center}
    \caption{Descontinuidade em $x = 1$}
    \label{fig:saltoEx}
\end{figure}

Os limites à direita e à esquerda são, respectivamente:
\begin{equation}
\notag
    f(1 - 0) = -1\text{  e  }f(1 + 0) = 1
\end{equation} 

Portanto, o salto da função em $x = 1$ é
\begin{equation}
\notag
    \delta = f(1 + 0) - f(1 - 0) = 2
\end{equation} 

Disso, podemos supor o seguinte:
\\

Se $f(x)$ é uma função contínua no intervalo $[0, 2\pi]$ e $f(0) \neq f(2\pi)$,
descontinuidades aparecem ao realizar a extensão periódica de $f(x)$ de $[0, 2\pi]$
para todo o eixo-x, e todo os saltos descontínuos serão de tamanho
\begin{equation}
\notag
\begin{split}
    \delta & = f(0 + 0) - f(0 - 0)\\
    &\text{    ou}\\
    \delta & = f(2\pi + 0) - f(2\pi - 0)
\end{split}
\end{equation} 
%\chapter{Funções suaves e semi-suaves}

A função $f(x)$ é dita suave se no intervalo $[a, b]$, $f(x)$
possui derivada contínua em $[a,b]$. Em linguagem geométrica,
isso significa que a direção da tangente muda continuamente,
ou seja sem saltos enquanto se move pela curva de $f(x)$.

Portanto, o gráfico de uma função suave é uma curva ``suave",
i.e. não possui ``cantos" ou ``pontas".

Uma função $f(x)$ é dita ``semi-suave" no intervalo $[a,b]$
caso $f(x)$ e suas derivadas são ambas contínuas em $[a,b]$,
ou se possui um número finito de descontinuidades em $[a,b]$.
A Figura \ref{fig:suave} seria um exemplo de uma função suave.

\begin{figure}[H]
\begin{center}
    
    \begin{tikzpicture}
    \begin{axis}[
        Axis Style,
        ymin=-0.3,
        ymax=3,
        ticks=none,
        width=7cm]
    \addplot [mark=none, thick, domain=pi:3*pi] {sin(deg(x/2)) + 1.5 +sin(deg(x))};
    \end{axis}
    \end{tikzpicture}
\end{center}
    \caption{Uma função dita suave}
    \label{fig:suave}
\end{figure}



\begin{figure}
\centering
\begin{subfigure}{.5\textwidth}
  \centering
    \begin{tikzpicture}
    \begin{axis}[
        Axis Style,
        xmin=-pi,
        ymin=-0.3,
        ymax=4,
        width=7cm,
        ticks=none]
        \addplot[name path=A, mark=none, thick, domain=pi:2*pi] {sin(deg(x)) + 2};      
        \addplot [name path=A1,
            color=gray, thick, dashed]
            coordinates {(pi,0) (pi,2) } ;
        \addplot[name path=B, mark=none, thick, domain=2*pi:3*pi] {sin(deg(x))+2 + sin(deg(2*x))/2};
        \addplot [name path=B1,
            color=gray, thick, dashed]
            coordinates {(2*pi,0) (2*pi,2) } ;
        \addplot[name path=C, mark=none, thick, domain=3*pi:4*pi] {2};
        \addplot [name path=C1,
            color=gray, thick, dashed]
            coordinates {(3*pi,0) (3*pi,2) } ;      
    \end{axis}
    \end{tikzpicture} 
  \caption{Semi-suave contínua}
  \label{fig:semiCont}
\end{subfigure}%
\begin{subfigure}{.5\textwidth}
  \centering
    \begin{tikzpicture}
    \begin{axis}[
        Axis Style,
        xmin=-pi,
        ymin=-0.3,
        ymax=4,
        width=7cm,
        ticks=none]
        \addplot[name path=A, mark=none, thick, domain=pi:2*pi] {sin(deg(x)) + 3};      
        \addplot [name path=A1,
            color=gray, thick, dashed]
            coordinates {(pi,0) (pi,3) } ;
        \addplot[name path=B, mark=none, thick, domain=2*pi:3*pi] {sin(deg(x))+1 + sin(deg(2*x))/2};
        \addplot [name path=B1,
            color=gray, thick, dashed]
            coordinates {(2*pi,0) (2*pi,3) } ;
        \addplot[name path=C, mark=none, thick, domain=3*pi:4*pi] {3};
        \addplot [name path=C1,
            color=gray, thick, dashed]
            coordinates {(3*pi,0) (3*pi,3) } ;      
    \end{axis}
    \end{tikzpicture} 
  \caption{Semi-suave descontínua}
  \label{fig:semiDesc}
\end{subfigure}
\caption{Os dois tipos de funções semi-suaves}
\label{fig:test}
\end{figure}


\begin{definicao}
    Uma função $f(x)$ é \textbf{semi-suave contínua} se possuir
    um número finito de ``pontas" em um intervalo $[a,b]$.
\end{definicao}

\begin{definicao}
    Uma função $f(x)$ é \textbf{semi-suave descontínua} se possuir
    um número finito de descontinuidades em um intervalo $[a,b]$.
\end{definicao}



A função da Figura \ref{fig:semiCont} representa uma função semi-suave
contínua, e a Figura \ref{fig:semiDesc} representa uma função dita 
semi-suave descontínua.
%\chapter{Um critério de convergência para a série de Fourier}

Agora vamos deixar um pouco mais claro como que ficaria o critério
de convergência para uma série de Fourier quando existe finitos
pontos de descontinuidades em um intervalo.

\begin{definicao}
    A Série de Fourier de uma função semi-suave (contínua ou descontínua)
    $f(x)$ de período $2\pi$ converge para todos os valores de $x$. A soma 
    da série é igual a $f(x)$ eem todo ponto de continuidade é igual à
    \begin{equation}
    \notag
        \dfrac{f(x + 0) + f(x - 0)}{2}
    \end{equation}
    média aritmética dos limites à esquerda e à direita, em todos os 
    pontos de descontinuidades. 
\end{definicao}

A Figura \ref{fig:semiConv} seria um exemplo de convergência de uma função
semi-suave descontínua. 

\begin{figure}[H]
\begin{center}
    \begin{tikzpicture}
    \begin{axis}[
        Axis Style,
        ymin=-2,
        ymax=3,
        xmin=-10,
        xmax=15,
        ytick=\empty,
        xtick={
            -3.14159,
            3.14159,
            9.42478
        },
        xticklabels={
            $-\pi$\ \ \ \ \ \ \ ,
            $\pi$\ \ \ \ \ \ ,
            $3\pi$\ \ \ \ \ \ 
        }]
    \addplot [mark=none, thick, domain=3*pi:5*pi] {sin(deg(x/2 + 2*pi)) + 0.5 };
    \addplot [mark=none, thick, domain=pi:3*pi] {sin(deg(x/2 + pi)) + 0.5 };
    \addplot [
        color=gray, thick, dashed]
        coordinates {(-pi,-0.5) (-pi,0.4) } ;
    \addplot [
        color=gray, thick, dashed]
        coordinates {(-pi,0.58) (-pi,1.5) } ;
    \addplot [mark=none, thick, domain=-pi:pi] {sin(deg(x/2)) + 0.5 };
    \addplot [
        color=gray, thick, dashed]
        coordinates {(pi,-0.5) (pi,0.4) } ;
    \addplot [
        color=gray, thick, dashed]
        coordinates {(pi,0.58) (pi,1.5) } ;
    \addplot [mark=none, thick, domain=-3*pi:-pi] {sin(deg(x/2 - pi)) + 0.5 };
    \addplot [
        color=gray, thick, dashed]
        coordinates {(3*pi,-0.5) (3*pi,0.4) } ;
    \addplot [
        color=gray, thick, dashed]
        coordinates {(3*pi,0.58) (3*pi,1.5) } ;

\addplot[
    scatter,
    only marks,
    point meta=explicit symbolic,
    scatter/classes={
        a={mark=o}},
    ]
    table[meta=label] {
        x y label
        -3.14159 0.5 a
        3.14159 0.5 a
        9.42478 0.5 a
    };
    \end{axis}
    \end{tikzpicture}
\end{center}
\caption{A convergência de uma função semi-suave descontínua}
\label{fig:semiConv}
\end{figure}

\begin{definicao}
    Se $f(x)$ é contínuo em todos os pontos, então a série converge 
    absoluta e uniformemente.
\end{definicao}

Suponha que a função $f(x)$ é definida apenas no intervalo $[-\pi, \pi]$,
e é semi-suave nesse intervalo e contínuo nas extremidades. Como 
mencionado anteriormente na seção 7, a série de Fourier de $f(x)$ coincide
com a série de Fourier da função na qual é a extensão periódica de $f(x)$
em todo eixo-x. Mas nesse caso, tal expansão obviamente levaria a uma função 
$f(x)$ que seria semi-suave em todo eixo-x. Pontanto, o critário que 
formulado implica que a série de Fourier de $f(x)$ irá convergir em 
todos os pontos. Particularmente, para $-\pi < x < \pi$, a série 
irá convergir para $f(x)$ em pontos de continuidade e terá valor de
\begin{equation}
\notag
    \dfrac{f(x + 0) + f(x - 0)}{2}
\end{equation}
nos pontos de descontinuidades. Mas o que acontecerá para extremidades
no intervalo $[-\pi, \pi]$?

Nesse caso, dois casos são possíveis:
\begin{enumerate}
    \item $f(-\pi) = f(\pi)$. Então, a extensão periódica obviamente
    resultará em uma função na qual é contínua nos pontos onde $x = (2k + 1)\pi$
    para $k \in \mathbb{N}$. Daí, pelo critério, a série de Fourier irá
    convergir para $f(x)$ nas extremidades de $[-\pi,\pi]$.

    \item  $f(-\pi)\neq f(\pi)$. Neste caso, a extensão periódica 
    resulta em uma função que é descontínua nos pontos onde $x = (2k + 1)\pi$
    para $k \in \mathbb{N}$. Daí teremos
    \begin{equation}
    \notag
        \begin{split}
            f(-\pi - 0) = f(\pi)\text{,   } & f(-\pi + 0) = f(-\pi) \\
            f(\pi + 0) = f(-\pi)\text{,   } & f(\pi - 0) = f(\pi)             
        \end{split}
    \end{equation}
    Disso ficamos com
    \begin{equation}
    \notag
        \begin{rcases}
            \dfrac{f(-\pi + 0) + f(-\pi - 0)}{2}, \\
            \dfrac{f(\pi + 0) + f(\pi - 0)}{2}
        \end{rcases}
        =\dfrac{f(\pi) + f(-\pi)}{2}
    \end{equation}
\end{enumerate}


%\chapter{Funções par e ímpar}

Seja uma função $f(x)$, definida em um intervalo ou em todo eixo-x, 
tal que ela seja simétrica em relação a origem das coordenadas.
\\

\begin{definicao}
\label{def:funcPar}
    Dizemos que uma função é \textit{par}, tal que 
    \begin{equation}
    \notag
        f(-x) = f(x)
    \end{equation} 
    para todo x.    
\end{definicao}

Por exemplo, a função $f(x) = cos(x)$ é uma função par

\begin{figure}[H]
\begin{center}
    \begin{tikzpicture}
    \begin{axis}[
        Axis Style,
        xmin=-2*pi,
        xmax=2*pi,
        width=10cm,
        ticks=none]
    \addplot [mark=none, thick, domain=-(2*pi)+0.5:(2*pi)-0.5] {sin(deg(x+pi/2))};
    \label{senoide}
    \end{axis}
    \end{tikzpicture}
    \caption{Uma função par}
    \label{fig:funcPar}
\end{center}
\end{figure}
ou seja, uma função par é qualquer função que seja simétrica em 
relação ao eixo-y
\\

Portanto, se interpretarmos a integral como sendo uma área, temos

\begin{equation}
\label{eq:intPar}
    \int_{-l}^{l} f(x)dx = 2\int_{0}^{l} f(x)dx
\end{equation}

para qualquer $l$, dado que $f(x)$ é integrável nesse intervalo.

Vale ressaltar que a equação \ref{eq:intPar} é derivada da definição \ref{def:funcPar}.
No exemplo ilustrado na Figura \ref{fig:funcPar}, a integral $\int_{0}^{l} f(x)dx = 0$,
porém existem outras funções como por exemplo $f(x) = cos^2(x)$, que é uma função par
e sua integral não é zero.

\begin{definicao}
\label{def:funcImpar}
    Dizemos que uma função é \textit{ímpar}, tal que 
    \begin{equation}
    \notag
        f(-x) = - f(x)
    \end{equation} 
    para todo x.    
\end{definicao}

Como exemplo de uma função ímpar, temos a própria senóide
\begin{figure}[H]
\begin{center}
    \begin{tikzpicture}
    \begin{axis}[
        Axis Style,
        xmin=-2*pi,
        xmax=2*pi,
        width=10cm,
        ticks=none]
    \addplot [mark=none, thick, domain=-(2*pi)+0.5:(2*pi)-0.5] {sin(deg(x))};
    \label{senoide}
    \end{axis}
    \end{tikzpicture}
    \caption{Uma função ímpar}
    \label{fig:funcImpar}
\end{center}
\end{figure}

Para a função ímpar, sua integral em um intervalo $[-l,l]$ é

\begin{equation}
\label{eq:intImpar}
    \int_{-l}^{l} f(x)dx = 0
\end{equation}

Sendo assim, podemos definir as seguintes propriedades \\
\begin{definicao}
\label{def:opPar}
    \begin{enumerate}
        \item[(a)] O produto de duas funções pares ou duas funções ímpares 
        é uma função par;
        \item[(b)] O produto entre uma função ímpar e uma função par é uma 
        função ímpar;
    \end{enumerate}
\end{definicao}

Para $(a)$, podemos verificar com as funções $f(x) = sen(x) *  sen(2x)$ e 
$g(x) = cos(x) * cos(2x)$.


\begin{figure}[H]
\centering
\begin{subfigure}{.5\textwidth}
  \centering
    \begin{tikzpicture}
    \begin{axis}[
        Axis Style,
        xmin=-2*pi,
        xmax=2*pi,
        ymin=-3,
        ymax=3,
        width=7cm,
        ticks=none]
    \addplot [mark=none, thick, domain=-(2*pi)+0.5:(2*pi)-0.5] {sin(deg(x)) * sin(deg(2*x))};
    \label{senoide}
    \end{axis}
    \end{tikzpicture}
  \caption{Gráfico de $f(x)$}
  \label{fig:multImpar}
\end{subfigure}%
\begin{subfigure}{.5\textwidth}
  \centering

    \begin{tikzpicture}
    \begin{axis}[
        Axis Style,
        xmin=-2*pi,
        xmax=2*pi,
        ymin=-3,
        ymax=3,
        width=7cm,
        ticks=none]
    \addplot [mark=none, thick, domain=-(2*pi)+0.5:(2*pi)-0.5] {sin(deg(x+pi/2)) * sin(deg(2*x+pi/2))};
    \label{senoide}
    \end{axis}
    \end{tikzpicture}
  \caption{Gráfico de $g(x)$}
  \label{fig:multPar}
\end{subfigure}
\label{fig:multFunc}
\end{figure}

\begin{figure}[H]
  \centering
    \begin{tikzpicture}
    \begin{axis}[
        Axis Style,
        xmin=-2*pi,
        xmax=2*pi,
        ymin=-3,
        ymax=3,
        width=7cm,
        ticks=none]
    \addplot [mark=none, thick, domain=-(2*pi)+0.5:(2*pi)-0.5] {sin(deg(x)) * sin(deg(2*x+pi/2))};
    \label{senoide}
    \end{axis}
    \end{tikzpicture}
  \caption{Gráfico de $h(x)$}
  \label{fig:multImparPar}
\end{figure}



%\chapter{Séries de senos e cossenos}

    Seja $f(x)$ uma função \textbf{par} definida no intervalo $[-\pi,\pi]$, ou então
    uma função periódica.\\
    
    A função $h(x) = cos(nx)$, para $n \in \mathbb{N}$, é uma 
    função par, então pela definição \ref{def:opPar}, a função $g(x) = f(x)cos(nx)$ 
    também é par.\\
    
    Por outro lado, a função $h(x) = sen(nx)$, para $n \in \mathbb{N}^*$,
    é ímpar, e assim, a função $g(x) = f(x)sen(nx)$ é também ímpar pela definição 
    \ref{def:opPar}.\\

    Portanto, usando as funções \ref{eq:65}, \ref{eq:intPar} e \ref{eq:intImpar}, temos 
    que os coeficientes de Fourier da função \textbf{par} $g(x)$ são:

    \begin{equation}
    \label{eq:121}
        \begin{split}
            a_n = &\dfrac{1}{\pi}\int_{-\pi}^{\pi}f(x)cos(nx) dx = \dfrac{2}{\pi}\int_{0}^{\pi} f(x)cos(nx) dx \\
            b_n = &\dfrac{1}{\pi}\int_{-\pi}^{\pi}f(x)sen(nx) dx = 0
        \end{split}
    \end{equation}

    Com isso, podemos afirmar que a série de Fourier de funções pares contém apenas $cossenos$, i.e.

    \begin{equation}
        f(x) ~ \dfrac{a_0}{2} + \sum\limits_{n=1}^{\infty}a_n cos(nx),
    \end{equation}

    onde os coeficientes $a_n$ são dados pela fórmula \ref{eq:121}.\\

    Agora, considere outra situação. A mesma função $f(x)$ agora é uma função \textbf{ímpar}.\\

    Sendo assim, como a função $h(x) = cos(nx)$ é uma função par, a função $g(x) = f(x)cos(nx)$
    é uma função ímpar, pela definição \ref{def:opPar}.\\
    
    Por outro lado, sendo $h(x) = sen(nx)$ uma função ímpar, a função $g(x) = f(x)sen(nx)$ é par, pela mesma 
    definição \ref{def:opPar}.\\
    
    Portanto, usando as funções \ref{eq:65}, \ref{eq:intPar} e \ref{eq:intImpar}, temos 
    que os coeficientes de Fourier da função \textbf{ímpar} $g(x)$ são:

    \begin{equation}
    \label{eq:122}
        \begin{split}
            a_n = &\dfrac{1}{\pi}\int_{-\pi}^{\pi}f(x)cos(nx) dx = 0 \\
            b_n = &\dfrac{1}{\pi}\int_{-\pi}^{\pi}f(x)sen(nx) dx = \dfrac{2}{\pi}\int_{0}^{\pi} f(x)cos(nx) dx 
        \end{split}
    \end{equation}

    Com isso, podemos afirmar que a série de Fourier de funções ímpares contém apenas $senos$, i.e.

    \begin{equation}
        f(x) ~ \sum\limits_{n=1}^{\infty}b_n sen(nx),
    \end{equation}

    onde os coeficientes $b_n$ são dados pela fórmula \ref{eq:122}.
\chapter{Exemplos de expansões em séries de Fourier}
\section*{Exemplo 1}
Expandir a função $f(x) = x^2$ definida no intervalo $-\pi < x < \pi$ em 
Série de Fourier. A Figura \ref{fig:functIntRepEx} abaixo representa a função
junto com sua extensão periódica ao longo do eixo-x. A função extendida
é contínua e semi-suave. Portanto, pelo critério de convergência do Cap. 10,
a série de Fourier de $f(x) = x^2$ converge em todo intervalo $[-\pi,\pi]$, e
também na sua extensão periódica fora desse intervalo. Além disso, a convergência
é absoluta e uniforme. 

\begin{figure}[H]
    \begin{tikzpicture}
    \begin{axis}[
        Axis Style,
        xmin=-pi,
        ymin=-3,
        ymax=10,
        ytick style={draw=none},
        yticklabels={},
        xtick={
            -3.14159,
            3.14159,
            9.42478
        },
        xticklabels={
            $-\pi$,
            $\pi$, $3\pi$
        }]
        \addplot[name path=A, mark=none, thick, domain=-pi:pi] {x^2};
        \addplot[name path=B, mark=none, thick, domain=pi:3*pi] {x^2 - 4*pi*x + 39.5};
        \addplot[name path=C, mark=none, thick, domain=-pi:5*pi] {x^2 - 8*pi*x + 157.9};
        \addplot [name path=border,
                color=gray, thick, dashed]
                coordinates {(-pi,0) (-pi,pi*pi) } ;
        \addplot [name path=border,
                color=gray, thick, dashed]
                coordinates {(pi,0) (pi,pi*pi) } ;
        \addplot [name path=border,
                color=gray, thick, dashed]
                coordinates {(3*pi,0) (3*pi,pi*pi) };
    \end{axis}
    \end{tikzpicture} 

    \caption{A mesma função expandida para todo eixo-x}
    \label{fig:functIntRepEx}
\end{figure}

Como a função $f(x)$ é par, não precisamos obter o coeficiente $b_n$ como mostrado 
no capítulo anterior, assim, temos
\begin{equation}
\notag
\label{eq:exemplo1}
    \begin{split}
        a_0 &= \dfrac{2}{\pi} \int\limits_{0}^{\pi}x^2 dx = \dfrac{2}{\pi}\left[\dfrac{x^3}{2}\right]_{x=0}^{x=\pi} = \dfrac{2\pi^2}{3}\\
        \text{Adiante, temos o }a_n &\\
        a_n &= \dfrac{2}{\pi} \int\limits_{0}^{\pi}x^2 cos(nx) dx\\
        \text{Usando a fórmula \ref{eq:43}}&\\
        &= -\dfrac{2}{\pi}\left( \left[\dfrac{x^2sen(nx)}{n}\right]_{x=0}^{x=\pi} - \dfrac{2}{\pi n}\int_{0}^{\pi}\dfrac{2x sen(nx)}{n}dx \right)\\
        sen(n\pi)\text{ é sempre zero e integral de }& cos(nx) = \dfrac{sen(nx)}{n}\\
        &= -\dfrac{4}{\pi n}\int_{0}^{\pi}x sen(nx)dx \\
        &= -\dfrac{4}{\pi n} \left(\left[\dfrac{x cos(nx)}{n}\right]_{x=0}^{x=\pi} - \int_{0}^{\pi}\dfrac{cos(nx)}{n} \right)\\
        &= -\dfrac{4}{\pi n^2} \left(\left[x cos(nx)\right]_{x=0}^{x=\pi} - \int_{0}^{\pi}cos(nx)\right)\\
        &= -\dfrac{4}{\pi n^2} \left(\left[x cos(nx)\right]_{x=0}^{x=\pi} - \dfrac{sen(\pi n)}{n}\right)\\
        &= -\dfrac{4}{\pi n^2} \left(\left[x cos(nx)\right]_{x=0}^{x=\pi}\right)\\
        &= -\dfrac{4}{\pi n^2} \left( - \pi cos(\pi n) + 0 cos(0) \right)\\
        &= \dfrac{4}{n^2}cos(\pi n) = (-1)^n \dfrac{4}{n^2}\\
        \text{e por fim, }b_n = 0 &
    \end{split}
\end{equation}

Assim temos os coeficientes da expansão de Fourier e usando a fórmula \ref{eq:61}, temos

\begin{equation}
    \label{eq:exemplo1SF}
    f(x) = \dfrac{2\pi^2}{3} - 4\sum\limits_{n=1}^{\infty}(-1)^n \dfrac{cos(nx)}{n^2}
\end{equation}
\end{document}