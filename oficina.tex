\documentclass{report}

\renewcommand{\thesection}{}
\renewcommand{\thesubsection}{\arabic{section}.\arabic{subsection}}
\makeatletter
\def\@seccntformat#1{\csname #1ignore\expandafter\endcsname\csname the#1\endcsname\quad}
\let\sectionignore\@gobbletwo
\let\latex@numberline\numberline
\def\numberline#1{\if\relax#1\relax\else\latex@numberline{#1}\fi}
\makeatother

\newcommand{\senoide}{\mbox{$y = \mathcal{A}\sin{(\omega x + \varphi)}$}}

\usepackage[utf8]{inputenc}

\title{Série de Fourier}
\author{Gustavo Higuchi}
\date{\today}

\usepackage{natbib}
\usepackage{graphicx}
\usepackage{amssymb}
\usepackage{amsthm}
\usepackage{amsmath}
\usepackage{color}   %May be necessary if you want to color links
\usepackage[portuguese, ruled, linesnumbered]{algorithm2e}
\usepackage{float}
\usepackage{pgfplots}
\usepackage{caption}
\usepackage{subcaption}
\usepackage[portuguese]{babel}
\usepackage{mwe}
\usepackage{subfig}

\usepgfplotslibrary{fillbetween}
\pgfkeys{/pgfplots/Axis Style/.style={
    width=13.5cm, height=6cm,
    axis x line=center, 
    axis y line=middle, 
    samples=200,
    ymin=-1.5, ymax=1.5,
    xmin=0, xmax=13.0,
    domain=0:4*pi
}}

\newtheorem{definicao}{Definição}
\newtheorem{teorema}{Teorema}

\theoremstyle{definition} 
\newtheorem{propriedade}{Propriedade}
\newtheorem*{exemplo}{Exemplo}


\usepackage{mathtools}
\DeclarePairedDelimiter\ceil{\lceil}{\rceil}
\DeclarePairedDelimiter\floor{\lfloor}{\rfloor}

% usado para linkar cada section na tabela de conteúdo com a respectiva
% página no documento
\usepackage{hyperref}
\hypersetup{
    colorlinks,
    citecolor=black,
    filecolor=black,
    linkcolor=black,
    urlcolor=black,
    linktoc=all
}
% o começo do documento
\begin{document}

% compila o título
\maketitle

% compila a tabela de conteúdos
\tableofcontents
\listoffigures
\newpage


\chapter{Funções periódicas}

\begin{definicao}
\label{def1}
    
Uma função $f(x)$ é dita periódica se existe uma constante $T > 0$, tal que 
\begin{equation}
    f(x + T) = f(x)
\end{equation}
para qualquer $T \in \mathbb{R}$. 
\end{definicao}
Essa constante T é chamada de período da função $f(x)$. As funções periódicas 
mais comuns são $\sin{x}$, $\cos{x}$, $\tan{x}$, etc. Funções periódicas surgem
em muitas aplicações matemáticas e em problemas de física e engenharia.\\

Se plotarmos o gráfico da função $y=f(x)$ em qualquer intervalo fechado 
\mbox{$a \leq x \leq a + T$}, é possível obter o gráfico de $f(x)$ através da 
repitição periódica da porção do gráfico correspondente a \mbox{$a \leq x \leq a + T$}.
Na figura ~\ref{fig:periodExp}, temos uma função periódica de período $T=2\pi$.
\\
\begin{figure}[H]
    \begin{tikzpicture}
    \begin{axis}[
        Axis Style,
        xtick={
            -6.28318, -4.7123889, -3.14159, -1.5708, 1,
            1.5708, 3.14159, 4.7123889, 6.28318, 7.28318, 
            9.42478, 10.99558, 12.56638
        },
        xticklabels={
            $-2\pi$, $-\frac{3\pi}{2}$, $-\pi$, $-\frac{\pi}{2}$, $a$,
            $\frac{\pi}{2}$, $\pi$, $\frac{3\pi}{2}$, $2\pi$, $a+2\pi$,
            $3\pi$, $\frac{7\pi}{2}$, $4\pi$
        }]
    \addplot [mark=none, thick] {sin(deg(x)) + 1/2*sin(deg(2*x)) + 1/4*sin(deg(3*x))};

    \addplot [name path=border,
            color=gray, thick, dashed]
            coordinates {(1,0) (1,1.35) } ;
    \addplot [name path=border,
            color=gray, thick, dashed]
            coordinates {(1+2*3.14159,0) (1+2*3.14159,1.35) };
    \end{axis}  
    \end{tikzpicture} 
    \caption{Observe que $f(a) = f(a + 2\pi)$}
    \label{fig:periodExp}
\end{figure}

Se $T$ é um período da função periódica $f(x)$, então seus múltiplos $2T$, $3T$, $4T$, etc 
também são períodos da função $f(x)$. Isso é verificado facilmente ao inspecionar 
os gráficos de uma função periódica, ou pela série de igualdades:\\

\begin{equation}
\label{prop2}
    f(x) = f(x + T) = f(x + 2T) = f(x + 4T) = ...
\end{equation} 
\\
Assim, temos\\
\begin{definicao}
    Se uma função f(x) possui um período $T$, então $kT$ também é um período de
    f(x), ou seja \textbf{se um período existe, ele não é único}
\end{definicao}

Vamos mostrar que o resultado da soma de duas funções periódicas de período T
é também uma função de período T. Então, dadas as funções $f(x) = sen(x)$ e $g(x) = sen(2x)$,
seus gráficos são, respectivamente:\\
\begin{figure}[H]
    \begin{tikzpicture}
    \begin{axis}[
        Axis Style,
        xtick={
            -6.28318, -4.7123889, -3.14159, -1.5708,
            1.5708, 3.14159, 4.7123889, 6.28318, 7.85398,
            9.42478, 10.99558, 12.56638
        },
        xticklabels={
            $-2\pi$, $-\frac{3\pi}{2}$, $-\pi$, $-\frac{\pi}{2}$,
            $\frac{\pi}{2}$, $\pi$, $\frac{3\pi}{2}$, $2\pi$,
            $\frac{5\pi}{2}$, $3\pi$, $\frac{7\pi}{2}$, $4\pi$
        }]
    \addplot [mark=none, thick] {sin(deg(x))};
    \end{axis}
    \end{tikzpicture}
    \caption{$f(x)=sen(x)$}
    \label{fig:senx}
\end{figure}

\begin{figure}[H]
    \begin{tikzpicture}
    \begin{axis}[
        Axis Style,
        xtick={
            -6.28318, -4.7123889, -3.14159, -1.5708,
            1.5708, 3.14159, 4.7123889, 6.28318, 7.85398,
            9.42478, 10.99558, 12.56638
        },
        xticklabels={
            $-2\pi$, $-\frac{3\pi}{2}$, $-\pi$, $-\frac{\pi}{2}$,
            $\frac{\pi}{2}$, $\pi$, $\frac{3\pi}{2}$, $2\pi$,
            $\frac{5\pi}{2}$, $3\pi$, $\frac{7\pi}{2}$, $4\pi$
        }]
    \addplot [mark=none, thick] {sin(deg(2*x))};
    \label{sen2x}
    \end{axis}
    \end{tikzpicture}
    \caption{$f(x)=sen(2x)$}
    \label{fig:sen2x}
\end{figure}

Assim, temos duas funções periódicas de período $T = 2\pi$, vale notar que o período mínimo 
de $f(x)$, $T_f = 2\pi$, é maior que o período mínimo de $g(x)$, $T_g = \pi$, mas
que ambas as funções tem o período em comum de $T = 2\pi$. Para somar essas duas
funções, basta somar o valor de $f(x)$ para cada valor de x ao valor de $g(x)$ para 
cada valor de x. Então, teremos o seguinte:
\begin{figure}[H]
    \begin{tikzpicture}
    \begin{axis}[
        Axis Style,
        ymin=-2.5,
        ymax=2.5,
        ytick={-2,-1,0,1,2},
        yticklabels={-2,-1,0,1,2},
        xtick={
            -6.28318, -4.7123889, -3.14159, -1.5708,
            1.5708, 3.14159, 4.7123889, 6.28318, 7.85398,
            9.42478, 10.99558, 12.56638
        },
        xticklabels={
            $-2\pi$, $-\frac{3\pi}{2}$, $-\pi$, $-\frac{\pi}{2}$,
            $\frac{\pi}{2}$, $\pi$, $\frac{3\pi}{2}$, $2\pi$,
            $\frac{5\pi}{2}$, $3\pi$, $\frac{7\pi}{2}$, $4\pi$
        }]
    \addplot [mark=none, thick] {sin(deg(x)) + sin(deg(2*x))};
    \label{sen2x}
    \end{axis}
    \end{tikzpicture}
    \caption{Resultado da soma entre a função $f(x)$ e $g(x)$, uma função $h(x)$ com período $T = 2\pi$}
    \label{fig:addExp}
\end{figure}

É possível ver que, a função $f(x)$ possui um período mínimo maior que a função $g(x)$,
assim, \textbf{a função resultante é uma função de período mínimo $T = 2\pi$}, a 
subtração funciona de forma semelhante.\\

Agora podemos afirmar que a soma e a diferença de duas funções 
periódica de período T é também uma função periódica com período T, onde
T é o maior período entre as funções.\\

E quanto à multiplicação e à divisão? Podemos afirmar o mesmo?\\

A resposta é sim, funciona de forma semelhante da soma e da subtração, multiplicando
os valores $f(x)$ por $g(x)$, para todo x. Ficamos com seguinte:
\begin{figure}[H]
    \begin{tikzpicture}
    \begin{axis}[
        Axis Style,
        ymin=-2.5,
        ymax=2.5,
        ytick={-2,-1,0,1,2},
        yticklabels={-2,-1,0,1,2},
        xtick={
            -6.28318, -4.7123889, -3.14159, -1.5708,
            1.5708, 3.14159, 4.7123889, 6.28318, 7.85398,
            9.42478, 10.99558, 12.56638
        },
        xticklabels={
            $-2\pi$, $-\frac{3\pi}{2}$, $-\pi$, $-\frac{\pi}{2}$,
            $\frac{\pi}{2}$, $\pi$, $\frac{3\pi}{2}$, $2\pi$,
            $\frac{5\pi}{2}$, $3\pi$, $\frac{7\pi}{2}$, $4\pi$
        }]
    \addplot [mark=none, thick] {sin(deg(x)) * sin(deg(2*x))};
    \end{axis}
    \end{tikzpicture}
    \caption{Resultado de uma multiplicação entre a função $f(x)$ e $g(x)$, uma função $h(x)$ com período $T = 2\pi$}
    \label{fig:multExp}
\end{figure}

Por mais esquisito que a função fique, podemos observar que a função resultante
permaneceu com o período $T = 2\pi$. Dessa forma, está claro que operações 
de funções que partilham um mesmo período, terá uma função resultante com o mesmo
período.\\

\begin{definicao}
    Seja $f(x)$ e $g(x)$ duas funções periódicas com período em comum $T$, a soma, subtração,
    multiplicação e divisão das duas funções resulta em uma função periódica de cujo
    período mínimo é o maior período entre $f(x)$ e $g(x)$.
\end{definicao}

Agora, outro aspecto importante de se notar é a integral de uma função periódica.
Lembrando que podemos interpretar a integral de uma função como área, no caso 
a área de uma função periódica é a área entre a curva definida pela função $f(x)$
e o eixo-x, áreas acima do eixo-x são positivas, e abaixo do eixo-x são negativas.\\

Se observarmos a figura \ref{fig:int_area}, é possível concluir que ambas as 
áreas são iguais.a área em \textcolor{blue}{azul} e a
área em \textcolor{red}{vermelho} representam as áreas das integrais da função
periódica $f(x)=sen(4x)+sen(2x)$ de período $T=\pi$ para intervalos de tamanho $\pi$. 
\\

\begin{figure}[H]
    \begin{tikzpicture}
    \begin{axis}[
        Axis Style,
        ymin=-2.5,
        ymax=2.5,
        ytick={-2,-1,0,1,2},
        yticklabels={-2,-1,0,1,2},
        xtick={
            -6.28318, -4.7123889, -3.14159, -1.5708,
            1.5708, 3.14159, 4.7123889, 6.28318, 7.85398,
            9.42478, 10.99558, 12.56638
        },
        xticklabels={
            $-2\pi$, $-\frac{3\pi}{2}$, $-\pi$, $-\frac{\pi}{2}$,
            $\frac{\pi}{2}$, $\pi$, $\frac{3\pi}{2}$, $2\pi$,
            $\frac{5\pi}{2}$, $3\pi$, $\frac{7\pi}{2}$, $4\pi$
        }]
        \addplot[name path=A, mark=none, thick] {sin(deg(4*x)) + sin(deg(2*x))};
        \addplot[name path=B]{0};
        \addplot[blue!40] fill between[of=A and B,
            soft clip={domain=0:3.14159},];
        \addplot[red!40] fill between[of=A and B, 
            soft clip={domain=4.712385:7.853975},];
    \end{axis}
    \end{tikzpicture}
\caption{Observe que ambas as áres são iguais}
\label{fig:int_area}
\end{figure}

Disso temos a seguinte definição\\

\begin{definicao}
\label{def:functPer}
    Se f(x) é integrável em um intervalo de tamanho T,
    então é integrável em qualquer outro intervalo de tamanho T, e o valor da integral
    é o mesmo\\
    \begin{equation}
    \label{int_prop1}
        \int_a^{a+T} \! f(x) \, \mathrm{d}x = \int_b^{b+T} \! f(x) \, \mathrm{d}x.
    \end{equation}
    para qualquer a, b. \\
\end{definicao}


Daqui em diante, quando uma função $f(x)$ de período $T$ for integrável, então
ela será integrável em qualquer intervalo de tamanho $T$.%1
\chapter{Harmonicos}
\label{cap:harm}
A função periódica mais simples é $y = \sin{x}$ e se imaginar constanstes $\mathcal{A} = 1$,
$\omega = 1$ e $\varphi = 0$, podemos reescrever a mesma função da seguinte forma:
\begin{equation}
\label{eq:harm}
    \senoide
\end{equation} 
\\
onde $\mathcal{A}$, $\omega$ e $\varphi$ são constantes. Essa função é chamada de função 
\textit{harmonica} de amplitude $\mathcal{A}$, frequência $\omega$ e fase
inicial $\varphi$. Neste caso, o período dessa harmonica é $T = 2\pi / \omega$
\begin{equation}
\label{harm_ex}
    \mathcal{A}\sin{\left[\omega\left(x+\dfrac{2\pi}{\omega}\right) + \varphi\right]} = \mathcal{A}\sin{[(\omega x + \varphi) + 2\pi]} = \mathcal{A}\sin{(\omega x + \varphi)}
\end{equation}


Agora, vamos definir daqui em diante que a frequência seja $\omega > 0$, afinal, 
\mbox{$sen(-a) = - sen (a)$} e examinar o comportamento da função \ref{eq:harm} para 
diferentes valores de amplitude, frequência e fase inicial.
\\
\\


Para $\mathcal{A}=1$, $\omega = 1$ e $\varphi = 0$, temos a curva senóide comum $y = sen(x)$\\
\\

\begin{figure}[H]
    \begin{tikzpicture}
    \begin{axis}[
        Axis Style,
        xtick={
            -6.28318, -4.7123889, -3.14159, -1.5708,
            1.5708, 3.14159, 4.7123889, 6.28318, 7.85398,
            9.42478, 10.99558, 12.56638
        },
        xticklabels={
            $-2\pi$, $-\frac{3\pi}{2}$, $-\pi$, $-\frac{\pi}{2}$,
            $\frac{\pi}{2}$, $\pi$, $\frac{3\pi}{2}$, $2\pi$,
            $\frac{5\pi}{2}$, $3\pi$, $\frac{7\pi}{2}$, $4\pi$
        }]
    \addplot [mark=none, thick] {sin(deg(x))};
    \label{senoide}
    \end{axis}
    \end{tikzpicture}
    \caption{Gráfico de uma senóide comum $f(x) = sen(x)$}
    \label{fig:senoide}
\end{figure}

Agora, se considerarmos o harmonico $g(x) = sen(wx)$ e se atribuirmos um $\omega < 1$,
teríamos um harmonico representado pela figura \ref{fig:compSen}. É possível observar
que o gráfico é uma senóide comum que sofreu uma \textit{``compressão"}.
\begin{figure}[H]
    \begin{tikzpicture}
    \begin{axis}[
        Axis Style,
        xtick={
            -6.28318, -4.7123889, -3.14159, -1.5708,
            1.5708, 3.14159, 4.7123889, 6.28318, 7.85398,
            9.42478, 10.99558, 12.56638
        },
        xticklabels={
            $-2\pi$, $-\frac{3\pi}{2}$, $-\pi$, $-\frac{\pi}{2}$,
            $\frac{\pi}{2}$, $\pi$, $\frac{3\pi}{2}$, $2\pi$,
            $\frac{5\pi}{2}$, $3\pi$, $\frac{7\pi}{2}$, $4\pi$
        }]
    \addplot [mark=none, thick] {sin(deg(2*x))};
    \end{axis}
    \end{tikzpicture}
    \caption{Gráfico da função $g(x) = sen\left(\dfrac{x}{2}\right)$}
    \label{fig:compSen}
\end{figure}

Por outro lado, se atribuirmos um $\omega > 1$, teríamos um harmonico 
representado pela figura \ref{fig:expSen}. É possível observar
que o gráfico é uma senóide comum que sofreu uma \textit{``expansão"}.

\begin{figure}[H]
    \begin{tikzpicture}
    \begin{axis}[
        Axis Style,
        xtick={
            -6.28318, -4.7123889, -3.14159, -1.5708,
            1.5708, 3.14159, 4.7123889, 6.28318, 7.85398,
            9.42478, 10.99558, 12.56638
        },
        xticklabels={
            $-2\pi$, $-\frac{3\pi}{2}$, $-\pi$, $-\frac{\pi}{2}$,
            $\frac{\pi}{2}$, $\pi$, $\frac{3\pi}{2}$, $2\pi$,
            $\frac{5\pi}{2}$, $3\pi$, $\frac{7\pi}{2}$, $4\pi$
        }]
    \addplot [mark=none, thick] {sin(deg((x/2))};
    \end{axis}
    \end{tikzpicture}
    \caption{Gráfico da função $g(x) = sen\left(\dfrac{x}{2}\right)$}
    \label{fig:expSen}
\end{figure}

Agora considere o harmonico $h(x) = sen(\omega x + \varphi)$, o gráfico 
de $h(x)$ é obtido deslocando o gráfico de $f(x) = sen(\omega x)$ ao
longo do eixo x por $-\varphi/\omega$. Ou seja, dado $\omega = 1$ e 
$\varphi = 1/2$, teremos a curva que representa o $cos(x)$ representado
pela figura \ref{fig:deslocSen} que nada mais é que a senóide  comum 
deslocada para esquerda.\\

\begin{figure}[H]
    \begin{tikzpicture}
    \begin{axis}[
        Axis Style,
        xtick={
            -6.28318, -4.7123889, -3.14159, -1.5708,
            1.5708, 3.14159, 4.7123889, 6.28318, 7.85398,
            9.42478, 10.99558, 12.56638
        },
        xticklabels={
            $-2\pi$, $-\frac{3\pi}{2}$, $-\pi$, $-\frac{\pi}{2}$,
            $\frac{\pi}{2}$, $\pi$, $\frac{3\pi}{2}$, $2\pi$,
            $\frac{5\pi}{2}$, $3\pi$, $\frac{7\pi}{2}$, $4\pi$
        }]
    \addplot [mark=none, thick] {sin(deg(x+(pi/2)))};
    \end{axis}
    \end{tikzpicture}
    \caption{Gráfico da função $h(x) = seg\left(x + \dfrac{\pi}{2}\right)$}
    \label{fig:deslocSen}
\end{figure}

Finalmente, o gráfico \senoide é obtido multiplicando todas as ordenadas do gráfico de 
$f(x) = sen(x)$ por $\mathcal{A}$, então dado $\mathcal{A} = 2$, $\omega = 1$ e 
$\varphi = 0$, temos

\begin{figure}[H]
    \begin{tikzpicture}
    \begin{axis}[
        Axis Style,
        ymin=-2.5,
        ymax=2.5,
        ytick={-2,-1,0,1,2},
        xtick={
            -6.28318, -4.7123889, -3.14159, -1.5708,
            1.5708, 3.14159, 4.7123889, 6.28318, 7.85398,
            9.42478, 10.99558, 12.56638
        },
        xticklabels={
            $-2\pi$, $-\frac{3\pi}{2}$, $-\pi$, $-\frac{\pi}{2}$,
            $\frac{\pi}{2}$, $\pi$, $\frac{3\pi}{2}$, $2\pi$,
            $\frac{5\pi}{2}$, $3\pi$, $\frac{7\pi}{2}$, $4\pi$
        }]
    \addplot [mark=none, thick] {2*sin(deg(x))};
    \end{axis}
    \end{tikzpicture}
    \caption{Gráfico da função $y = 2sen(x)$}
    \label{fig:ampSen}
\end{figure}

Todas funções anteriores são ditas \textit{harmonicas}, ou seja, é possível obtê-las
a partir da senóide comum. Disso, tiramos a seguinte definição\\

\begin{definicao}
\label{def:harmonico}
    O gráfico de uma harmonica é obtido do gráfico da curva senóide 
    comum por uma compressão (ou expansão) uniforme ao longo dos eixos,
    mais um deslocamento ao longo do eixo x, e é dado pela equação \ref{eq:harm}
\end{definicao}


Assim, podemos utilizar uma conhecida fórmula matemática para derivar
o seguinte:\\
\begin{equation}
    \mathcal{A}sen(\omega x + \varphi) = \mathcal{A}(cos(\omega x)sen(\varphi) + sen(\omega x)cos(\varphi)
\end{equation}
Disso, temos\\
\begin{equation}
    a = \mathcal{A}sen(\varphi)\text{\hspace{10pt},\hspace{10pt}}b = \mathcal{A}cos(\varphi)
\label{eq:ab_harm}
\end{equation}
e então podemos dizer que todo harmonico pode ser representado na forma
\begin{equation}
    y = a \hspace{1pt}cos(\omega x) + b\hspace{1pt}sen(\omega x)
\label{eq:harmSimpl}
\end{equation}
\\
Do mesmo jeito que uma função com a forma \ref{eq:harmSimpl} é um harmonico também. 
Para provar isso, basta resolver \ref{eq:ab_harm} para $a$ e $b$. Temos
\begin{equation}
    \begin{split}
        A = \sqrt{a^2 + b^2}\hspace{5pt},\hspace{10pt} &sen(\varphi) = \dfrac{a}{A} = \dfrac{a}{\sqrt{a^2 + b^2}}\\
        e\hspace{10pt} & cos(\varphi) = \dfrac{b}{A} = \dfrac{b}{\sqrt{a^2 + b^2}}
    \end{split}
\end{equation} 
do qual $\varphi$ pode ser encontrado.\\

Assim, podemos escrever os harmonicos na forma \ref{eq:harmSimpl}. Na Figura \ref{fig:ampSen},
teríamos

\begin{equation}
    a = 2sen(0) = 0\text{, e } b = 2cos(0) = 2
\end{equation}

Usaremos a notação dada pela equação \ref{eq:harmSimpl} como a definição de um harmonico.\\


\texttt{\textbf{Não sei se precisa daqui para baixo --------------------------------------}}
Também será conviniente explicitar o período $T$ em \ref{eq:harmSimpl}. Se definirmos
$T = 2l$, então, como $T = 2\pi/\omega$, temos
\begin{equation}
    \notag
    \omega = \dfrac{2\pi}{T}=\dfrac{\pi}{l}
\end{equation}
e assim, o harmonico com período $T=2l$ pode ser escrito da seguinte forma\\
\begin{equation}
    a\cos{\dfrac{\pi x}{l}} + b\sin{\dfrac{\pi x}{l}}
\end{equation}%2
\chapter{Polinômios trigonométricos e séries}
Dado o período $T=2l$, considere os harmonicos\\
\begin{equation}
    a_k\cos{\dfrac{\pi kx}{l}} + b_k\sin{\dfrac{\pi kx}{l}},\text{\hspace{5pt}para k = 1,2,3,...}
\end{equation}
\\
Com frequencia $\omega_k = k\pi/l$ e períodos $T_k = \dfrac{2\pi}{\omega_k} = \dfrac{2l}{k}$. 
Uma vez que 
\begin{equation}
\notag
    T = 2l = kT_k
\end{equation}  
\\
\textbf{o número $T=2l$ é simultaneamente o período de todos os harmonicos},
pois um múltiplo de um período é também um período (Sec 1). Então, toda soma na 
forma\\
\begin{equation}
    s_n(n) = \mathcal{A} + \sum\limits_{k=1}^{n}(a_k\cos{\dfrac{k\pi x}{l} + b_k\sin{\dfrac{k\pi x}{l}}})
\end{equation}
\\
é uma função de período $2l$, uma vez que é uma soma de funções de período 
$2l$. Vale notar que $A$ é uma constante e não afeta a periodicidade da função,
inclusive é possível considerar que uma constante é uma função periódica, onde 
qualquer valor pode ser um período.\\
\\
Essa função $s_n(x)$ é chamada  de \textbf{polinômio trigonométrico de ordem n}(
e período $2l$).\\
\\
Por mais que seja a soma de vários harmonicos, um polinômio trigonométrico pode 
ser usado para representar uma função de natureza muito mais complexa que a 
de um harmonico. E geralmente é o caso. Escolhendo as constantes corretamente,
podemos formar funções com gráficos bem diferentes de um simples harmonico.
\\
Na primeira Figura \ref{fig:periodExp}, o polínomio que representa aquele gráfico é\\
\begin{equation}
    y = \sin{x} + \dfrac{1}{2}\sin{2x} + \dfrac{1}{4}\sin{3x}
\end{equation}
\\
Facilmente verificado com algum software de plot de função.\\
\\

A \textbf{série trigonométrica infinita}\\ 
\begin{equation}
    f(x) = A + \sum\limits_{k=1}^{\infty}(a_k\cos{\dfrac{k\pi x}{l}} + b_k\sin{\dfrac{k\pi x}{l}})
\label{eq:serie_inf}
\end{equation}
também representa uma função de período $2l$. As funções como \ref{eq:serie_inf} podem
ser usadas para representar fenômenos de origem muito mais complexa que um polinomio.

Sendo assim, o gráfico de uma função periódica $f(x)$ pode ser obtido através da 
sobreposição de todos os harmonicos que o compõe, i.e., pode ser representado
como uma soma de harmonicos simples.
Então a pergunta que fica é:\\
\textit{Qualquer função que tenha período 2l pode ser representado por uma soma de séries 
trigonométricas?}\\
\\
A resposta é sim, e na realidade, é possível ser usado em grande quantidade de problemas!
Diversos outros fenômenos podem ser ser representado por uma série trigonométrica.
\\
\\  

\textbf{Se}
\begin{equation}
    f(x) = A + \sum\limits_{k=1}^{\infty}(a_k\cos{\dfrac{k\pi x}{l}} + b_k\sin{\dfrac{k\pi x}{l}})
\label{eq:serieLonga}
\end{equation}

\textbf{Então}, podemos definir, por comodidade, que $\dfrac{\pi x}{l} = t$ ou que $x = \dfrac{tl}{\pi}$,
assim teremos\\
\begin{equation}
    g(t) = f(tl/\pi) = A + \sum\limits_{k=1}^{\infty}(a_k\cos{kt} + b_k\sin{kt})
\label{eq:serieSimples}
\end{equation}
\\
onde os harmonicos dessa série tenham período $2\pi$. 

É possível verificar que se a função $f(x)$ de período $2l$ possui a expansão 
\ref{eq:serieLonga}, então a função $g(x)$ de período $2\pi$ possui a expansão 
\ref{eq:serieSimples}, e que o contrário é verdadeiro também. 

Por ser mais legível, daqui em diante usaremos a expansão \ref{eq:serieSimples} 
e ao final faremos a tradução para o mais genérico \ref{eq:serieLonga}.\\%3
\input{./src/4-terminologia.tex}%4
\input{./src/5-revisaoTrigonometrica.tex}%5
\chapter{Séries de Fourier para funções de período $2\pi$}
Suponha que a função $f(x)$ de período $2\pi$ tenha a seguinte expansão:\\
\begin{equation}
    f(x) = \dfrac{a_0}{2} + \sum\limits_{k=1}^{\infty}(a_k\cos{kx} + b_k\sin{kx})
\label{eq:61}
\end{equation}
\\
onde, para simplificar para próximas fórmulas, vamos denotar a constante da 
expansão como sendo $\dfrac{a_0}{2}$.\\
\\
Agora, vamos resolver o problema para achar os valores de $a_0, a_k, b_k$, para
\mbox{$k \in \mathbb{N}$}, por um conhecimento em $f(x)$.\\
\\
Para isso, vamos fazer a seguinte suposição:\\
\\
$\to$ A série \ref{eq:61} e a série a seguir podem ser integradas termo a termo, ou seja
a integral das somas é igual a soma das integrais.
\\
\\
Então, integrando \ref{eq:61} no intervalo $[0, 2\pi]$, ficamos com:\\
\begin{equation}
\begin{split}
    \int_{0}^{2\pi} f(x)\hspace{5pt}dx = &\\
     &\dfrac{a_0}{2}\int_{0}^{2\pi}dx + \sum\limits_{k=1}^{\infty}(a_k\int_{0}^{2\pi}\cos{kx}dx + b_k\int_{0}^{2\pi}\sin{kx}dx)
\end{split}
\end{equation}

Pela Definição \ref{def:52}, todas as integrais somem, ficando apenas com a parte 
constante:

\begin{equation}
    \int_{0}^{2\pi}f(x) dx = \pi a_0
    \label{eq:62}
\end{equation}

Agora, se multiplicarmos os dois lados por $cos(nx)$ e integrar o resultado no 
intervalo $[0, 2\pi]$ como antes, desta vez teremos:\\

\begin{equation}
\begin{split}
    \int_{0}^{2\pi} f(x)\cos{nx}\hspace{5pt}dx = &\dfrac{a_0}{2}\int_{0}^{2\pi}\cos{nx}dx + \\
    &\sum\limits_{k=1}^{\infty}(a_k\int_{0}^{2\pi}\cos{kx}\cos{nx}dx + b_k\int_{0}^{2\pi}\sin{kx}\cos{nx}dx)
\end{split}
\end{equation}
\\
Pela Definição \ref{def:52}, todas as integrais desaparecem, com exceção de uma, a 
de coeficiente $a_n$.

\begin{equation}
\notag
    \int_{0}^{2\pi}cos^2(nx)dx = \pi
\end{equation}
\\ 
E disso, temos\\
\\
\begin{equation}
\label{eq:63}
    \int_{0}^{2\pi}f(x)cos(nx)dx = a_n\pi
\end{equation}
\\
De mesmo modo\\
\\
\begin{equation}
\label{eq:64}
    \int_{0}^{2\pi}f(x)\sin{nx}dx = b_n\pi
\end{equation}
\\
Então, dado \ref{eq:63} e \ref{eq:64}, temos\\
\begin{equation}
\label{eq:65}
    \begin{split}
        a_n &= \dfrac{1}{\pi}\int_{0}^{2\pi}f(x)cos(nx)dx\\
        b_n &= \dfrac{1}{\pi}\int_{0}^{2\pi}f(x)sen(nx)dx
    \end{split}
\end{equation}
\\
Finalmente, se $f(x)$ é integrável e pode ser expandido em uma série trigonométrica,
e se essa série e a série obtida multiplicando por $\cos{nx}$ e $\sin{nx}$ ($n = 1, 2, 3, ...$)
pode ser integrada termo a termo, então os coeficientes $a_n$ e $b_n$ são dados pela
fórmula \ref{eq:65}. Estes coeficientes são conhecidos como \textit{coeficientes de Fourier}
da função $f(x)$, que representa a série trigonométrica conhecida como \textit{Série de
Fourier} de $f(x)$.\\

\begin{teorema}
    Se uma função $f(x)$ de período $2\pi$ pode ser expandida
    em uma série trigonométrica na qual converge uniformemente 
    em todo eixo-x, então essa é uma Série de Fourier de $f(x)$.
\end{teorema}
\begin{proof}
    Supondo que $f(x)$ satisfaz \ref{eq:61}, onde as séries são
    uniformemente convergente. \\
    \\
    Temos então, pelo Teorema \ref{teo:unifConv}, que $f(x)$ é
    contínuo e integrável termo a termo. Isso nos dá a fórmula
    \ref{eq:62}.\\

    Analisando a igualdade
    \begin{equation}
        f(x)cos(nx) = \dfrac{a_0 cos(nx)}{2} + \sum\limits_{k=1}^{\infty}(a_kcos(kx)cos(nx) + b_ksen(kx)cos(nx))
    \label{eq:67}
    \end{equation} 
    e mostrar que a série à direita é uniformemente convergente setando
    \begin{equation}
        s_m(x) = \dfrac{a_0}{2} + \sum\limits_{k=1}^{\infty}(a_kcos(kx)+b_ksen(kx))
    \end{equation}
    e seja $\epsilon$ um número arbitrário positivo.

    Se a série \ref{eq:61} converge uniformemente, então existe
    um número $\mathcal{N}$ tal que 
    \begin{equation}
        |f(x) - s_m(x)| \leq \epsilon
    \end{equation}
    para todo $m \geq \mathcal{N}$. 

    O produto $s_m(x)cos(nx)$ é obviamente a m-ésima soma parcial
    da série \ref{eq:67}, então a desigualdade
    \begin{equation}
        |f(x)cos(nx) - s_m(x)cos(nx)| = |f(x) - s_m(x)||cos(nx)| \leq \epsilon
    \end{equation}
    é verdade para todo $m \geq \mathcal{N}$.

    Com isso, podemos dizer que a série \ref{eq:67} converge uniformemente.
    E disso temos que a série pode ser integrada termo a termo, o resultado
    disso é a equação \ref{eq:63}. De modo semelhante, provamos a fórmula
    \ref{eq:64}.

    Finalmente, as fórmulas \ref{eq:65} são válidas para os coeficientes $a_n$
    e $b_n$, que significa que a fórmula \ref{eq:61} é a Série de Fourier de $f(x)$.
\end{proof}%6
\chapter{Séries de Fourier para funções definidas em um intervalo de tamanho $2\pi$}

Existem funções que são definidas apenas em um intervalo. Por questão de 
simplicidade, imagina que esse intervalo seja $[-\pi, \pi]$, ou qualquer
outro intervalo de tamanho $2\pi$.

Neste caso, não estamos falando de uma funções periódica, e sim de uma 
função definida em um intervalo fechado, mas ainda assim, é possível
escrever a Série de Fourier para tal funções.

Para isso, temos que observar ao fato de que as funções \ref{eq:65}
envolvem apenas um intervalo de tamanho $2\pi$. Então, podemos 
interpretar a série de Fourier para uma função $f(x)$ definida 
em um intervalo como sendo a repetição de $f(x)$ daquele intervalo
para todo o eixo-x.
\\

Isso pode dar um nó na cabeça, então vamos por partes!
\\

Dado a função $f(x) = x^2$, para $x \in [-\pi,\pi]$. Temos o gráfico da 
função representado pela Figura \ref{fig:functInt}.

\begin{figure}[H]
    \begin{tikzpicture}
    \begin{axis}[
        Axis Style,
        xmin=-pi,
        ymin=-10,
        ymax=10,
        xtick={
            -6.28318, -4.7123889, -3.14159, -1.5708,
            1.5708, 3.14159, 4.7123889, 6.28318, 7.85398,
            9.42478, 10.99558, 12.56638
        },
        xticklabels={
            $-2\pi$, $-\frac{3\pi}{2}$, $-\pi$, $-\frac{\pi}{2}$,
            $\frac{\pi}{2}$, $\pi$, $\frac{3\pi}{2}$, $2\pi$,
            $\frac{5\pi}{2}$, $3\pi$, $\frac{7\pi}{2}$, $4\pi$
        }]
        \addplot[name path=A, mark=none, thick, domain=-pi:pi] {x^2};      
    \end{axis}
    \end{tikzpicture} 

    \caption{Uma função definida em um intervalo de $[-\pi, \pi]$}
    \label{fig:functInt}
\end{figure}

Se expardirmos a função para todo eixo-x, teremos exatamente a série
de Fourier daquela função, ficaria parecido com o gráfico representado 
pela Figura \ref{fig:functIntRep}.

\begin{figure}[H]
    \begin{tikzpicture}
    \begin{axis}[
        Axis Style,
        xmin=-pi,
        ymin=-10,
        ymax=10,
        xtick={
            -6.28318, -4.7123889, -3.14159, -1.5708,
            1.5708, 3.14159, 4.7123889, 6.28318, 7.85398,
            9.42478, 10.99558, 12.56638
        },
        xticklabels={
            $-2\pi$, $-\frac{3\pi}{2}$, $-\pi$, $-\frac{\pi}{2}$,
            $\frac{\pi}{2}$, $\pi$, $\frac{3\pi}{2}$, $2\pi$,
            $\frac{5\pi}{2}$, $3\pi$, $\frac{7\pi}{2}$, $4\pi$
        }]
        \addplot[name path=A, mark=none, thick, domain=-pi:pi] {x^2};
        \addplot[name path=B, mark=none, thick, domain=pi:3*pi] {x^2 - 4*pi*x + 39.5};        
        \addplot[name path=C, mark=none, thick, domain=-pi:5*pi] {x^2 - 8*pi*x + 157.9};        
    \end{axis}
    \end{tikzpicture} 

    \caption{A mesma função expandida para todo eixo-x}
    \label{fig:functIntRep}
\end{figure}

Este caso seria o mais simples, onde $f(-\pi) = f(\pi)$ e portanto
uma expansão periódica dela seria contínua. 

Agora considera a função $g(x) = x$ no mesmo intervalo, teríamos um
gráfico como na Figura \ref{fig:functImpar}

\begin{figure}[H]
    \begin{tikzpicture}
    \begin{axis}[
        Axis Style,
        xmin=-pi,
        ymin=-4,
        ymax=4,
        xtick={
            -6.28318, -4.7123889, -3.14159, -1.5708,
            1.5708, 3.14159, 4.7123889, 6.28318, 7.85398,
            9.42478, 10.99558, 12.56638
        },
        xticklabels={
            $-2\pi$, $-\frac{3\pi}{2}$, $-\pi$, $-\frac{\pi}{2}$,
            $\frac{\pi}{2}$, $\pi$, $\frac{3\pi}{2}$, $2\pi$,
            $\frac{5\pi}{2}$, $3\pi$, $\frac{7\pi}{2}$, $4\pi$
        }]
        \addplot[name path=A, mark=none, thick, domain=-pi:pi] {x};      
    \end{axis}
    \end{tikzpicture} 

    \caption{Uma função definida em um intervalo de $[-\pi, \pi]$}
    \label{fig:functImpar}
\end{figure}

Se simplesmente extender a função como fizemos com $f(x)$, teremos 
um gráfico representado pela Figura \ref{fig:functImparRep}

\begin{figure}[H]
    \begin{tikzpicture}
    \begin{axis}[
        Axis Style,
        xmin=-pi,
        ymin=-4,
        ymax=4,
        xtick={
            -6.28318, -4.7123889, -3.14159, -1.5708,
            1.5708, 3.14159, 4.7123889, 6.28318, 7.85398,
            9.42478, 10.99558, 12.56638
        },
        xticklabels={
            $-2\pi$, $-\frac{3\pi}{2}$, $-\pi$, $-\frac{\pi}{2}$,
            $\frac{\pi}{2}$, \ \ \ $\pi$, $\frac{3\pi}{2}$, $2\pi$,
            $\frac{5\pi}{2}$, \ \ \ \ \ \ $3\pi$, $\frac{7\pi}{2}$, $4\pi$
        }]
        \addplot[name path=A, mark=none, thick, domain=-pi:pi] {x};      
        \addplot [name path=A1,
            color=gray, thick, dashed]
            coordinates {(pi,-pi) (pi,pi) } ;
        \addplot[name path=B, mark=none, thick, domain=pi:3*pi] {x - 2*pi};
        \addplot [name path=B1,
            color=gray, thick, dashed]
            coordinates {(3*pi,-pi) (3*pi,pi) } ;
        \addplot[name path=C, mark=none, thick, domain=3*pi:5*pi] {x - 4*pi};      
    \end{axis}
    \end{tikzpicture} 

    \caption{Uma função definida em um intervalo de $[-\pi, \pi]$}
    \label{fig:functImparRep}
\end{figure}

Então, temos que $g(-\pi) \neq g(\pi)$ e ao extender periodicamente
a função $g(x)$, teríamos descontinuidades nesses pontos. Então, para 
esses valores coincidirem, é preciso alterar os valores da função $g(x)$
nos pontos $x = -\pi$ e $x = \pi$.
\\

\begin{enumerate}
    \item[(i)] Podemos ignorar os valores de $g(x)$ em $x = -\pi$ e 
    $x = \pi$, tornando a função indefinida nesses pontos, e assim
    indefinida nos pontos $x = (2k + 1)\pi$, para $k \in \mathbb{N}$
    \item[(ii)] Podemos modificar os valores de $g(x)$ em $x = -\pi$ e
    $x = \pi$ para que satisfaça $g(-\pi) = g(\pi)$.
\end{enumerate}

Apenas como exemplo, a série de Fourier da função $g(x)$ no intervalo
de $[-\pi,\pi]$ é dada exatamente por 
\begin{equation}
    g(x) = 2\sum\limits_{n=1}^{\infty}\dfrac{(-1)^{n+1}sen(nx)}{n}
\end{equation}
e será explicada com detalhes mais adiante.

Isso acontece que a série de Fourier necessita apenas dos coeficientes,
e estes, por sua vez, necessita apenas de um intervalo definido, e 
portanto, qualquer uma das duas alternativas que escolhermos, a série
de Fourier será a mesma. 
 


Pela definição de funções periódicas \ref{def:functPer}, não teríamos uma 
função periódica válida. 


Temos uma função com período $T = 2\pi$ e 
$g(\pi) \neq g(\pi + 2\pi)$, onde temos a descontinuidade. Então, para estes 
casos, temos que alterar o valor da função $g(x)$ para que a igualdade 
$g(x) = g(x + 2\pi)$ seja válida.

%7
\input{./src/8-limites.tex}%8
\chapter{Funções suaves e semi-suaves}

A função $f(x)$ é dita suave se no intervalo $[a, b]$, $f(x)$
possui derivada contínua em $[a,b]$. Em linguagem geométrica,
isso significa que a direção da tangente muda continuamente,
ou seja sem saltos enquanto se move pela curva de $f(x)$.

Portanto, o gráfico de uma função suave é uma curva ``suave",
i.e. não possui ``cantos" ou ``pontas".

Uma função $f(x)$ é dita ``semi-suave" no intervalo $[a,b]$
caso $f(x)$ e suas derivadas são ambas contínuas em $[a,b]$,
ou se possui um número finito de descontinuidades em $[a,b]$.
A Figura \ref{fig:suave} seria um exemplo de uma função suave.

\begin{figure}[H]
\begin{center}
    
    \begin{tikzpicture}
    \begin{axis}[
        Axis Style,
        ymin=-0.3,
        ymax=3,
        ticks=none,
        width=7cm]
    \addplot [mark=none, thick, domain=pi:3*pi] {sin(deg(x/2)) + 1.5 +sin(deg(x))};
    \end{axis}
    \end{tikzpicture}
\end{center}
    \caption{Uma função dita suave}
    \label{fig:suave}
\end{figure}



\begin{figure}
\centering
\begin{subfigure}{.5\textwidth}
  \centering
    \begin{tikzpicture}
    \begin{axis}[
        Axis Style,
        xmin=-pi,
        ymin=-0.3,
        ymax=4,
        width=7cm,
        ticks=none]
        \addplot[name path=A, mark=none, thick, domain=pi:2*pi] {sin(deg(x)) + 2};      
        \addplot [name path=A1,
            color=gray, thick, dashed]
            coordinates {(pi,0) (pi,2) } ;
        \addplot[name path=B, mark=none, thick, domain=2*pi:3*pi] {sin(deg(x))+2 + sin(deg(2*x))/2};
        \addplot [name path=B1,
            color=gray, thick, dashed]
            coordinates {(2*pi,0) (2*pi,2) } ;
        \addplot[name path=C, mark=none, thick, domain=3*pi:4*pi] {2};
        \addplot [name path=C1,
            color=gray, thick, dashed]
            coordinates {(3*pi,0) (3*pi,2) } ;      
    \end{axis}
    \end{tikzpicture} 
  \caption{Semi-suave contínua}
  \label{fig:semiCont}
\end{subfigure}%
\begin{subfigure}{.5\textwidth}
  \centering
    \begin{tikzpicture}
    \begin{axis}[
        Axis Style,
        xmin=-pi,
        ymin=-0.3,
        ymax=4,
        width=7cm,
        ticks=none]
        \addplot[name path=A, mark=none, thick, domain=pi:2*pi] {sin(deg(x)) + 3};      
        \addplot [name path=A1,
            color=gray, thick, dashed]
            coordinates {(pi,0) (pi,3) } ;
        \addplot[name path=B, mark=none, thick, domain=2*pi:3*pi] {sin(deg(x))+1 + sin(deg(2*x))/2};
        \addplot [name path=B1,
            color=gray, thick, dashed]
            coordinates {(2*pi,0) (2*pi,3) } ;
        \addplot[name path=C, mark=none, thick, domain=3*pi:4*pi] {3};
        \addplot [name path=C1,
            color=gray, thick, dashed]
            coordinates {(3*pi,0) (3*pi,3) } ;      
    \end{axis}
    \end{tikzpicture} 
  \caption{Semi-suave descontínua}
  \label{fig:semiDesc}
\end{subfigure}
\caption{Os dois tipos de funções semi-suaves}
\label{fig:test}
\end{figure}


\begin{definicao}
    Uma função $f(x)$ é \textbf{semi-suave contínua} se possuir
    um número finito de ``pontas" em um intervalo $[a,b]$.
\end{definicao}

\begin{definicao}
    Uma função $f(x)$ é \textbf{semi-suave descontínua} se possuir
    um número finito de descontinuidades em um intervalo $[a,b]$.
\end{definicao}



A função da Figura \ref{fig:semiCont} representa uma função semi-suave
contínua, e a Figura \ref{fig:semiDesc} representa uma função dita 
semi-suave descontínua.%9
\input{./src/10-criterioConvergencia.tex}%10
\input{./src/11-funcaoPar.tex}%11
\chapter{Séries de senos e cossenos}

    Seja $f(x)$ uma função \textbf{par} definida no intervalo $[-\pi,\pi]$, ou então
    uma função periódica.\\
    
    A função $h(x) = cos(nx)$, para $n \in \mathbb{N}$, é uma 
    função par, então pela definição \ref{def:opPar}, a função $g(x) = f(x)cos(nx)$ 
    também é par.\\
    
    Por outro lado, a função $h(x) = sen(nx)$, para $n \in \mathbb{N}^*$,
    é ímpar, e assim, a função $g(x) = f(x)sen(nx)$ é também ímpar pela definição 
    \ref{def:opPar}.\\

    Portanto, usando as funções \ref{eq:65}, \ref{eq:intPar} e \ref{eq:intImpar}, temos 
    que os coeficientes de Fourier da função \textbf{par} $g(x)$ são:

    \begin{equation}
    \label{eq:121}
        \begin{split}
            a_n = &\dfrac{1}{\pi}\int_{-\pi}^{\pi}f(x)cos(nx) dx = \dfrac{2}{\pi}\int_{0}^{\pi} f(x)cos(nx) dx \\
            b_n = &\dfrac{1}{\pi}\int_{-\pi}^{\pi}f(x)sen(nx) dx = 0
        \end{split}
    \end{equation}

    Com isso, podemos afirmar que a série de Fourier de funções pares contém apenas $cossenos$, i.e.

    \begin{equation}
        f(x) ~ \dfrac{a_0}{2} + \sum\limits_{n=1}^{\infty}a_n cos(nx),
    \end{equation}

    onde os coeficientes $a_n$ são dados pela fórmula \ref{eq:121}.\\

    Agora, considere outra situação. A mesma função $f(x)$ agora é uma função \textbf{ímpar}.\\

    Sendo assim, como a função $h(x) = cos(nx)$ é uma função par, a função $g(x) = f(x)cos(nx)$
    é uma função ímpar, pela definição \ref{def:opPar}.\\
    
    Por outro lado, sendo $h(x) = sen(nx)$ uma função ímpar, a função $g(x) = f(x)sen(nx)$ é par, pela mesma 
    definição \ref{def:opPar}.\\
    
    Portanto, usando as funções \ref{eq:65}, \ref{eq:intPar} e \ref{eq:intImpar}, temos 
    que os coeficientes de Fourier da função \textbf{ímpar} $g(x)$ são:

    \begin{equation}
    \label{eq:122}
        \begin{split}
            a_n = &\dfrac{1}{\pi}\int_{-\pi}^{\pi}f(x)cos(nx) dx = 0 \\
            b_n = &\dfrac{1}{\pi}\int_{-\pi}^{\pi}f(x)sen(nx) dx = \dfrac{2}{\pi}\int_{0}^{\pi} f(x)cos(nx) dx 
        \end{split}
    \end{equation}

    Com isso, podemos afirmar que a série de Fourier de funções ímpares contém apenas $senos$, i.e.

    \begin{equation}
        f(x) ~ \sum\limits_{n=1}^{\infty}b_n sen(nx),
    \end{equation}

    onde os coeficientes $b_n$ são dados pela fórmula \ref{eq:122}.

    Um problema que surge frequentemente é quando se expande em série de cossenos
    ou série de senos uma função absolutamente integrável $f(x)$ definida no
    intervalo $[0,\pi]$. 

    Neste caso, para expandir em série de cossenos, basta fazer uma extensão $par$ 
    do intervalo $[0,\pi]$ para o intervalo $[-\pi,0]$, como mostra a Figura 
    \ref{fig:extPar}. Computacionalmente falando, não é necessário realizar a extensão
    par [NÃO DIZ PORQUÊ].

    Por outro lado, para expandir em série de senos, basta fazer uma extensão \textit{ímpar}
    do intervalo $[0,\pi]$ para o intervalo $[-\pi,0]$, como mostra a Figura 
    \ref{fig:extImpar}.

\begin{figure}[H]
\centering
\begin{subfigure}{.5\textwidth}
  \centering
    \begin{tikzpicture}
    \begin{axis}[
        Axis Style,
        xmin=-4,
        xmax=4,
        ymin=-3,
        ymax=3,
        width=7cm,
        ticks=none]
    \addplot [mark=none, thick, domain=0:pi] {x^2/3 + 1};
    \addplot [mark=none, dotted, domain=-pi:0] {x^2/3 + 1};
    \label{senoide}
    \end{axis}
    \end{tikzpicture}
  \caption{Extensão par em pontilhado}
  \label{fig:extPar}
\end{subfigure}%
\begin{subfigure}{.5\textwidth}
  \centering

    \begin{tikzpicture}
    \begin{axis}[
        Axis Style,
        xmin=-4,
        xmax=4,
        ymin=-3,
        ymax=3,
        width=7cm,
        ticks=none]
    \addplot [mark=none, dotted, domain=-pi:0] {-x^2/3 - 1};
    \addplot [mark=none, thick, domain=0:pi] {x^2/3 + 1};
    \label{senoide}
    \end{axis}
    \end{tikzpicture}
  \caption{Extensão ímpar em pontilhado}
  \label{fig:extImpar}
\end{subfigure}
\label{fig:extParImpar}
\end{figure}%12
\input{./src/13-exemplos.tex}%13
\input{./src/14-complexForm.tex}%14
\input{./src/15-formaGenerica.tex}
\chapter{Transformada de Fourier}


A transformada de Fourier é um conceito simples, porém que requer bastante
atenção. Assim como qualquer outro conceito matemático, a quantidade de 
detalhes nos faz pensar que é algo de outro planeta. Nesta sessão, quero
simplificar um pouco o que é a transformada de Fourier para que seja
mais compreensível e, ao final, explicar em termos formais assustadores.


\section{Analogia}
Até agora vimos extensivamente o que é expandir uma função $f(x)$ periódica
ou apenas definida em um intervalo, em uma série de Fourier. 

Agora, vamos deixar mais claro o que é a \textit{transformada de Fourier}.
Para isso, vamo fazer uma analogia para se entender a idéia e assim,
entraremos em mais detalhes formais.\\

Então, vamos imaginar uma sopa. Isso mesmo, uma sopa.\\

Esta sopa é composta de ingredientes. Para efeitos práticos, vamos supor
que toda sopa tenha um número finito de ingredientes e que você saiba 
quais são eles.\\

Dito isso, vamos imaginar que um belo dia, você experimentou \textbf{A SOPA},
tão saborosa e cremosa que fez você queer descobrir como foi feita.\\

Pois bem, sabemos do que a sopa pode ser composta, mas não sabemos quanto de cada
ingrediente aquela sopa possui.\\

Então, se possuirmos um jeito de ``filtrar" os ingredientes e obter apenas 
suas quantidades, conseguiríamos fazer a mesma sopa! \\

Bom, para isso ser possível, vamos criar um \textit{filtro} para cada 
ingrediente e despejar a sopa em cada um desses filtros. Se despejarmos 
a sopa em cada um dos filtros, um cálculo simples de quanto tinha de massa 
antes e depois, conseguimos obter quanto tinha no total de cada ingrediente.\\

É assim que a transformada de Fourier funciona. Podemos dizer que a sopa
está para um sinal, assim como os ingredientes estão para os harmonicos 
(Definição \ref{def:harmonico}, do Cap. \ref{cap:harm}) que compõe o 
sinal. Ou seja, a transformada de Fourier é, basicamente, \textit{transformar} 
um sinal em diversos harmonicos. \\

Vamos com calma, primeiro temos que associar com aquilo que já vimos.\\

Se supormos que $f(x)$ é uma função que representa um sinal, podemos 
afirmar que ele é composto de diversos harmonicos. Como já vimos antes,
podemos calcular o coeficiente de cada harmonico separadamente, dado
que a função satisfaça todos os requisitos (são ortogonais, etc NÃO ME 
LEMBRO DE TODOS).

\end{document}